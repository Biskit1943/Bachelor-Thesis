% !TeX root = ../bachelor.tex
\paragraph{stubble}\label{python-tools:stubble}\mbox{}
\newline
Stubble zählt zwar zu den \gls{mock}ing Tools, jedoch werden hier \Glspl{stub} gegeben.
So lässt sich ein Objekt durch ein selbst geschriebenes Objekt ersetzen.

Angenommen man hat ein Objekt \lstinline{class Original} welches eine Abhängigkeit
zu dem zu testenden Objekt darstellt, so lässt sich mit stubble ein Objekt \lstinline{class Fake}
erstellen welches einen \gls{decorator} besitzt \lstinline{@stubclass(Original)}. Dadurch
wird überprüft, dass alle Methoden aus \lstinline{class Fake} der Signatur derer aus
\lstinline{class Original} entsprechen. Wenn \lstinline{class Fake} von \lstinline{class Original}
erbt, werden nur die Funktionen überschrieben die in \lstinline{class Fake} definiert wurden.

Um allerdings kleine Helfer zu \lstinline{class Fake} hinzu zu fügen, müssen diese mit
\lstinline{@exempt} annotiert werden. Ist das objekt von \lstinline{class Original} bereits
initialisiert, so lässt es sich an \lstinline{class Fake} übergeben, wenn diese von
\lstinline{reahl.stubble.Delegate} erbt, so lässt sich zur Laufzeit ein Objekt durch
einen \Gls{stub} ersetzen. Als letztes Hauptfeature wird die Möglichkeit geboten
\Glspl{stub} mit setuptools zu verwenden.
\newline
\\
Die Hauptfeatures von stubble sind jedoch die vorgefertigten \Glspl{stub}. So werden
dem Entwickler folgende \Glspl{stub} geboten:
\begin{itemize}
    \item \href{https://www.reahl.org/docs/4.0/devtools/stubble.d.html\#systemoutstub}{SystemOutStub}\footnote{https://www.reahl.org/docs/4.0/devtools/stubble.d.html\#systemoutstub}
    \subitem Ein Stub, der den Standard Out ersetzt
    \item \href{https://www.reahl.org/docs/4.0/devtools/stubble.d.html\#callmonitor}{CallMonitor}\footnote{https://www.reahl.org/docs/4.0/devtools/stubble.d.html\#callmonitor}
    \subitem Ein Stub, zum überprüfen welche Methoden wann, wie und wie oft aufgerufen wurde
\end{itemize}

Diese \Glspl{stub} lassen sich als \glspl{context} einsetzen, wie auch ein weiteres Feature von stubble.
\lstinline{replaced} lässt den Entwickler eine Methode oder Funktion durch eine andere ersetzen,
solange der \glspl{context} aktiv ist.

Des weiteren bietet stubble einige Experimentelle Features die, nach \cite{reahl.stubble:4.0} eher
weniger für den praktischen Einsatz geeignet sind, aber dennoch interessante Features darstellen.
So ist es möglich \lstinline{class Fake} von \lstinline{reahl.stubble.Impostor} erben zu lassen,
wodurch jede Instanz von \lstinline{class Fake} eine Instanz von \lstinline{class Original} wäre,
wie in Listing \ref{listing:reahl.stubble:impostor} zu erkennen ist.
Ein weiteres experimentelles Feature ist das ersetzen eines bereits bestehenden Objekts
(\lstinline{Delegation}). Dies wurde bereits in den Hauptfeatures beschrieben, da es als sehr
praktisch an zu sehen ist. Lediglich die Instanz Variablen eines Objektes machen hierbei Probleme, da
diese nicht mit einem \Gls{stub} ersetzt werden können. So werden auch Objekt variablen die von
Originalen Methoden gesetzt oder verändert werden nicht im \Gls{stub} gesetzt.
Aus diesem Grund wird \lstinline{Delegation} als Experimentelles Features gesehen, da ein Bug, der
durch dieses Feature entstanden ist schwer zu finden ist.

Zusammenfassend ist stubble ein stabiles Tool, das dem Entwickler diverse Möglichkeiten an die Hand
gibt Objekte vor ihrer Instanziierung als auch danach durch einen \Gls{stub} zu ersetzen.

Im Bezug auf die Anwendbarkeit für TDD ist stubble demnach ein Tool das durchaus in Betracht gezogen
werden kann, auch die Abhängigkeiten halten sich mit einer
(\href{https://github.com/benjaminp/six}{six}\footnote{https://github.com/benjaminp/six}) in grenzen.

Die Effizienz mit der ein Entwickler stubble einsetzen kann ist hoch, denn abgesehen von einem
\gls{decorator} benötigt der Entwickler nichts weiter um Objekte zu ersetzen.

Lediglich die Komplexität von stubble ist nicht sehr hoch, zwar wird dem Entwickler alles geboten um
ein Objekt zu ersetzen, jedoch aber auch nicht mehr. Zusätzlich bietet stubble mit seinen
vorgefertigten \Glspl{stub} eine gute schnelle Möglichkeit den Stdout zu ersetzen oder ein Objekt
zu Monitoren. Um jedoch selbst ein Objekt zu ersetzen wird vergleichsweise viel Code benötigt, da
nicht einfach nur der Return Wert neu gesetzt wird sondern die Methode neu geschrieben werden muss.
Dies kann allerdings bei Komplexeren Anforderungen wiederum zu Vorteil werden weshalb sich das ganze
wiederum ausgleicht.

Der Code für den Vergleichstest von stubble ist in Listing \ref{listing:reahl.stubble:example} zu finden.
Mit stubble alleine lässt sich der in Listing \ref{listing:base:my_mock_module} definierte Code nicht
zu 100\% optimal testen. So ist die nicht existierende Methode \lstinline{Helper.help(self)} nicht einfach
zu ersetzen, lediglich wenn man in der \Gls{stub} Klasse die Methode \lstinline{help(self)} schreibt und
sie mit \lstinline{@exempt} annotiert. Jedoch verliert man dadurch die Funktionalität von stubble, welche
überprüft ob die Signatur der Funktionen übereinstimmen. Alternativ hätte man auch einfach eine Klasse
nehmen können die ohne \lstinline{@stubclass} auskommt. Ansonsten Zeigt der Code wie ein recht simples
Tool viel erreichen kann, wenn auch mit viel Code abseits der Tests. Dennoch ging das überschreiben von
\lstinline{os.path.abspath(path)} sehr einfach durch \lstinline{reahl.stubble.replaced}. Sehr einfach
ging auch das abfangen des STDOUTs mithilfe der vorgefertigten \Glspl{stub}.