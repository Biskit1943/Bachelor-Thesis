% !TeX root = ../bachelor.tex
\paragraph{stubble}\label{python-tools:stubble}\mbox{}
\newline
\lstinline{stubble} zählt zwar zu den \gls{mock}ing Tools, jedoch werden hier
nur \Glspl{stub} als Funktionalität gegeben. Dennoch ist es möglich, mithilfe 
eines der vorgefertigten \Glspl{stub}, \gls{mock}ing Funktionalität zu erhalten.
\newline

Angenommen ein Objekt \lstinline{class Original} hat eine Abhängigkeit zu dem zu
testenden Objekt. Mit \lstinline{stubble} lässt sich ein Objekt
\lstinline{class Fake} erstellen, welches einen \gls{decorator} besitzt
\lstinline{@stubclass(Original)}. Dadurch stellt \lstinline{stubble} sicher,
dass alle Methoden aus \lstinline{class Fake} der Signatur derer aus
\lstinline{class Original} entsprechen. Zusätzlich kann \lstinline{class Fake}
von \lstinline{class Original} erben, wodurch nur die Funktionen überschrieben
werden, die in \lstinline{class Fake} definiert wurden.

Um Helfermethoden zu \lstinline{class Fake} hinzu zu fügen, müssen diese mit
\lstinline{@exempt} annotiert werden. Ist das Objekt von
\lstinline{class Original} bereits initialisiert, so lässt sich dieses an
\lstinline{class Fake} übergeben, wenn diese von 
\lstinline{reahl.stubble.Delegate} erbt. Dadurch ist es möglich zur Laufzeit
ein Objekt durch einen \Gls{stub} ersetzen. Als weiteres Feature wird die
Möglichkeit geboten, \Glspl{stub} mit \lstinline{setuptools} zu verwenden um
Installationen zu \gls{stub}ben.

Die Hauptfeatures von stubble sind jedoch die vorgefertigten \Glspl{stub}. So
werden dem Entwickler folgende \Glspl{stub} geboten:
Zum einen der
\href{https://www.reahl.org/docs/4.0/devtools/stubble.d.html\#systemoutstub}{SystemOutStub}\footnote{https://www.reahl.org/docs/4.0/devtools/stubble.d.html\#systemoutstub},
der den Standard Out ersetzt und zum anderen der
\href{https://www.reahl.org/docs/4.0/devtools/stubble.d.html\#callmonitor}{CallMonitor}\footnote{https://www.reahl.org/docs/4.0/devtools/stubble.d.html\#callmonitor},
der dazu dient zu überprüfen, welche Methoden wann, mit welchen Parametern und
wie häufig aufgerufen wurden.
Diese \Glspl{stub} lassen sich als \Glspl{context} einsetzen, wodurch es
leichter wird, Code mit diesen zu schreiben. Ein weiterer \Glspl{context} ist
\lstinline{replaced}. Dieser lässt den Entwickler eine Methode oder Funktion
durch eine andere ersetzen, wodurch solange der \Gls{context} aktiv ist, die
neue Methode/Funktion genutzt wird.

Des Weiteren bietet stubble einige experimentelle Features, die jedoch eher
weniger für den praktischen Einsatz geeignet sind (\cite{reahl.stubble:4.0}),
aber dennoch interessante Features darstellen. So ist es möglich,
\lstinline{class Fake} von \lstinline{reahl.stubble:Impostor} erben zu lassen,
wodurch jede Instanz von \lstinline{class Fake} eine Instanz von
\lstinline{class Original} ist. Das Beispiel dazu ist in Listing
\ref{listing:reahl.stubble:Impostor} zu finden.

Ein weiteres experimentelles Feature ist das Ersetzen eines bereits bestehenden
Objekts (\lstinline{Delegation}). Dies wurde bereits in den Hauptfeatures
beschrieben, da es als sehr praktisch an zu sehen ist. Lediglich die
Instanz-variablen eines Objektes machen hierbei Probleme, da diese nicht mit
einem \Gls{stub} ersetzt werden können. So werden auch Objektvariablen, die von
originalen Methoden gesetzt oder verändert werden, nicht im \Gls{stub} gesetzt.
Aus diesem Grund wird \lstinline{Delegation} als experimentelles Features
gesehen, da ein Bug, der durch dieses Feature entstanden ist, schwer zu finden
ist (\cite{reahl.stubble:4.0}).

Mit stubble alleine lässt sich dieser in Listing 
\ref{listing:base:my_mock_module} definierte Code nicht zu 100\% optimal testen
(vgl. Listing \ref{listing:reahl.stubble:example}). So ist die nicht
existierende Methode \lstinline{Helper.help(self)} nicht ohne Probleme zu
ersetzen. Dies gelingt, lediglich, wenn in der \Gls{stub} Klasse die Methode
\lstinline{help(self)} definiert und sie mit \lstinline{@exempt} annotiert
wird. Jedoch verliert man dadurch die Funktionalität von stubble, welche
überprüft ob die Signatur der Funktionen übereinstimmen. Alternativ wäre es
möglich eine Klasse zu nehmen, die ohne die \lstinline{@stubclass} auskommt.
Ansonsten zeigt der Code, wie ein recht simples Tool viel erreichen kann, wenn
auch mit viel Code abseits der Tests. Dabei ließ sich
\lstinline{os.path.abspath(path)} ohne Komplikationen mit Hilfe von
\lstinline{reahl.stubble.replaced} ersetzen. Das Abfangen des STDOUTs mit Hilfe
der vorgefertigten \Glspl{stub} erwies sich als unkompliziert.
\newline

Im Bezug auf die Anwendbarkeit für TDD ist stubble demnach ein Tool, das
durchaus in Betracht gezogen werden kann. Auch die Abhängigkeiten halten sich
mit einer Abhängigkeit
(\href{https://github.com/benjaminp/six}{six}\footnote{https://github.com/benjaminp/six})
in Grenzen.

Die Effizienz mit der ein Entwickler stubble einsetzen kann ist hoch, denn
abgesehen von einem \gls{decorator} benötigt der Entwickler nichts weiter um
Objekte zu ersetzen.

Lediglich die Komplexität von stubble ist geringer. Zwar wird dem Entwickler
alles geboten um ein Objekt zu ersetzen, jedoch aber auch nicht mehr.
Zusätzlich bietet stubble mit seinen vorgefertigten \Glspl{stub} eine gute
schnelle Möglichkeit den \lstinline{stdout} zu ersetzen oder ein Objekt zu
überwachen. Um jedoch selbst ein Objekt zu auszutauschen, wird vergleichsweise
viel Code benötigt, da nicht nur der Rückgabewert neu gesetzt wird,
sondern die Methode neu geschrieben werden muss. Dies kann allerdings bei
komplexeren Anforderungen wiederum zum Vorteil werden. Durch die
\Glspl{annotation} ist der Code, der geschrieben wird, sehr übersichtlich und 
gut
gegliedert.