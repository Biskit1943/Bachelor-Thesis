% !TeX root = ../bachelor.tex
\subsection{Test-driven development}\label{einleitung:tdd}
Das Test-driven development (TDD) wurde durch Kent Beck, einem Amerikanischen 
Softwareingenieur und Autor bekannt. Dieser beschreibt in seinem Buch "`Test 
Driven Development: By Example"' wie bei der Anwendung von TDD vorgegangen 
werden soll. Demnach sieht der Entwicklungsprozess folgendermaßen aus 
(\cite{beck:tdd}):
\newline
\\
Zunächst sollte geklärt sein, welche Funktionen erfüllt werden müssen. Anhand
dieser beginnt dann die Anwendung von TDD. Dazu wird der folgende Zyklus
beschrieben, der sich so lange wiederholt, bis alle Features implementiert
und getestet wurden. Der Erste Schritt dazu ist das Schreiben eines Tests, der 
neue Funktionalität oder Verbesserung zu einer bestehenden Unit hinzufügt. 
Dieser Test sollte möglichst kurz und aussagekräftig sein. Dazu muss der 
Entwickler die Spezifikationen und Anforderungen des Features genau verstehen 
um den Test wirklich effektiv schreiben zu können. Ein Entwickler, der die 
Anforderung nicht ganz versteht, ist nicht in der Lage einen Test zu 
schreiben, welcher alle Bereiche abdeckt. Der Entwickler, der den Test 
geschrieben hat, muss jedoch nicht auch der Entwickler sein, der die dazu 
passende Funktionalität implementiert.
\\
Als nächstes werden alle Tests die bereits bestehen, sowie der Neue
durchgeführt um zu sehen, ob der neue Test fehlschlägt. Dies bestätigt unter
anderem auch, dass der neue Test die anderen Tests nicht beeinflusst und auch
ohne neuen Code, der Funktionalität implementiert, nicht bestanden wurde. 
Dadurch kann der Entwickler ausschließen, dass ein Test immer korrekt ist, 
wodurch das Vertrauen des Entwicklers in den Test gesteigert wird.
\\
Der nächste Schritt ist die Implementierung des Codes, der getestet werden soll.
Dabei ist unwichtig, wie der Code geschrieben wird. Das bedeutet, dass es
erlaubt ist "`schmutzigen"' (Code der nicht den gefordertem Standard 
entspricht) Code zu produzieren, welcher so nicht akzeptiert werden würde. Die 
Hauptsache dabei ist, dass der Test bestanden wird. Die Codequalität spielt 
hier keine Rolle, da sie in Punkt Fünf überarbeitet wird. Der Entwickler darf 
in diesem Prozess auch keinen weiteren Code, der nicht notwendig ist um den 
Test zu bestehen, hinzufügen. Dies verhindert, dass Code geschrieben wird, der 
keine zugehörigen Tests hat. Ist die gesamte Logik implementiert, werden alle 
Tests durchgeführt und überprüft, ob diese erfolgreich waren. Sollte dies nicht 
der Fall sein, muss Punkt Drei wiederholt werden, welcher die Implementierung 
der Funktionalität ist.
\\
Sind alle Tests bestanden wird im finalen Schritt der neue Code aufgeräumt.
Dieser Punkt ist sehr wichtig, da in Punkt Drei der Code nur funktionieren 
muss. Dadurch kann es sein, dass hier Code entstanden ist, der nicht den 
qualitativen Anforderungen entspricht. Sämtliche Objekte sollten einen 
aussagekräftigen Namen erhalten und der Code sollte, sofern dies nötig ist an 
einen Ort verlegt werden, der seiner logischen Aufgabe entspricht. 
Duplikationen müssen entfernt werden und nach jeder Aufräumaktion sollten die 
Tests noch einmal ausgeführt durchlaufen, um zu verifizieren, dass alles noch 
funktioniert.
\newline
\\
Dieser Zyklus wird nun von vorne so lange wiederholt, bis jedes Feature 
implementiert ist und die Software als fertig gestellt gilt. Die Schritte, die 
dabei gemacht werden sollen, sollten möglichst klein sein, um eine höhere 
Testabdeckung zu erreichen.