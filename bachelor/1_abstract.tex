% !TeX root = ../bachelor.tex
\begin{abstract}
Da die Anforderungen an Software im laufe der Zeit stark angestiegen sind,
haben sich einige Technologien Entwickelt, die dazu verwendet werden diese
Anforderungen zu gewährleisten. Eine dieser Technologien ist das Test-driven
development. Diese Arbeit befasst sich mit dieser Technologie im Bezug zu den
aktuellen Test-Tools der Programmiersprache Python und analysiert diese auf
ihre Tauglichkeit im Bezug zum Test-driven development.
\newline

Dabei werden verschiedenen Arten von Tools untersucht, die unit-testing-,
mock-testing- und fuzz-testing Tools. Diese werden werden anhand ihrer
Anwendbarkeit, Effizient, Komplexität und Erweiterbarkeit beschrieben und
analysiert, um mit Hilfe eines Vergleichs derer einen Leitfaden für das am
besten für Test-driven development geeignete Tool zu bieten. Das daraus
Resultierende Ergebnis soll aufzeigen, welche Tools miteinander die Beste Wahl
anhand der Kriterien darstellen.
\newline

\textbf{TODO: Ergebnis}
\end{abstract}