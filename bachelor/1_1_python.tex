% !TeX root = ../bachelor.tex
\subsection{Die Programmiersprache Python}\label{einleitung:python}

Python ist eine dynamisch typisierte Programmiersprache, sie ist Open Source
unter der Python Software Foundation License
(PSFL\footnote{https://docs.python.org/3/license.html}). Eine dynamisch
typisierte Programmiersprache bestimmt zur Laufzeit, welchem Typ eine Variable
angehört. Dies ist möglich, da Python eine interpretierte und keine kompilierte
Sprache ist. Interpretierte Sprachen werden, zur Laufzeit, in Maschinencode
umgewandelt, während kompilierte Sprachen davor umgewandelt werden. Beides hat
seine Vor- und Nachteile auf diese hier aber nicht weiter eingegangen werden
sollen, da dies nicht die Aufgabe dieser Arbeit ist.

Python unterstützt verschiedene Programmier-Paradigmen, wie die funktionale
Programmierung oder die objektorientierte Programmierung. Dadurch, dass Python
interpretiert ist, wird die Sprache auch oft als Skript Sprache genutzt.

Das Hauptmerkmal der Sprache ist die Syntax, die dabei verwendet werden muss.
Bei Python werden keine geschweiften Klammern genutzt, stattdessen werden
Einrückungen in Form von vier Leerzeichen verwendet. Dadurch entsteht ein
leicht zu lesender Code, der gerade für Anfänger besser zu verstehen ist.