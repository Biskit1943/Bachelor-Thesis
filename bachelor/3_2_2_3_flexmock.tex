% !TeX root = ../bachelor.tex
\paragraph{flexmock}\label{python-tools:flexmock}\mbox{}
\newline
\lstinline{flexmock} ist ein weiteres Tool, dass von der Ruby Community
inspiriert wurde. Dabei diente das gleichnamige Tool
\href{https://github.com/jimweirich/flexmock}{flexmock}\footnote{https://github.com/jimweirich/flexmock}
als Inspiration. "`Es ist jedoch nicht das Ziel von Python flexmock, ein
Klon der Ruby-Version zu sein. Stattdessen liegt der Schwerpunkt auf der
vollständigen Unterstützung beim Testen von Python-Programmen und der möglichst
unauffälligen Erstellung von gefälschten Objekten."'\footnote{Übersetzt aus dem Englischen}
(\cite{flexmock:docs:0.10.3}). Die Features von \lstinline{flexmock} sind denen
von \nameref{python-tools:mocktest} sehr ähnlich. So bildet \lstinline{flexmock}
seine Tests, \glspl{stub} und \glspl{mock} wie ausgesprochen dar.
\newline

\lstinline{flexmock} bietet dem Nutzer einiges an Funktionalität, auch wenn es
sich erst in der Version \lstinline{0.10.4} (Stand 22. April 2019) befindet. So
lässt sich bereits ein \Gls{stub} für Klassen, Module und Objekte mithilfe von
\lstinline{flexmock()} einrichten. Damit ist es möglich alles Notwendige für
ein erfolgreiches Durchlaufen der Tests zu erstellen.

Auch \gls{mock}ing ist kein Problem mit \lstinline{flexmock}. Mit der gleichen
Funktion, mit der \Glspl{stub} erstellt werden, lassen sich
\Glspl{mock} erstellen. Dadurch ist es möglich, einen \Gls{stub} als einen
\Gls{mock} und umgekehrt zu verwenden. Daraus resultierend besteht für den
Entwickler die Möglichkeit zu überprüfen wie oft etwas aufgerufen wurde, welche
Parameter verwendet wurden, welcher Rückgabe Wert zu erwarten ist und/oder
welche Exception zu geworfen werden soll.

Zusätzlich ist es möglich, originale Funktionen eines Objektes zu
\gls{mock}en. Dadurch enthält der Entwickler die Möglichkeit
zu überprüfen, ob eine unveränderte Funktion oder Methode bestimmten
Anforderungen Entspricht. Das Interface dafür ist das Selbe wie bei einem
\Gls{stub}.
\lstinline{flexmock} bietet die Möglichkeit einen leeren \Gls{mock} zu
erstellen und diesem eine beliebige Methoden hinzu zu fügen. Dies ist jedoch
nicht bei Objekten möglich, deren Methoden bereits fest stehen. Deswegen kann
eine neue Methode zu einem gemockten Objekt nicht hinzugefügt werden.
\newline

Ein besonders interessantes Feature ist die Überprüfung des Ergebnisses nach
Typ. Dabei wird \lstinline{.and_return()} nur der Typ mitgegeben, der erwartet
wird. So würde
\\% Code geht über Zeile hinaus
\lstinline{.and_return((int, str, None)} jedes \lstinline{Tuple}
zulassen, das als ersten Wert einen \lstinline{int} hat, als zweiten einen
String und als drittes None. Dabei kann selbstverständlich auf jede beliebige
Instanz überprüft werden, so auch auf eigene Klassen. Dies ist allerdings nur
bei \lstinline{.should_call()} und bei \lstinline{.should_receive()} nicht
möglich.

Das letzte in der Dokumentation erwähnte Feature ist die Ersetzung von Klassen
noch vor ihrer Instanziierung (\cite{flexmock:docs:0.10.3}). Dabei gibt es
mehrere Ansätze, die verschiedene Ausgänge haben. Der erste Ansatz ist
auf Modul Level. Dabei wird im Modul die Klasse mit einem Objekt, das
entweder ein \lstinline{flexmock} oder ein Beliebig anderes sein kann, ersetzt.
Nachteil dieser Methode ist, dass durch das ersetzen der Klasse durch eine
Funktion, Probleme herbeigeführt werden. Um dem entgegen zu wirken, kann
alternativ \lstinline{.new_instance(obj)} verwendet werden, welches beim
Erstellen das Objekt zurück gibt, welches \lstinline{.new_instnce()} übergeben
wurde. Ein weiterer Lösungsweg könnte sein, die \lstinline{__new__()} Methode
der Klasse mit \lstinline{.should_receive('__new__').and_return(obj)} zu
ersetzen, was im Endeffekt das gleiche ist, nur verständlich ausgeschrieben.
\newline

Der Code zu Listing \ref{listing:base:my_mock_module} wurde mit Hilfe des Codes
von \lstinline{flexmock} getestet. Dabei ergaben sich die eben aufgelisteten
Eigenschaften (vgl. Listing \ref{listing:flexmock:example}).
\lstinline{flexmock} überzeugt dabei mit seiner Einfachheit des Aufsetzens. In
der \lstinline{setUp()} wird global im Modul 
\lstinline{my_package.my_mock_module} die Klasse \lstinline{NotMocked} ersetzt.
Dadurch besteht die Möglich in jedem Testfall eine Annahme zu tätigen, die
auf alle Objekte der Klasse \lstinline{NotMocked} zutreffen. Der Aufwand diesen
Code zu schreiben war demnach sehr gering. Lediglich das Erstellen einer
Methode, die im originalen Objekt nicht existiert, war unmöglich. Aus diesem
Grund ist der Code in \lstinline{test_call_helper_help()} nicht korrekt und
wirft eine Exception. Diese kann in Listing \ref{listing:flexmock:exception}
eingesehen werden. Das Problem wurde mit Hilfe des Erstellers des Tools
versucht zu lösen, jedoch konnte bis zur Abgabe dieser Arbeit keine
funktionierende Lösung erarbeitet werden.
\newline

Die Anwendbarkeit von \lstinline{flexmock} ist ausgezeichnet, da es zunächst
keine weiteren Abhängigkeiten, außer sich selbst, installiert und mit allen
Test-runnern kompatibel ist. Im Bezug auf TDD ist \lstinline{flexmock} sehr zu
empfehlen. Dem Entwickler werden zahlreiche Funktionalität geboten,
\Glspl{stub} oder \Glspl{mock} zu erstellen und zu verwenden. Dabei lässt sich
von der Aufrufanzahl bis zu den verwendeten Parametern, alles überprüfen.
Lediglich das Setzen einer nicht implementieren Methode war mit
\lstinline{flexmock} nicht möglich.

Im Aspekt Effizienz ist \lstinline{flexmock} ein Parade-Beispiel für den 
benötigten Aufwand, Tests zu implementieren. Der Entwickler muss selbst wenig
von \gls{mock}ing, verstehen um die Funktionalität von \lstinline{flexmock}
nutzen zu können, da sich der Code wie eine Aussage liest. Das Gleiche gilt
natürlich auch für die \Glspl{stub}.

Genauso ist die Komplexität von \lstinline{flexmock} ausgezeichnet.
\lstinline{flexmock} bieten für den Entwickler zahlreiche sinnvolle Aspekte und
darüber hinaus  ein Interface, welches das Schreiben von übersichtlichen Code
ermöglicht. Dies liegt vor allem an der deskriptiven Programmierung, die von
\lstinline{flexmock} geboten wird. Auch die Unübersichtlichkeit des Codes hält 
sich dabei in Grenzen.