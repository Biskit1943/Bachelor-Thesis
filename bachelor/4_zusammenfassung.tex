% !TeX root = ../bachelor.tex
\subsection{Zusammenfassung}\label{zusammenfassung}

Um einen schnellen gesamt Überblick zu schaffen, werden unter diesem Punkt die
einzelnen Vor- und Nachteile der Tools noch einmal zusammengefasst. Dabei wird
auf die in Kapitel \nameref{python-tools} genannten Aspekte referenziert.

Die Zusammenfassende Wertung wird mit einem Wert zwischen Eins und Zehn
bestimmt, wobei Eins der schlechteste Wert ist und Zehn der Beste. Diese Wertung
erfolgt aus den Analysen des jeweiligen Tools in den vorherigen Kapiteln. Um die
Zusammenfassung besser zu gliedern werden die Tools nicht mehr zwischen STDLIB
und externe Pakete unterschieden, sondern nach Funktionalität gegliedert (unit-,
\gls{mock}- und \gls{fuzz}-testing Tools). Zu jeder Wertung erfolgt eine kleine
Zusammenfassung der Analyse, um die Wertung nachvollziehbar zu gestalten.

%===============================================================================
% Unit-testing Tools
%===============================================================================
\subsubsection{Unit-testing Tools}\label{zusammenfassung:unit}

\paragraph{unittest}\label{zusammenfassung:unit:unittest}\mbox{}
Als unit-testing Tool kann \lstinline{unittest} 
\begin{itemize}
    \item \underline{Anwendbarkeit} (10/10):\newline
    \lstinline{unittest} bietet dank dem submodul \lstinline{unittest.mock}
    alles an Funktionalität um TDD einsetzen zu können. Da es in der STDLIB ist,
    werden keine externen Abhängigkeiten benötigt.
    
    \item \underline{Effizienz} (7/10):\newline
    Die Vorarbeit, die benötigt ist um Testen zu können hält sich in grenzen.
    Jedoch ist die Auswertung der Tests durch den Monochromen Output nicht
    sehr gut lesbar und bei vielen Fehlern unübersichtlich.
    
    \item \underline{Komplexität} (8/10):\newline
    Die von \lstinline{unittest} bereit gestellten \lstinline{assert} Methoden
    bieten dem Entwickler eine Methode für jede mögliche Behauptung. Abseits
    davon bietet \lstinline{unittest} viel Funktionalität zum aufsetzen einer
    Test Umgebung, wobei der geschriebene Code viel seiner Übersichtlichkeit
    bewahrt.
    
    \item \underline{Erweiterbarkeit} (10/10):\newline
    Viele Tools basieren oder erweitern \lstinline{unittest}, wodurch es möglich
    ist neue Funktionalität zu \lstinline{unittest} hinzu zu fügen. Die Tools,
    die \lstinline{unittest} erweitern bieten meist einen erheblichen Mehrwert
    weshalb hier 10 Punkte vergeben werden.
\end{itemize}

\paragraph{doctest}\label{zusammenfassung:unit:doctest}\mbox{}
\newline
\begin{itemize}
    \item \underline{Anwendbarkeit} (3/10):\newline
    Doctest bietet für sich alleine nicht genug Funktionalität um TDD betreiben
    zu können.
    
    \item \underline{Effizienz} (3/10):\newline
    Ein Entwickler der Python beherrscht, sollte keine Probleme haben
    \lstinline{doctest} anwenden zu können. Der Code der allerdings abseits
    der Test benötigt wird kann sehr groß ausfallen, wenn \Glspl{fixture}
    gewünscht sind. Auch der Output ist schnell unübersichtlich.
    
    \item \underline{Komplexität} (1/10):\newline
    Abgesehen der \lstinline{Option Flags} bietet \lstinline{unittest} keine
    Funktionalität um Tests zu schreiben, der Entwickler kann sich hier
    lediglich der STDLIB oder allen anderen Paketen bedienen die importiert
    werden können.
    
    \item \underline{Erweiterbarkeit} (5/10):\newline
    \lstinline{doctest} selbst ist nicht erweiterbar, kann jedoch als
    Erweiterung für andere Test Tools genutzt werden. So wird von Haus aus
    eine \lstinline{unittest} Integration bereit gestellt. Aus diesem Grund
    werden hier 5 Punkte vergeben, da das Einsatzgebiet von \lstinline{doctest}
    eher als Erweiterung gesehen werden sollte und nicht als Tool für TDD.
\end{itemize}

\paragraph{pytest}\label{zusammenfassung:unit:pytest}\mbox{}
\newline
\begin{itemize}
    \item \underline{Anwendbarkeit} (9/10):\newline
    \lstinline{pytest} bietet alles um TDD betreiben zu können, lediglich die
    der Fakt, dass es ein Externes Paket mit Sechs Abhängigkeiten ist gibt einen
    Punkt Abzug.
    
    \item \underline{Effizienz} (10/10):\newline
    Der Einstieg in \lstinline{pytest} erfolgt ohne Probleme und große
    Vorarbeit. Der Output ist gegliedert und in Farbe, wodurch er leicht zu
    lesen und zu verstehen ist.
    
    \item \underline{Komplexität} (10/10):\newline
    Durch die umfassenden Features von \lstinline{pytest} ist es dem Entwickler
    möglich viel aus seinen Tests heraus zu holen ohne neue Konstrukte lernen zu
    müssen. Der dabei entstehende Code bewahrt auch bei viel Code ein gewisses
    Maß an Lesbarkeit.
    
    \item \underline{Erweiterbarkeit} (10/10):\newline
    Die 618 (Stand 2. April 2019) verfügbaren Erweiterung rechtfertigen die Zehn
    erhaltenen Punkte.
\end{itemize}

%===============================================================================
% Mock-testing Tools
%===============================================================================
\subsubsection{Mock-testing Tools}\label{zusammenfassung:mock}

\paragraph{stubble}\label{zusammenfassung:mock:stubble}\mbox{}
\newline
\begin{itemize}
    \item \underline{Anwendbarkeit} (9/10):\newline
    Die Funktionalität, die zum \gls{stub}en benötigt wird ist mit
    \lstinline{stubble} fast komplett gegeben und die Abhängigkeiten belaufen
    sich auf nur Eine.
    
    \item \underline{Effizienz} (10/10):\newline
    Vorarbeit und viel Code werden mit \lstinline{stubble} nicht benötigt.
    
    \item \underline{Komplexität} (7/10):\newline
    \lstinline{stubble} bietet lediglich die Möglichkeit Objekte zu ersetzen,
    jedoch nicht mehr. Die Vorgefertigten \Glspl{stub} sind allerdings sehr
    nützlich. Da, keine \gls{mock}ing Funktionalität geboten wird gibt es hier
    Punktabzug.
\end{itemize}

\paragraph{mocktest}\label{zusammenfassung:mock:mocktest}\mbox{}
\newline
\begin{itemize}
    \item \underline{Anwendbarkeit} (9/10):\newline
    \lstinline{mocktest} bietet seinem Nutzer alles um Objekte zu \gls{mock}en
    oder zu \gls{stub}en. Zusätzlich erweitert \lstinline{mocktest} unittest
    wodurch es möglich ist Test Umgebungen zu erschaffen. Zusätzlich zu sich
    selbst wird Eine Abhängigkeit benötigt, für die es einen Punkt Abzug gibt.
    
    \item \underline{Effizienz} (10/10):\newline
    Da \lstinline{mocktest} deskriptiv ist, kann es ohne großen Aufwand
    verwendet werden. Der Code der dazu benötigt wird beschränkt sich auf
    ein Minimum.
    
    \item \underline{Komplexität} (10/10):\newline
    An Funktionalität fehlt es dem Tool nicht.

\end{itemize}

\paragraph{flexmock}\label{zusammenfassung:mock:flexmock}\mbox{}
\newline
\begin{itemize}
    \item \underline{Anwendbarkeit} (9/10):\newline
    Im Bezug zu TDD wird dem Anwender fast alles geboten, was benötigt wird und
    das mit nur einer Abhängigkeit, das Paket selbst.
    
    \item \underline{Effizienz} (10/10):\newline
    Die Anwendung von \lstinline{flexmock} ist deskriptiv, weshalb es für einen
    ohne größere Vorkenntnisse anwendbar ist. Der dabei zu schreibende Code ist
    Minimal.
    
    \item \underline{Komplexität} (9/10):\newline
    Das Interface von \lstinline{flexmock} ermöglicht sauberen und
    übersichtlichen Code zu schreiben und bietet dabei einiges an Funktionalität
    um zu \gls{mock}en und zu \gls{stub}en. Punktabzug gab es für die fehlende
    Möglichkeit nicht existierende Funktionen und Methoden zu erstellen.
\end{itemize}

\paragraph{doublex}\label{zusammenfassung:mock:doublex}\mbox{}
\newline
\begin{itemize}
    \item \underline{Anwendbarkeit} (X/10):\newline
    TBD.
    
    \item \underline{Effizienz} (X/10):\newline
    TBD.
    
    \item \underline{Komplexität} (X/10):\newline
    TBD.
\end{itemize}

%===============================================================================
% Fuzz-testing Tools
%===============================================================================
\subsubsection{Fuzz-testing Tools}\label{zusammenfassung:fuzz}

\paragraph{hypothesis}\label{zusammenfassung:fuzz:hypothesis}\mbox{}
\newline
\begin{itemize}
    \item \underline{Anwendbarkeit} (X/10):\newline
    TBD.
    
    \item \underline{Effizienz} (X/10):\newline
    TBD.
    
    \item \underline{Komplexität} (X/10):\newline
    TBD.
    
    \item \underline{Erweiterbarkeit} (X/10):\newline
    TBD.
\end{itemize}