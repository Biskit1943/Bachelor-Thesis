% !TeX root = ../bachelor.tex
\newpage
% Die Listings sollen im Inhanltsverzeichnis auftauchen
\section*{Listings}
\addcontentsline{toc}{section}{Listings}
\fancyhead[L]{Listings} % Kopfzeile links

\lstinputlisting[
    language=Python,
    label=listing:base:my_module,
    caption=Basis Modul zum testen,
    lastline=42
]{analyse/my_package/my_module.py}

\lstinputlisting[
    language=Python,
    label=listing:base:my_mock_module,
    caption=Basis Modul zum testen von mocks,
]{analyse/my_package/my_mock_module.py}

\lstinputlisting[
    language=Python,
    label=listing:base:my_fuzz_module,
    caption=Basis Modul zum testen von fuzz-testing,
]{analyse/my_package/my_fuzz_module.py}
\newpage
%=====================================================================%

\lstinputlisting[
    language=Python,
    label=listing:unittest:example,
    caption=unittest einfaches Beispiel
]{analyse/unittest_/example.py}

\lstinputlisting[
    label=listing:unittest:example_output_success,
    caption=unittest einfaches Beispiel: Output erfolgreich
]{analyse/unittest_/example_output_success.txt}

\lstinputlisting[
    label=listing:unittest:example_output_failure,
    caption=unittest einfaches Beispiel: Output misslungen
]{analyse/unittest_/example_output_failure.txt}
\newpage
%=====================================================================%

\lstinputlisting[
    language=Python,
    label=listing:unittest:advanced,
    caption=unittest my\_module
]{analyse/unittest_/advanced_example.py}
\newpage
%=====================================================================%

\lstinputlisting[
    language=Python,
    label=listing:doctest:example,
    caption=doctest unterscheidet Typen
]{analyse/doctest/example.py}

\lstinputlisting[
    label=listing:doctest:example_output,
    caption=doctest unterscheidet Typen: Output
]{analyse/doctest/example_output.txt}

\lstinputlisting[
    label=listing:doctest:output,
    caption=doctest verbose Output
]{analyse/doctest/output.txt}
\newpage
%=====================================================================%

\lstinputlisting[
    language=Python,
    label=listing:doctest:advanced,
    caption=doctest: my\_module,
    firstline=19
]{analyse/doctest/advanced.py}

\lstinputlisting[
    language=Python,
    label=listing:doctest:advanced_text,
    caption=doctest my\_module: Textdatei
]{analyse/doctest/advanced.txt}
\newpage
%=====================================================================%

\lstinputlisting[
    label=listing:pytest:requirements,
    caption=pytest Abhängigkeiten
]{analyse/py.test/after.requirements}

\lstinputlisting[
    language=Python,
    label=listing:pytest:advanced,
    caption=pytest my\_module
]{analyse/py.test/advanced.py}

\lstinputlisting[
    label=listing:pytest:advanced_output,
    caption=pytest my\_module: Output
]{analyse/py.test/advanced_output.txt}
%=====================================================================%

\lstinputlisting[
    language=Python,
    label=listing:reahl.stubble: Impostor,
    caption=stubble: Impostor
]{analyse/stubble/impostor.py}

\lstinputlisting[
    language=Python,
    label=listing:reahl.stubble my_mock_module,
    caption=stubble my\_mock\_module
    lastline=91
]{analyse/stubble/example.py}
%=====================================================================%

\lstinputlisting[
    language=Python,
    label=listing:mocktest:example,
    caption=mocktest my\_mock\_module
]{analyse/mocktest_/example.py}
%=====================================================================%

\lstinputlisting[
    language=Python,
    label=listing:flexmock:my_mock_module,
    caption=flexmock my\_mock\_module
]{analyse/flexmock_/example.py}
