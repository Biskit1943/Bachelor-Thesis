\section{Methodik zur Analyse von Python Testtools}\label{methodik}
Die zu analysierenden Tools, werden zwischen Standardbibliothek (STDLIB) und 
externen Tools unterschieden. Zusätzlich findet eine Gliederung in 
unit-,\gls{mock}- und \gls{fuzz} Tools statt.

Die unit-testing Tools sind Tools, die Funktionalität zum Testen bereitstellen.
Jedoch wird für TDD weit mehr als nur Tests benötigt. Aus diesem Grund werden
\gls{mock}-testing - und \gls{fuzz} Tools zusätzlich behandelt. Dabei soll
zwischen einer reinen Erweiterung eines Tools und der Erweiterung von allen
Tools unterschieden werden. Ist zum Beispiel ein Tool nur zusammen mit einem
anderen Tool, welches die Funktionalität bereit stellt Tests aus zu führen,
so ist dieses Tool als Erweiterung zu sehen und wird nicht analysiert. Bietet
ein Tool allerdings Funktionen zum Erweitern von verschiedenen Test Tools, so
wird es hier zur  Analyse verwendet. Diese Unterscheidung wird genutzt um
Duplikationen in den einzelnen Vergleichen zu vermeiden und verhindert, dass
Tools mit sehr vielen Erweiterungen den Rahmen dieser Arbeit sprengen.Sollten
sich für ein Tool besonders interessante Erweiterungen finden, so werden diese
in der Analyse des jeweiligen Tools erwähnt und verlinkt.
\newline

Jedes Tool wird anhand folgender Aspekte untersucht. Anwendbarkeit: Bietet das
Tool alles, um TDD betreiben zu können? Bei unit-testing Tools beinhaltet dies
\Glspl{fixture} und \Glspl{mock} während es bei den \gls{mock}- und \gls{fuzz} 
Tools um die verschiedenen Features geht. Sowohl das \Gls{mock}en als auch das
\Gls{stub}en sind für das TDD von großer Bedeutung. Einerseits ist es wichtig
nach zu verfolgen, welches Modul wann und mit welchen Parametern aufgerufen
wurde. Andererseits ist es genau so wichtig Tests ab zu schotten. Dies
geschieht mithilfe der \Glspl{stub}, welche es ermöglichen externe
Abhängigkeiten zu ersetzten um Fehler aus anderen Quellen aus zu schließen 
(\cite{percival:tdd:python}).
Zusätzlich wird in diesem Aspekt auch die benötigten Abhängigkeiten eines
Pakets untersucht, um dieses betreiben zu können. Mehr
Abhängigkeiten führen dazu, dass mehr Entwicklern vertraut werden muss, dass
diese Ihre Pakete aktuell und fehlerfrei halten.
Der zweite Aspekt ist die Effizienz mit der sich ein Tool einsetzen lässt. Dabei
wird darauf geachtet, wie viel Aufwand benötigt wird, das Tool ein zu setzen. Im
Besonderen wird hier auf die Vorarbeit, die benötigt wird um das Tool verwenden
zu können, geachtet. Des weiteren wird in diesem Aspekt bei den unit-testing
Tools die Effizienz, mit der Tests ausgewertet werden können, analysiert.
Als dritter Aspekt wird die Komplexität des Tools bewertet. Dabei wird diese
nicht zwangsläufig als negativ gesehen. Hat ein Tool viel
Funktionalität neben seiner Basis Features, so hat es in diesem Bezug eine hohe
Komplexität, die positiv zu sehen ist. Hingegen wird viel oder
unübersichtlicher Code den das Tool produziert als negativer Komplex gesehen.
Zuletzt wird bei den unit-testing Tools auf die Erweiterbarkeit geachtet. Diese
umfasst Erweiterungen der Entwickler als auch der Community. Bei den mock- und
fuzz-testing Tools, wird dieser Aspekt aus gelassen, da es sich bei diesen Tools
vorwiegend um Erweiterungen zu den unit-testing Tools handelt.
\newline

Zum Vergleich der einzelnen unit-testing Tools untereinander werden diese auf
den in Listing \ref{listing:base:my_module} abgebildeten Code angewendet. Da
sich diese Arbeit auch mit \gls{mock}-testing Tools beschäftigt, wurde auch für
diese ein Modul geschrieben, welches zu testen ist.
Der Code dazu befindet sich in Listing \ref{listing:base:my_mock_module}. Durch
die Anwendung des jeweiligen Tools auf diesen Code, ist es möglich die Aspekte
dieser untereinander zu vergleichen.

Das Modul aus Listing \ref{listing:base:my_module} für die unit-testing Tools
enthält eine selbst geschriebene Implementierung der Funktion \lstinline|pow()|,
welche eine Zahl \lstinline|a| mit einer Zahl \lstinline|b| quadriert. Um die
Komplexität zu erhöhen wurde eine in-memory Datenbank implementiert, welche
eine Items Tabelle enthält. Jedes Item hat eine ID (\lstinline|id|), einen Namen
(\lstinline|name|), einen Lagerplatz (\lstinline|storage_location|) und eine
Anzahl der vorrätigen Items \lstinline|amount|. Dazu besitzt jedes Item eine
Methode \lstinline|do_something()| welche eine externe Funktion/Methode,
die jedoch noch nicht existiert \lstinline|do_something_which_does_not_exist()|,
aufruft.

Für die \gls{mock}-testing Tools wurde eine Klasse geschrieben, welche 
Funktionen besitzt, durch ihren Namen ausdrücken, welche Funktionalität sie 
eigentlich haben sollten. Um jedoch testen zu können, ob das jeweilige Tool 
diese Methode \gls{stub}en kann, gibt nicht jede dieser Methoden das gewünschte 
Ergebnis zurück. Das Tool soll dafür sorgen, dass das gewünschte Ergebnis 
erfolgt. Selbstverständlich ist realer Code viel komplexer als in diesem Listing
(\ref{listing:base:my_mock_module}) dargestellt, jedoch reicht dieser Code aus,
um viele Tools an die Grenzen Ihrer Funktionalität zu bringen. So muss in
\lstinline{call_internal_functin_n_times} überprüft werden, ob sie \lstinline{n}
mal aufgerufen wurde. Auch \lstinline{call_helper_help} wird nicht ohne 
weiteres funktionieren, da die Methode der Klasse \lstinline{Helper} noch nicht 
implementiert wurde. Auch die Methode \lstinline{return_false_filepath}
bringt das ein oder andere Tool an die grenzen seiner Funktionalität, da eine 
externe Methode aus der STDLIB ersetzt werden muss, wobei dies nicht global 
geschehen darf.

Für die Fuzz-testing Tools werden keine Beispiele geschrieben, da diese meist 
auf komplexen Beispielen basieren um Sinn zu ergeben. Stattdessen werden im 
jeweiligen Kapitel kleinere Beispiele genannt, wie das Tool ein zu setzen ist 
oder es wird auf Beispiele aus der jeweiligen Dokumentation verwiesen.

Verfügt ein Tool über keinen Test-runner, so wird der von der STDLIB gestellte 
Runner \lstinline{unittest} verwendet.