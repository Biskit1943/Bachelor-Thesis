% !TeX root = ../bachelor.tex
\subsection{Die Programmiersprache Python}\label{einleitung:python}

Python ist eine dynamisch Typisierte Programmiersprache, sie ist Open Source unter der
Python Software Foundation License (PSFL)\footnote{https://docs.python.org/3/license.html}.
Eine dynamisch Typisierte Programmiersprache bestimmt zur Laufzeit welchem Typ eine Variable
angehört, dies ist möglich, da Python eine Interpretierte und keine Kompilierte Sprache ist.
Interpretierte Sprachen werden während sie ausgeführt werden in Maschinen Code umgewandelt,
während Kompilierte Sprachen vorher umgewandelt werden, beides hat seine Vor- und Nachteile
auf diese hier aber nicht weiter eingegangen werden soll.

Python unterstützt verschiedene Programmier-Paradigmen, wie die Funktionale Programmierung oder
die Objekt Orientierte Programmierung. Dadurch, dass Python interpretiert ist wird die Sprache
auch oft als Skript Sprache genutzt.

Das Hauptmerkmal der Sprache ist die Syntax die verwendet wird, in Python werden keine geschweiften
Klammern genutzt, stattdessen verwendet man Einrückungen in Form von vier Leerzeichen. Dadurch entsteht
ein leicht zu lesender Code der gerade für Anfänger leichter zu verstehen ist.
