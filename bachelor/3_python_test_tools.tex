% !TeX root = ../bachelor.tex
\section{Python Test-Tools}\label{python-tools}

Dieses Kapitel befasst sich mit den von der STDLIB bereitgestellten Test-Tools
sowie denen aus externen Paketen.
Diese werden unter \ref{python-tools:stdlib} und \ref{python-tools:extlib}
zusammengefasst, wobei die Tools aus externen Paketen aufgeteilt werden in
\nameref{python-tools:extlib:python}, \nameref{python-tools:extlib:framework} und
\nameref{python-tools:extlib:sonstige}.
\newline
\\
Jedes Tool wird anhand folgender Aspekte untersucht:
\begin{itemize}
    \item \underline{Anwendbarkeit}:\newline
    Bietet das Tool alles, um TDD betreiben zu können? (Bei TDD
    kann man Tests \gls{mock}en, wodurch sie nicht fehlschlagen.)
    Mit wie vielen Paketen muss das Tool betrieben werden?
    
    \item \underline{Effizienz}:\newline
    Wie viel lässt sich mit diesem Tool möglichst einfach und
    schnell erreichen? Ist besonders viel Vorarbeit notwendig um die Tests
    auf zu setzen oder kann sofort mit dem Schreiben der Tests begonnen
    werden?
    \newline
    Genauso stellt sich die Frage, wie Effizient der Entwickler die Tests
    auswerten kann.
    
    \item \underline{Komplexität}:\newline
    Wie komplex ist das Tool? Das heißt, wie viel Funktionalität
    bietet das Tool dem Entwickler von Haus aus, aber auch wie schwer
    ist es einen Code zu schreiben oder wie schnell wird ein Code unübersichtlich, da
    das Tool viel Code abseits der Tests benötigt.
    
    \item \underline{Erweiterbarkeit}:\newline
    Wie leicht lässt sich das Tool mit anderen Tools erweitern?
    Gibt es vielleicht Erweiterungen der Community für dieses Tool, die sehr
    hilfreich sind?
\end{itemize}

% !TeX root = ../bachelor.tex
\subsection{Tools der Standard Bibliothek}\label{python-tools:stdlib}

Die Standard Bibliothek von Python bietet zwei verschiedene Test-Tools(\cite{wiki.python:PythonTestingToolsTaxonomy}).
Zum einen ist dies
\href{http://pyunit.sourceforge.net/pyunit.html}{unittest}\footnote{http://pyunit.sourceforge.net/pyunit.html}
und zum anderen
\href{https://docs.python.org/3/library/doctest.html}{doctest}\footnote{https://docs.python.org/3/library/doctest.html}.
Diese beiden Tools reichen sind im ihrem Umfang bereits so vielseitig dass, es einfach ist eine hohe Test-Abdeckung eines Programms oder eine Bibliothek zu erreichen.
\newline
\\
Beide Tools zählen zu den "`Unit Testing Tools"'(\cite{wiki.python:PythonTestingToolsTaxonomy}) -
auf Deutsch Modul Test-Tools - mit deren Hilfe die einzelnen Module eines Programms
getestet werden können. In einem Programm oder einer Bibliothek wären dies die einzelnen
Funktionen und Methoden.

% !TeX root = ../bachelor.tex
\subsubsection{unittest}\label{python-tools:unittest}

Unittest

% !TeX root = ../bachelor.tex
\subsubsection{doctest}\label{python-tools:doctest}

"`Das doctest Modul sucht nach Textstücken, die wie interaktive Python-Sitzungen
aussehen, und führt diese Sitzungen dann aus, um sicherzustellen, dass sie genau
wie gezeigt funktionieren."'\footnote{Übersetzt aus dem Englischen}
(\cite{docs.python:doctest}). Diese Textstücke müssen sich in Kommentaren
befinden, da sie sonst von Python als Code interpretiert werden.

\lstinline{doctest} bietet dem Nutzer eine Möglichkeit mithilfe von einem Test
den Code zu dokumentieren. Da \lstinline{doctest} lediglich Code ausführt wie in
einer interaktiven Python-Sitzung, ist auch nur das möglich aus zu führen was im
Code geschrieben ist. Test Fälle oder gar \Glspl{mock} sind hierbei nur bedingt
realisierbar, \Glspl{fixture} hingegen sind teilweise realisierbar in Form von
Code der vor dem Test ausgeführt wird.

\lstinline{doctest} selbst nutzt keine Assert Funktionen oder Methoden,
stattdessen wird der Output eines Befehls überprüft. So ergibt eine Funktion
\lstinline{return_3(s=None, i=None)} bei \lstinline{s=True} eine \lstinline{'3'}
und bei \lstinline{i=True} eine \lstinline{3}. Dieses Beispiel ist in Listing
\ref{listing:doctest:example} zu sehen, um einen Output zu erzeugen wurde noch
ein fehlerhafter Test hinzu gefügt. Dieser Output ist in Listing
\ref{listing:doctest:example_output} zu sehen.

Möchte der Entwickler einen etwas größeren Test schreiben, so kann er sich einer
Funktionalität von \lstinline{doctest} bedienen, die es Ihm ermöglicht Tests in
eine externe Datei zu verlagern. Dabei wird die Datei als ein \Gls{docstring}
behandelt, wodurch sie keine \lstinline{"""} benötigt. Um jedoch Code in diese
Datei ausführen zu können muss das zu testende Modul importiert werden.

Durch das schreiben der Tests in den \Glspl{docstring} werden die Module in
denen Tests geschrieben wurden schnell sehr unübersichtlich und lang, wenn der
Entwickler große Tests schreibt. Zwar wird durch externe Textdateien Abhilfe
geschaffen, dennoch sind zu lange und komplexe doctests schwer zu lesen und
nach zu vollziehen.
\newline

Das Listing \ref{listing:doctest:advanced} zeigt den Code aus
\ref{listing:base:my_module} mit \Glspl{docstring} versehen. Dieser Test wurde
mithilfe der \lstinline{main} Funktion von Python ausgeführt. Der in Listing
\ref{listing:doctest:advanced_text} ausgelagerte Test, wurde mithilfe des CLIs
von \lstinline{doctest} ausgeführt. Da beide Test keine Fehler werfen existiert
auch kein Output für diese Tests.
Die einzige Möglichkeit hier einen Output zu bekommen wäre mit \lstinline{-v} im
CLI oder \lstinline{verbose=True} im Funktionsaufruf. Die dort dargestellten
Informationen sind lediglich welche Kommandos ausgeführt wurden, was erwartet
wurde und das, dass erwartete eingetroffen ist. Ein kleines Beispiel ist in
Listing \ref{listing:doctest:output} zu sehen.

Da \lstinline{doctest} selbst keine Test Fälle unterstützt besitzt
\nameref{python-tools:unittest} eine Integration für \lstinline{doctest},
diese ermöglicht es Tests aus Kommentaren sowie Textdateien in unittest zu
integrieren und zu gliedern. Wie \nameref{python-tools:unittest}, ist auch
doctest in der STDLIB, wodurch keine externen Abhängigkeiten geladen werden
müssen.
\newline

Im Bezug auf die Anwendbarkeit von \lstinline{unittest} lässt sich sagen, dass
nicht alles zur Verfügung steht um TDD zu betreiben. Tests können nicht vor der
eigentlichen Funktionalität geschrieben werden und  \Glspl{fixture} und
\Glspl{mock} werden nicht unterstützt.

Die Anforderungen für \lstinline{doctest} sind sehr gering. Das testen geschieht
mithilfe der interaktiven Python-Sitzung, welche jedem Python Entwickler bekannt
ist. Die Tests selbst sind aufrufe der geschriebenen Funktionen und Methoden und
ein Abgleich des Outputs mit dem Erwarteten Wert. Demnach kann ein Entwickler
sehr Effizient mit \lstinline{doctest} umgehen ohne viel wissen zu benötigen.

\lstinline{doctest} selbst bietet keinerlei Komplexität von sich aus,
stattdessen ist die Komplexität die möglich ist vom Entwickler abhängig, da
dieser die Tests komplett selber schreiben muss. Je nach zu testender
Funktion/Methode kann dies abhängig des Test Aufwands schnell unübersichtlich
werden, wenn viel Code abseits des eigentlichen Tests benötigt wird.

Als Erweiterung bestehen lediglich die Integration in
\lstinline{unittest}\footcite{docs.python:doctest} sowie
\lstinline{pytest}\footcite{docs.pytest.org:4.4}.
\newline
\newline
\\
\textbf{Muss nach Zusammenfassung}
Da sich mit Doctest leider Funktionen nicht präparieren lassen, ist dieses
Test Modul für die alleinige Anwendung in TDD nicht nutzbar. Das Tool bietet
keine Möglichkeiten Objekte zu \gls{mock}en wodurch Tests an nicht
implementieren Methoden und Funktionen scheitern. Auch \Glspl{fixture} sind nur
bedingt realisierbar und benötigen viel Code der dupliziert werden muss, da
dieser vor und nach jedem Test geschrieben werden muss.

% !TeX root = ../bachelor.tex
\subsection{Tools abseits der Standard Bibliothek}\label{python-tools:extlib}
Abseits der Standard Bibliothek gibt es einige Tools deren Nutzung von
Vorteil gegenüber der STDLIB ist. Diese werden unter diesem Punkt aufgeführt.
\newline
\\
Auf \url{https://wiki.python.org/moin/PythonTestingToolsTaxonomy} werden viele
externe Tools gelistet, jedoch scheinen viele inaktiv zu sein da ihre \glspl{commit}
teilweise mehr als ein Jahr zurück liegen. Dies ist weder für eine Bibliothek noch
für ein Tool ein gutes Zeichen, da sich die Anforderungen stetig ändern und niemals
alle Bugs gefixt sind.
\newline
\\
Manche der dort aufgelisteten Tools sind bereits oder werden gerade in andere Tools
integriert. Dies kann zum einem sein, da ein Tool eine Erweiterung für ein anderes
war und die Entwickler die Änderungen angenommen haben und zum anderen um die Tools
zu verbessern und mehr Entwickler zur Verfügung zu haben. Eventuell sind auch
andere Gründe dafür verantwortlich, jedoch war dies der Grund bei zum Beispiel
\href{https://github.com/Yelp/Testify/}{Testify}
von Yelp \footnote{https://github.com/Yelp/Testify/}.
\newline
\\
Da sowohl Software als auch Programmiersprachen sich stetig weiter entwickeln werden in
dieser Arbeit nur jene Tools behandelt die sich diesen Entwicklungen Anpassen, sei dies
durch das unterstützen der aktuellsten Version von Python als auch durch neue Innovationen,
sowie \Gls{bug}-fixes. Aus diesem Grund werden Tools deren letzter \Gls{commit} weiter
als ein Jahr zurück liegt hier nicht behandelt.
\newline
\\
Lässt man zusätzlich die Erweiterungen von Tool zunächst außen vor, so bleiben nach
\cite{wiki.python:PythonTestingToolsTaxonomy} folgende Modul Test-Tools zur Verfügung:
\begin{itemize}
    \item \href{https://github.com/pytest-dev/pytest/}{pytest}\footnote{https://github.com/pytest-dev/pytest/}
\end{itemize}
Tools wie \href{https://www.reahl.org/docs/4.0/devtools/tofu.d.html}{reahl.tofu}\footlabel{reahl.tofu}{https://www.reahl.org/docs/4.0/devtools/tofu.d.html} oder
\href{https://pypi.org/project/zope.testing/}{zope.testing}\footlabel{zope.testing}{https://pypi.org/project/zope.testing/} sind zwar mehr oder weniger aktiv, da sie
beide auf die neusten Python Versionen patchen, jedoch bieten sie sonst keinen Mehrwert in ihren Updates.
\href{https://www.reahl.org/docs/4.0/devtools/tofu.d.html}{reahl.tofu}\footref{reahl.tofu} selbst ist
auch nur eine Erweiterung für bestehende Test Tools, wie zum Beispiel pytest, weshalb es hier nicht behandelt wird. So wird
\href{https://pypi.org/project/zope.testing/}{zope.testing}\footref{zope.testing} auch nicht behandelt, da dieses Tool
wie bereits beschrieben nicht aktiv weiter entwickelt wird und es eher dafür gemacht wurde \href{http://www.zope.org/en/latest/}{zope}\footnote{http://www.zope.org/en/latest/}
Applikationen zu testen und nicht Python Applikationen.

Ein sehr bekanntest Projekt ist \href{https://pypi.org/project/nose/1.3.7/}{nose}\footnote{https://pypi.org/project/nose/1.3.7/}, welches aber
nicht mehr in Entwicklung ist. Der Nachfolger \href{https://pypi.org/project/nose2/}{nose2}\footnote{https://pypi.org/project/nose2/} allerdings
ist unter sehr aktiver Entwicklung und auch wenn das Tool nicht von \cite{wiki.python:PythonTestingToolsTaxonomy} erwähnt wird, so wird es doch
hier mit verglichen werden.

Auch sehr bekannt ist \href{https://pypi.org/project/testtools/}{testtools}\footnote{https://pypi.org/project/testtools/}, jedoch ist auch
hier keine wirkliche weitere Entwicklung zu verzeichnen.
\newline
\\
Demnach wäre die liste an Modul Test-Tools folgende:
\begin{itemize}
    \item \href{https://github.com/pytest-dev/pytest/}{pytest}\footnote{https://github.com/pytest-dev/pytest/}
    \item \href{https://pypi.org/project/nose2/}{nose2}\footnote{https://pypi.org/project/nose2/}
\end{itemize}

Die Quelle aus dem Offiziellen Python Wiki beschreibt sonst keine weiteren Tools.
Sucht auf Google nach weiteren Tools, so findet man die oben gelisteten wenn man
nach den gleichen Kriterien Tools heraus filtert.
\newline
\\
Als weitere Kategorie werden \Gls{mock}-Tools geführt. Auch wenn fast alle
unittest-Tools integriertes \gls{mock}ing haben, so lässt sich mit diesen
Tools meist mehr erreichen. Es wurden die gleichen Filter-Kriterien verwendet wie bei den
Test-Tools oben.
\begin{itemize}
    \item \href{https://www.reahl.org/docs/4.0/devtools/stubble.d.html}{stubble}\footnote{https://www.reahl.org/docs/4.0/devtools/stubble.d.html}
    \item \href{https://github.com/timbertson/mocktest/tree/master}{mocktest}\footnote{https://github.com/timbertson/mocktest/tree/master}
    \item \href{https://github.com/bkabrda/flexmock}{flexmock}\footnote{https://github.com/bkabrda/flexmock}
    \item \href{https://bitbucket.org/DavidVilla/python-doublex}{python-doublex}\footnote{https://bitbucket.org/DavidVilla/python-doublex}
    \item \href{https://github.com/ionelmc/python-aspectlib}{python-aspectlib}\footnote{https://github.com/ionelmc/python-aspectlib}
\end{itemize}
Auch hier finden sich sonst keine weiteren Tools die Open-source sind.
\newline
\\
Als letzte Kategorie werden hier die \Gls{fuzz} Tools behandelt, da diese eine
gute Möglichkeit bieten Code ausgiebig zu testen. Das wohl umfangreichste und nach den obigen Kriterien
einzige Tool ist \href{https://github.com/HypothesisWorks/hypothesis}{hypothesis}\footnote{https://github.com/HypothesisWorks/hypothesis}.

% !TeX root = ../bachelor.tex
\paragraph{pytest}\label{python-tools:pytest}\mbox{}
\newline
Das am 04. August 2009 in Version 1.0.0 veröffentlichte Tool pytest (auch py.test genannt)
ist ein sehr umfangreiches und weit entwickeltes Tool. Seit 2009 wird das Tool stets
weiter entwickelt und vorangetrieben, wodurch es eine Menge an Features gewonnen hat.
\noindent
Die Basis Features von pytest sind folgende:
\begin{itemize}
    \item Simple \lstinline{assert} statements.
    \subitem Kein \lstinline{self.assert}
    \subitem Error Überprüfung mit \Gls{context}
    \item Informativer Output in Farbe
    \subitem Der gesamte Output kann angepasst werden
    \item Feature reiche \Glspl{fixture}
    \subitem Vordefinierte \Glspl{fixture} von pytest
    \subitem Geteilte \Glspl{fixture} unter Tests
    \subitem Globale \Glspl{fixture} zwischen Modulen
    \subitem Parametrisierung von Test als \Glspl{fixture}
    \item Überprüfung von \lstinline{stdout} und \lstinline{stderr}
    \item Gliedern von Test in Fällen
    \subitem ... durch Markierung
    \subitem ... durch Nodes (Auswahl der Modul Abhängigkeit)
    \subitem ... durch String Abgleich der Funktions-Namen
\end{itemize}
\noindent
Da pytest nicht in der STDLIB ist, muss es mit einem \Gls{paket} installiert werden.
Dabei werden für \lstinline{pytest 4.4.0} die in Listing \ref{listing:pytest:requirements}
gezeigten Abhängigkeiten installiert. Demnach benötigt pytest sechs externe Abhängigkeiten
zusätzlich zu sich selbst.
\newline
\\
Wie anhand der Basis Features erkennbar ist, bietet pytest einiges um Tests zu schreiben und
aus zu führen. So ist mit den gebotenen \Glspl{fixture} bereits eine Voraussetzung mehr als
erfüllt, da pytest viele verschiedene Arten bietet \Glspl{fixture} zu benutzen.
\Glspl{fixture} werden in pytest allerdings nicht mit \lstinline{setUp} und \lstinline{tearDown}
geschrieben, sondern werden mithilfe eines \glspl{decorator} markiert und den Test Funktionen
als Parameter übergeben. Um die \lstinline{setUp} und \lstinline{tearDown} funktionalität zu
bekommen muss lediglich das keyword \lstinline{yield} verwendet werden, die \Gls{fixture} wird
dann den Code bis zum \lstinline{yield} ausführen und nach der Funktion den Rest nach
\lstinline{yield} ausführen.
\newline
\\
Diese lassen sich durch zusätzliche Parameter weiter anpassen. Diese
Funktionen sind in Kapitel fünf nach \cite{docs.pytest.org:4.4} beschrieben.

Des weiteren bietet pytest auch \gls{mock}ing, so lässt sich beispielsweise mit
\lstinline{monkeypatch.setattr()} der return Wert einer Funktion ersetzen. Dies wird
in pytest automatisch am ende der Funktion rückgängig gemacht, wodurch der Entwickler
sich mehr auf das eigentliche Testen konzentrieren kann.
\newline
\\
Auch hier bietet pytest weitere Möglichkeiten \Glspl{mock} zu verwenden, dazu hat
\cite{docs.pytest.org:4.4} in Kapitel sieben einige Worte geschrieben.

Ein weiteres sehr praktisches Feature von pytest ist die Gliederung in Fälle. Dies ist bei pytest
sehr fein einstellbar, so lässt sich wie in den Basis Features bereits beschrieben ein Test mit
einer oder mehreren Markierungen versehen wodurch eine Gliederung nach Markierung entsteht. Auch
durch das selektieren bestimmten Wörtern lassen sich Tests nach ihrem Namen gliedern, so entsteht
einerseits eine Gliederung zur Ausführung der Tests und andererseits eine Gliederung für die
Entwickler die sie selbst im Code sehen können. Als letzte alternative lassen sich Tests anhand
ihrer Module ausführen, dies geschieht durch die Angabe der Module. So würde
\lstinline{test_datei.py::TestKlasse::test_methode} den Test \lstinline{test_methode} der Klasse
\lstinline{TestKlasse} in der Datei \lstinline{test_datei.py} ausführen.
\newline
\\
Diese Features werden in Kapitel sechs von \cite{docs.pytest.org:4.4} beschrieben.

Selbstverständlich bietet pytest noch weitere Features jedoch sind diese nicht zwangsläufig notwendig
um TDD zu betreiben und sind mehr ein nice to have Feature als wirklich benötigt. Für eine
Vollständige Auflistung und Erklärung aller Features kann jederzeit unter
\url{https://docs.pytest.org/en/latest/} die aktuellste Version der Dokumentation abgefragt werden.

Demnach bietet pytest alles um TDD anwenden zu können und die sieben zusätzlichen Abhängigkeiten
die ein Entwickler bei der Nutzung von pytest eingeht sollten keinen Entwickler davon abhalten
pytest und seine Features zu genießen.
\newline
\\
Da pytest, wie bereits erwähnt keine neuen assert Methoden hinzufügt lässt sich sehr schnell und
einfach ein Test schreiben. Selbst die Nutzung von \Glspl{fixture} ist in pytest sehr einfach, da
lediglich eine Funktion geschrieben werden muss die mit \lstinline{@pytest.fixture} markiert wurde
und der Test Funktion oder Methode als Parameter übergeben wird. Den Rest erledigt pytest selbst
im Hintergrund. Genauso einfach gestaltet sich die Nutzung von \Glspl{mock}.

Die Effizienz, die bei der Nutzung von pytest entsteht ist demnach sehr hoch. Denn der Entwickler
muss nicht wirklich etwas neues dazu lernen um Tests zu schreiben oder zu präparieren. Genauso
leicht kann ein Entwickler den Output von pytest auswerten, da dieser erstens, in Farbe ist, was
die Lesbarkeit deutlich erhöht, zweitens sehr gut gegliedert ist und Wichtige Strukturen klar
darstellt und drittens nach den Wünschen der Entwickler sich gestalten und verbessern lässt.
\newline
\\
Durch die eben genannten Features im Bezug auf das schreiben von Tests mit assert, sowie
\Glspl{fixture} und \Glspl{mock} lässt sich sagen das pytest sehr viel Funktionalität bietet
wobei es trotzdem Struktur im Code bietet und komplexe Features einfach ermöglicht. Demnach
kann ein Entwickler mit wenig Code viel Testing Funktionalität erstellen.
\newline
\\
Sollten dem Entwickler die Features von pytest nicht ausreichen, so findet man unter
\url{http://plugincompat.herokuapp.com/} eine Liste von 618 (Stand: 2. April 2019) Erweiterungen
für \lstinline{pytest 4.3.0}. pytest selbst kann allerdings auch als Erweiterung zu unittest
genutzt werden, indem ein Entwickler pytest zum ausführen der unittests verwendet. Dadurch bietet
sich dem Entwickler ein verbesserter Output des Test Ergebnisses.
\newline
\\
Im Listing \ref{listing:pytest:advanced} und \ref{listing:pytest:advanced_output} befindet sich der
Code und der Output zum Test von dem in Listing \ref{listing:base:my_module} definierten Modul.
% !TeX root = ../bachelor.tex
\subsubsection{\Gls{mock}ing Tools}\label{python-tools:extlib:mock}

Um eine bessere Gliederung zur Verfügung zu stellen, werden die \Gls{mock}ing
Tools in diesem Kapitel gegliedert.
\\
Wie in \nameref{python-tools} bereits beschrieben, werden hier Tools analysiert,
die eine Erweiterung für Test Tools im Bezug auf verbessertes \gls{mock}ing zur 
Verfügung stellen.

% !TeX root = ../bachelor.tex
\paragraph{stubble}\label{python-tools:stubble}\mbox{}
\newline
Blubb
% !TeX root = ../bachelor.tex
\paragraph{mocktest}\label{python-tools:mocktest}\mbox{}
\newline
\lstinline{mocktest} ist nach \cite{mocktest:doc} ein \gls{mock}ing Tool, das von
\href{http://rspec.info/}{\lstinline{rspec}}\footlabel{rspec}{http://rspec.info/} inspiriert wurde, dabei soll
\lstinline{mocktest} kein Port von \lstinline{rspec}\footref{rspec} sein, sondern eine kleine leichtere Version in Python.
Da sich zum \lstinline{mocktest} derzeit in der Version \lstinline{0.7.3} (Stand 16. April 2019) befindet sind
noch nicht alle Features vollkommen entwickelt und ausgereift, dennoch lässt sich das Tool bereits Produktiv einsetzen,
da das sehr Nützliche Feature \lstinline{sohould_receive()} von \lstinline{rspec}\footref{rspec} für \lstinline{mocktest} implementiert wurde.

Das Hauptfeature von \lstinline{mocktest} ist die Test Isolation, die verhindert das \glspl{mock} Objekte außerhalb eines
Test Falles verändern und so andere Tests beeinflussen. Demnach werden Tests immer innerhalb einer \lstinline{mocktest.MockTransaction} ausgeführt, sofern diese ein oder mehrere Objekte \gls{mock}en sollen.

Innerhalb dieses Kontextes stehen dem Entwickler verschiedene Möglichkeiten zur Verfügung Tests aus zu führen. So lässt
sich Beispielsweise überprüfen ob eine Funkion oder Methode aufgerufen wurde, dabei spiel es keine Rolle ob es sich hierbei
um ein Globales Objekt handelt oder ein Lokales. Des weiteren lassen sich auch \Glspl{stub} nutzen, mit deren Hilfe
Funktionen und Methoden präpariert werden können, wie auch beim überprüfen des Aufrufs spielt es hier keine Rolle ob es
sich um ein Lokales oder Globales Objekt handelt. Die Präparation lässt sich nach bedarf auch anpassen, so kann der
Entwickler beispielsweise festlegen, dass nur bei einem bestimmten Aufruf mit bestimmten Parametern das \Gls{mock} Objekt
aufgerufen wird. Andernfalls wird das Originale Objekt genutzt.

Mocktest nutzt \lstinline{unittest} als Basis für seine Testumgebung, \lstinline{mocktest.TestCase} erbt von
\lstinline{unittest.TestCase}, wodurch die gesamt Funktionalität von \lstinline{unittest} gegeben wird. \lstinline{mocktest.TestCase} führt dabei in seiner \lstinline{setUp()} und \lstinline{tearDown()} methode Code aus, der
benötigt wird um \lstinline{mocktest} Nutzen zu können, möchte man jedoch nicht unittest verwenden, so ist es möglich
\lstinline{mocktest.MockTransaction} als Kontextmanager mit \lstinline{with MockTransaction:} zu nutzen. Zusätzlich wird
mit mocktest unittest um eine \lstinline{assert} Methode erweitert, \lstinline{assertRaises()} verfügt über verbesserte Funktionalität zum überprüfen auf Exceptions.

Zum überprüfen von \Glspl{mock} kann entweder \lstinline{when(obj)} oder \lstinline{expect(obj)} verwendet werden. Der
unterschied hier ist lediglich, dass \lstinline{when(obj)} nicht überprüft ob eine Methode aufgerufen wurde oder nicht.
Legt man hingegen Beispielsweise fest dass, \lstinline{expect(os).system} dann wird ein Fehler geworfen, wenn
\lstinline{os.system} nicht aufgerufen wurde, was bei \lstinline{when(os).system} nicht der Fall ist. Möchte man später
überprüfen wie oft und mit was ein Objekt aufgerufen wurde, ist es möglich \lstinline{received_calls} ab zu prüfen, dies
ist eine Liste von allen getätigten aufrufen und Ihren Parametern.
\noindent
Hier ist eine Liste von allen weiteren Features von \lstinline{mocktest}:
\begin{itemize}
    \item Ergebnisvariation mit \lstinline{.and_return(erg1, erg2, ergX)}, alternativ ist \lstinline{.then_return()} die gleiche Methode.
    \item Mit \lstinline{.at_least(n)}, \lstinline{.at_most(n))}, \lstinline{.between(a, b)} und \lstinline{.exact(n)} lässt sich festlegen
        wie viele aufrufe erfolgt sein müssen.
    \begin{itemize}
        \item Als alias wird \lstinline{.never()} und \lstinline{.once()} geboten für nie und einmal aufgerufen.
    \end{itemize}
    \item Durch \lstinline{.then_call(func)} wird die Funktion oder Methode \lstinline{func} ausgeführt anstatt der eigentlichen.
    \item Properties lassen sich mit \lstinline{.with_children(**kwargs)} setzen, wobei \lstinline{mock('mein_mock').with_children(x=1)}
        den Wert \lstinline{x}  für \lstinline{mein_mock} gleich \lstinline{1} setzt.
    \item Das gleiche ist mit \lstinline{.with_method(**kwargs)} möglich, nur für Methoden anstatt Properties.
\end{itemize}

Um TDD zu betreiben bietet \lstinline{mocktest} alles was ein Entwickler zum \gls{mock}en benötigt. So ist es möglich
bestehende Objekte gänzlich oder Teilweise zu ersetzen, oder neue Objekte zu erstellen die, die Signatur eines anderen
Objektes haben. Dabei kann selbstverständlich geprüft werden wie und wie oft etwas aufgerufen wurde. Auch Globale Objekte
lassen sich für den Zeitraum eines Tests ersetzen, dadurch ist alles an Funktionalität geboten, was ein \gls{mock}-testing
Tool bieten muss.

Bezüglich der Effizienz ist mocktest äußerst gut, da quasi kein Vorwissen von Nöten ist um Objekte zu \gls{mock}en. Die
Vorarbeit die geleistet werden muss hält sich je nach Anwendungsfall sehr in grenzen oder ist fast nicht existent. Nutzt
man die von \lstinline{mocktest} zur Verfügung gestellte Klasse \lstinline{mocktest.TestCase} für einen Testfall, so ist
bereits alles fertig und direkt einsetzbar. Selbstverständlich kann \lstinline{mocktest.TestCase} nach den Bedürfnissen
um Funktionalität erweitert werden, jedoch sollte sich dies als sehr leicht gestalten. Entscheidet man sich gegen
\lstinline{unittest}, so ist mit dem Kontextmanager \lstinline{mocktest.MockTransaction} schnell eine \Gls{mock}
Umgebung aufgesetzt und Nutzbar.

\lstinline{mocktest} lässt den Entwickler Code schreiben der sich wie gesprochen liest. Durch die Methoden
\lstinline{when(obj)} und \lstinline{expect(obj)} lässt sich ein \Gls{mock} erstellen der mit Methoden angepasst werden
kann, die wenn man sie aneinander reiht wie beschrieben lesen. Dies ist zum einen durch die verschiedenen Aliase möglich,
wie zum Beispiel \lstinline{.once()} welches \lstinline{.exact(1)} ausführt. auch \lstinline{.then_return()} ist sehr
leicht zu lesen und vereinfacht das verstehen des Codes enorm. Abseits der eigentlichen Funktionalität die mit
\lstinline{mocktest} gesetzt wird fällt quasi kein Code an der geschrieben werden muss.

Um auch hier eine Analyse erstellen zu können, wurde der Code aus Listing \ref{listing:base:my_mock_module} mit
\lstinline{mocktest} getestet. Dieser Code ist in Listing \ref{listing:mocktest:example} zu finden.

\textbf{analyse des Beispiels}
% !TeX root = ../bachelor.tex
\paragraph{flexmock}\label{python-tools:flexmock}\mbox{}
\newline
\lstinline{flexmock} ist ein weiteres Tool, dass von der Ruby Community inspiriert wurde.
Dabei diente das gleichnamige Tool
\href{https://github.com/jimweirich/flexmock}{flexmock}\footnote{https://github.com/jimweirich/flexmock}
als Inspiration.
"`[...]. Es ist jedoch nicht das Ziel von Python flexmock, ein Klon der
Ruby-Version zu sein. Stattdessen liegt der Schwerpunkt auf der vollständigen Unterstützung
beim Testen von Python-Programmen und der möglichst unauffälligen Erstellung von gefälschten
Objekten."'\footnote{Übersetzt aus dem Englischen} (\cite{flexmock:docs:0.10.3}).

Die Features von \lstinline{flexmock} sind denen von \nameref{python-tools:mocktest} sehr ähnlich.
So bildet \lstinline{flexmock} seine Tests, \glspl{stub} und \glspl{mock} wie ausgesprochen dar,
ein Feature aus der Ruby Welt, dass anscheinend sehr viel anklang findet bei Python Entwicklern.

\lstinline{flexmock} bietet dem Nutzer einiges an Funktionalität auch wenn es sich erst in der
Version \lstinline{0.10.4} (Stand 22. April 2019) befindet. So lässt sich ein \Gls{stub} für
Klassen, Module und Objekte mithilfe von \lstinline{flexmock()} einrichten. Damit lässt sich
alles was ein Entwickler zum erfolgreichen durchlaufen seiner Tests brauch einstellen. Auch
\gls{mock}ing ist kein Problem mit \lstinline{flexmock}. Mit der gleichen Funktion, mit der
auch \Glspl{stub} erstellt werden lässt sich auch ein \Gls{mock} abbilden. So kann ein
\Gls{stub} auch ein \Gls{mock} sein und umgekehrt.
Dadurch ist es dem Entwickler möglich zu überprüfen wie oft etwas aufgerufen wurde, welche
Parameter verwendet wurden, welcher Rückgabe wert zu erwarten ist und/oder welche Exception
zu erwarten ist.
Zusätzlich ist es möglich originale Funktionen eines Objektes zu \gls{mock}en, dadurch
enthält der Entwickler die Möglichkeit zu überprüfen ob eine unveränderte Funktion oder
Methode bestimmten Anforderungen Entspricht, dabei ist das Interface das gleiche wie bei
einem \Gls{stub}.
\lstinline{flexmock} bietet des weiteren den gefälschten Objekten neue Methoden hinzu zu fügen,
wodurch nicht existierende Funktionen abgebildet werden können oder Objekte in ihrem
Funktionsumfang temporär erweitert werden können.

Ein besonders interessantes Feature ist, dass überprüfen des Ergebnisses nach Typ. Dabei
wird \lstinline{.and_return()} nur der Typ mitgegeben, der erwartet wird. So würde
\lstinline{.and_return((int, str, None)} jedes \lstinline{Tuple} zulassen, dass als ersten Wert einen
\lstinline{int} hat, als zweiten einen String und als drittes None, dabei kann selbstverständlich auf
jede beliebige Instanz überprüft werden, auch auf eigene Klassen. Dies ist allerdings nur
möglich bei \lstinline{.should_call()} nicht bei \lstinline{.should_receive()}.

Das letzte in der von \cite{flexmock:docs:0.10.3} erwähnte Feature ist das ersetzen von
Klassen noch vor ihrer Instanziierung. Dabei gibt es mehrere Ansätze die allerdings auch
verschiedene Ausgänge haben, der erste ist auf Modul Level, dabei wird im Modul die Klasse
mit einem Objekt ersetzt, dass entweder ein \lstinline{flexmock} sein kann oder ein Beliebig
anderes. Der Nachteil dieser Methode ist, dass es eventuell zu Problemen führen kann, dass
die Klasse durch eine Funktion ersetzt wurde. Die alternative ist, \lstinline{.new_instance(obj)}
zu verwenden, welches beim erstellen das Objekt zurück gibt, welches \lstinline{.new_instnce()}
übergeben wurde oder man ersetzt die \lstinline{__new__()} Methode der Klasse mit
\lstinline{.should_receive('__new__').and_return(obj)} was im Endeffekt das gleiche ist, nur
ausgeschrieben.

Die Anwendbarkeit von \lstinline{flexmock} ist sehr gut, da es zunächst keine
weiteren Abhängigkeiten installiert außer sich selbst und mit allen Test-runnern
kompatibel ist. Im Bezug auf TDD lässt \lstinline{flexmock} nichts zu wünschen übrig,
dem Entwickler werden allerhand Funktionalität geboten \Glspl{stub} oder \Glspl{mock} zu
erstellen und zu verwenden. Dabei lässt sich alles überprüfen von der Aufruf Anzahl bis zu
den Parametern die verwendet wurden.

Im Aspekt Effizienz ist \lstinline{flexmock} ein Parade-Beispiel, einmal \lstinline{flexmock()}
aufgerufen kann bereits los gelegt werden, dabei muss der Entwickler selbst wenig von
\gls{mock}ing verstehen um die Funktionalität nutzen zu können, da sich der Code wie gesprochen
ließt (Sofern der Entwickler englisch sprechen kann). Das gleiche gilt für die \Glspl{stub} und
fake Objekte.

Das gleiche gilt für die Komplexität von \lstinline{flexmock}, an Funktionalität bietet
\lstinline{flexmock} alles was man sich als Entwickler wünschen kann und bietet dabei auch
noch ein Interface, dass es ermöglicht übersichtlichen Code zu schreiben. Dies liegt vor allem
an der deskriptiven Programmierung die von \lstinline{flexmock} geboten wird. Unübersichtlich
wird dadurch der Code nicht mehr als er im laufe der Zeit sowieso werden würde, eher noch hilft
\lstinline{flexmock} dabei die Übersicht länger zu wahren.

Der Code zu Listing \ref{listing:base:my_mock_module} wurde mithilfe des Codes aus Listing
\ref{listing:flexmock:example} getestet, dabei ergaben sich die eben aufgelisteten Aspekte von 
\lstinline{flexmock}. \lstinline{flexmock} überzeugt dabei mit der Einfachheit
mit der es aufgesetzt werden muss. In der \lstinline{setUp()} wird global im Modul
\lstinline{my_package.my_mock_module} die Klasse \lstinline{NotMocked} ersetzt. Dadurch ist
ex möglich in jedem Test Fall eine Annahme auf zu stellen, die dann auf alle Objekte
der Klasse \lstinline{NotMocked} zutreffen. Der aufwand diesen Code zu schreiben war demnach
sehr gering. Lediglich das erstellen einer Methode die im Originalen Objekt nicht existiert
war nicht möglich, aus diesem Grund ist der Code in \lstinline{test_call_helper_help()} nicht
korrekt und wirft eine Exception, diese kann in Listing \ref{listing:flexmock:exception}
eingesehen werden.
% !TeX root = ../bachelor.tex
\paragraph{doublex}\label{python-tools:doublex}\mbox{}
\newline
Mit \lstinline{doublex} wird dem Entwickler ein Tool an die Hand gelegt,
dessen Funktionalität sich voll und ganz um doubles dreht. Ein double ist je
nach Anwendung ein \Gls{mock}, ein \Gls{stub} oder ein Spy(Spion), der eine
Klasse oder ein Objekt imitiert (\cite{doublex:docs:1.8.1}).
\newline

\lstinline{doublex} bietet drei verschiedene Interfaces (vier zählt man die
Unterklasse von Spy dazu), \Gls{stub}, Spy (ProxySpy) und Mock. Da die
Dokumentation hier sehr schöne und treffende Beschreibungen nimmt, werden diese 
nun zitiert. "`\Glspl{stub} sagen dir, was du hören 
willst."'\footlabel{doublex:doc:trans}{Übersetzt aus dem Englischen} 
\footlabel{doublex:doc:cite}{\cite{doublex:docs:1.8.1}}. "`Spies erinnern sich 
an alles was Ihnen passiert."'\footref{doublex:doc:trans} 
\footref{doublex:doc:cite}. "`Proxy spies leiten Aufrufe an ihre originale 
Instanz weiter."'\footref{doublex:doc:trans} \footref{doublex:doc:cite}. "`Mock 
erzwingt das vordefinierte Skript."'\footref{doublex:doc:trans} 
\footref{doublex:doc:cite}.

Mit diesen Beschreibungen der einzelnen Interfaces kann ein Entwickler bereits
entscheiden, welches der Objekte relevant für die zu erledigende Arbeit ist, 
ohne
den Code kennen zu müssen. Um dennoch einen erweiterten Einblick zu
schaffen, werden die einzelnen Funktionen nun genauer beschrieben.

Wichtig hierbei ist zu beachten, dass \lstinline{doublex} lediglich
doubles, also Duplikate anbietet, die das gleiche Interface wie eine Klasse
haben, jedoch keine Instanz dieser Klasse sind. Dadurch bleiben manche
Möglichkeiten, wie zum Beispiel das Nutzen einer
implementierten Methode aus der original Klasse, dem Entwickler verwehrt.
Die Duplikate sollen als Ersatz für Objekte gelten, die in einer bestimmten
Umgebung ausgeführt werden müssen oder Änderungen vornehmen, die während
eines Tests unerwünscht sind.
\newline

Das \Gls{stub} Interface ist ein \Gls{stub} mit dem es dem Entwickler möglich 
ist, Rückgabewerte von Funktionen zu setzen oder ähnliche Aktionen zu 
definieren. Dabei ist es dem Entwickler möglich, zwischen verschiedenen 
Parametern zu unterscheiden oder gar auf alle Aufrufe zu prüfen.

Das Interface bietet die Möglichkeit Parameter von einem Objekt zu modifizieren
um den Ausgang von Methoden zu verändern. Hierbei wird zwischen einem \Gls{stub}
und einem "`free"' \Gls{stub} unterschieden. Ersterer nimmt eine Klasse als
Parameter und ersetzt diese, zweiterer nimmt keine Parameter und fungiert als
generelles Objekt. Dabei ist der wichtigste Unterschied, dass der normale
\Gls{stub} nur Methoden abändern kann, welche die originale Klasse auch 
besitzt. Der "`free"' \Gls{stub} hingegen ist in dieser Hinsicht ungebunden und 
kann alles sein, was der Entwickler programmiert.
\newline

Als zweites Interface wird in der Dokumentation das Spy Interface erwähnt
(\cite{doublex:docs:1.8.1}). Dieses bietet dem Nutzer die Möglichkeit Klassen zu
überwachen. Dabei legt der Entwickler fest, welche Methode wie oft und mit
welchen Parametern aufgerufen werden muss. Werden die Erwartungen nicht
getroffen, wird eine Exception geworfen. Mit Hilfe des Überprüfen der Parameter 
und mit Hilfe des exakten Wertes ist es möglich, einer Regex oder mit dem von
\lstinline{doublex} definierten Schlüsselwort \lstinline{ANY_ARG} zu erstellen, 
welches, wie der Name sagt, jedes Argument validiert.

Zusätzlich gibt es einen "`free"' Spy, welcher von keiner
Klasse abhängig ist. Hierdurch ist es möglich, jede beliebige Methode auf
diesem aus zu führen, ohne dass dies zu einer Exception führen würde.
\newline

Zusätzlich zum Spy und "`free"' Spy, gibt es noch einen \lstinline{ProxySpy},
welcher das zu überwachende Objekt aufruft und nicht ersetzt. Aus diesem Grund
erhält der \lstinline{ProxySpy} ein Objekt als Parameter und kein
Klasseninterface. Sämtliche Methoden werden auf das originale Objekt ausgeführt,
wodurch keine Veränderung an diesem vorgenommen werden kann.
\newline

Das letzte Interface, ist das Mock Interface. Mit diesem lässt sich vor dem Test
festlegen, welche Methode wann und wie aufgerufen werden muss, damit der Test
erfolgreich ist. Die Reihenfolge spielt dabei eine wichtige Rolle, außer es wird
\lstinline{any_order_verify()} genutzt. Das Mockobjekt lässt sich, wie
jedes andere Objekt, verändern und wie ein \Gls{stub} präparieren.

Auch der "`free"' Mock ist, wie bei den anderen Interfaces, verfügbar und bietet
die gleichen Möglichkeiten der freien Gestaltung des Objekts.
\newline

Zusammenfassend lässt sich sagen, dass jedes Interface die Möglichkeit bietet, 
eine
Klasse zu verändern, ausgenommen der \lstinline{ProxySpy}, der nur auf einer
Instanz arbeiten kann.

Zu jedem Interface ist ein freies verfügbar, welches mit beliebigen
Methoden aufgerufen werden kann ohne Fehler zu werfen. Abschließend lässt sich
sagen, dass jedes Interface, die Möglichkeit bietet,
\gls{stub}ing zu betreiben, wobei nur das \Gls{stub}interface sonst keine
weitere Funktionalität bietet.
\newline

Die Features von \lstinline{doublex} wurden auf den in Listing
\ref{listing:base:my_mock_module} definierten Code angewandt (vgl. Listing
\ref{listing:doublex:example}). Wie dort zu erkennen ist, konnte nicht jeder
Test ohne Probleme validiert werden. So war es zwar möglich die ersten drei
Tests erfolgreich zu validieren, jedoch ist es fraglich, ob dieser Code
wirklich etwas bringt, da die getesteten Objekte einfach nur zurück geben, was
erwartet wird und dabei nicht einmal das originale Objekt sind.

Bei dem Test, der eine interne Methode aufrufen soll, kommt das Spy Interface an
seine Grenzen, da es nicht überprüfen kann, ob die interne Methode aufgerufen
wurde. Stadtessen wird eine Exception geworfen, welche in einem
Kommentar in der Testmethode festgehalten wurde.

Das Testen, ob \lstinline{Helper.help()} aufgerufen wurde, war teilweise
erfolgreich, da es Dank des Duplikates ein Objekt mit dem richtigen Interface
bekommen hat. Es konnte jedoch nicht vortäuschen, die gewünschte Klasse zu sein.
Auch hier wurde der Fehler in einem Kommentar festgehalten.

Zuletzt sollte getestet werden, ob das \lstinline{os} Modul ausgetauscht werden
kann. Dies ist mit \lstinline{doublex} jedoch nicht möglich. Der Fehler
dazu wurde in einem Kommentar innerhalb der Funktion festgehalten, sowie eine
mögliche, wenn auch umständliche Lösung.
Diese würde das \lstinline{os} Objekt duplizieren und der Methode als
Parameter übergeben, was in der Praxis nicht gemacht werden
sollte, da dazu das Interface der Methode geändert werden müsste.
\newline

\lstinline{doublex} bietet dem Entwickler viel, aber nicht alles Notwendige für 
effizientes
TDD. Wie der Name des Tools verrät, handelt es sich um Duplikate von Objekten,
deren Funktionalität aber vom Entwickler festgelegt werden muss. So würde eine
Klasse mit einer Methode \lstinline{return_input(input)} die den übergebenen
Parameter zurück gibt, standardmäßig \lstinline{None} zurück geben, solange
nichts anderes im Duplikat festgelegt wurde. Lediglich die Signatur des Aufrufs
wird überprüft. Deswegen würde \lstinline{stub.reuturn_input()} eine Exception
werfen.

Zwar lässt sich das Standardverhalten von \lstinline{doublex} verändern, jedoch
kann nur ein fester Wert gesetzt werden, der für alle nicht ersetzten Methoden
gilt. Globale Objekte oder Module lassen sich ebenso nicht ersetzen. Hinzu 
kommt, dass das Tool drei Abhängigkeiten benötigt und selbst noch nicht in 
stabilem Zustand ist. Während der Analyse sind zwei 
\lstinline{DeprecationWarning}s aufgetaucht, wovon eine bereits mit Python 3.0 
ausgelaufen ist. Diese weisen auf eine nicht sehr aktive Entwicklung des Tools 
hin.

Selbstverständlich wurden die Fehler auf GitHub gemeldet, sodass der Entwickler
sich in Zukunft darum kümmern kann (Stand 23. April 2019). Jedoch wurden diese 
bis zum Abschluss dieser Arbeit nicht gefixed (Stand 19. Juli 2019).

Auch die Effizienz ist nicht optimal. Dadurch, dass das Duplikat nicht die
originalen Methoden aufruft, muss jede Methode mit einem Returnwert
überschrieben werden. Dies mag für manche Tests vollkommen ausreichen, macht
aber die Entwicklung mit TDD sehr schwer, da Tests von Zeit zu Zeit ausgeführt
werden müssen und dort die original Implementierung ab und zu genutzt
werden soll. Hat man allerdings eine externe unbeeinflussbare Klasse
kann dieses Tool durchaus nützlich werden.

An Komplexität fehlt es \lstinline{doublex} ein wenig. So kann der \Gls{stub}
lediglich festlegen, welche Funktion ausgeführt, welcher Rückgabewert zurück
gegeben oder welche Exception geworfen wird.

Das Spy Interface hingegen ist leicht zu nutzen und bietet fast alles, was ein
Entwickler brauchen kann, um zu überprüfen, ob und wie etwas aufgerufen wurde.
Jedoch scheitert es hier auch an der Implementierung von echten Objekten. So
fungiert der ProxySpy zwar als Spy auf einem Objekt, jedoch kann er nur
verzeichnen, welche Methode auf ihm aufgerufen wurde.

\lstinline{spy.call_other()} würde nicht \lstinline{obj.other()} registrieren,
wodurch es unmöglich ist zu überprüfen, ob \lstinline{other()} nun aufgerufen
wurde oder nicht. Das gleiche gilt für \Gls{mock}. Lediglich Methoden, die auf
dem Mock-Objekt aufgerufen werden, werden registriert. Aus diesem Grund ist es
auch hier nicht optimal nutzbar für TDD.
% !TeX root = ../bachelor.tex
\subsubsection{Fuzz-testing Tools}\label{python-tools:extlib:fuzz}

Zusätzlich zu den hier behandelten Test runnern und den \gls{mock}ing Tools werden
zusätzlich noch Fuzz-testing Tools behandelt. Unter Fuzz-testing versteht man das
Testen eines Programms oder einem Modul mit zufälligen Werten. Diese Werte können
komplett zufällig sein, aber auch einem gewissen Format entsprechen oder einem Typ.

Fuzz-testing funktioniert dabei anders als die gewohnten Methoden des Testens. Der
Normale Flow ist, dass Test Daten präpariert werden, dann werden Tests mit diesen
Daten aus geführt und danach validiert, dass das Ergebnis richtig ist. Mit
Fuzz-testing sieht das ganze etwas anders aus, zuerst wird festgelegt, welchem
Schema die Daten entsprechen müssen (Beispiel: Es wird eine IP erwartet,
\lstinline|[1-9]{1,3}.[1-9]{1,3}.[1-9]{1,3}.[1-9]{1,3}|), danach wird mit den Daten
der Test ausgeführt und danach validiert (\cite{hypothesis:doc:4.18.0}).

Der unterschied ist nun, das der Entwickler selbst sich nicht Daten ausdenken muss.
Im genannten Beispiel müsste der Entwickler sich verschiedene IP Adressen ausdenken,
die entweder richtig oder Falsch sind und danach diese überprüfen. Fuzz-testing
übernimmt das ausdenken der Daten und sorgt dafür, dass diese der Spezifikation
entsprechen. So kann eine viel größere Anzahl an Tests mit erheblich weniger aufwand
auf einem Modul oder Programm ausgeführt werden.

Das ist aber noch nicht alles, schlägt ein Test fehl, mit Daten die per Spezifikation
richtig sind, so wird der Wert der Daten für später gespeichert und sofern möglich
wird versucht mit ähnlichen Daten dieser Fehler zu erreichen. Dadurch hat ein
Entwickler am Ende der Tests Daten die fehlgeschlagen sind. Mit diesen kann er dann
die Weiteren Tests befüllen um so zu verhindern, dass diese Daten zu Fehlern führen.
Im Normalfall übernimmt dies allerdings das Tool, wodurch der Entwickler sich nur
darum kümmern muss, dass die Tests erfolgreich verlaufen.

Von Zeit zu Zeit entsteht dadurch ein Katalog an Daten für verschiedene Tests, der
vorgibt bei welchen Daten ein Test fehl geschlagen ist. Dadurch wird verhindert,
dass beim ändern des Codes dieser Fehler wieder auftritt. Führt man dies über den
gesamten Entwicklungsprozess hinweg durch, so erhält man am Ende ein sehr ausgiebig
getestetes Programm.

Diese Methode des Testens ist auch "`property based testing"', also Eigenschafts-
basierte Tests genannt. In dieser Arbeit, wird allerdings weiterhin der Englische
Begriff Fuzz-testing genutzt.

% !TeX root = ../bachelor.tex
\paragraph{hypothesis}\label{python-tools:hypothesis}\mbox{}
\newline
Das größte und am weitesten verbreitete Tool für Fuzz-testing mit Python ist
\lstinline{hypothesis}. Recherchen im Internet weisen auf kein weiteres 
aktuelles Tool hin, welches Open Source ist, demnach ist \lstinline{hypothesis} 
das einzige Tool welches zum Fuzz-testing mit Python derzeit genutzt werden 
kann (Python 3.7).

Bevor man anfangen kann mit \lstinline{hypothesis} zu testen muss man verstehen,
wie \lstinline{hypothesis} funktioniert. Möchte man Daten haben, die einer
Spezifikation entsprechen, benötigt man \lstinline{hypothesis.strategies}. In
\lstinline{hypothesis} sind Strategien Spezifikationen für die Generierung von
Daten. Wäre eine Spezifikation also, dass die Daten \lstinline{Integer} sind, 
dann würde man \lstinline{hypothesis.strategies.integers()} verwenden. Diese 
Strategie würde verschiedene Daten die \lstinline{Integer} sind ausgeben. 
Möchte man nur Werte zwischen A und B, so wäre die Strategie 
\lstinline{.integers(min_value=A, max_value=B)}. Dies kann man mit beliebig
vielen Daten Typen und Spezifikationen durchführen. Dabei verfügt 
\lstinline{hypothesis} über so viele verschiedene Spezifikationen, dass es 
schwer sein wird, alle zu kennen. Die Basics lassen sich allerdings in der
\href{https://hypothesis.readthedocs.io/en/latest/data.html}{Dokumentation}
\footnote{https://hypothesis.readthedocs.io/en/latest/data.html} finden.

Ist die Spezifikation bekannt und die Strategie gefunden, werden Tests mithilfe 
einer \Gls{annotation} der Test Funktion übergeben. Diese \Gls{annotation} 
nennt sich \lstinline{hypothesis.given()}. Ein kleines Beispiel dazu kann in 
Listing \ref{listing:hypothesis:given} gefunden werden. \lstinline{.given()} 
unterstützt dabei die \lstinline{*args} als auch die \lstinline{**kwargs} 
Übergabe der Strategien.

Hat man nun einen Fehler gefunden und möchte diesen für alle weiteren Tests 
testen, so kann man mit \lstinline{hypothesis.example()} den Test annotieren 
und die Werte, bei denen der Test fehl geschlagen ist, übergeben. Alternativ 
übernimmt dies \lstinline{hypothesis} durch das Speichern des Wertes im Cache. 
Allerdings kann dieser verloren gehen oder durch Umbenennen einer Funktion 
invalide werden. Um zu verhindern, dass der Cache von Entwickler zu Entwickler 
unterschiedlich ist, lässt sich der Cache in der VCS einbinden. Dazu sollte 
alles in \lstinline{.hypothesis/examples/} hinzu gefügt werden. Alle anderen 
Ordner in \lstinline{.hypothesis} sind nicht dafür ausgelegt in das VCS 
integriert zu werden, da sie zu Mergekonflikten führen können.

Möchte der Entwickler etwas mehr Information darüber, weshalb ein Test fehl 
geschlagen ist, so ist es ihm möglich mit \lstinline{hypothesis.note()} vor 
einer Assertion einen Text aus zu geben, der ihm diese Informationen gibt. 
\lstinline{note()} wird
allerdings nur aus gegeben, wenn der Test fehl schlägt.

Möchte man ein paar Statistiken darüber was \lstinline{hypothesis} gemacht hat 
und man nutzt \lstinline{pytest} als Test runner, so kann man mit 
\lstinline{--hypothesis-show-statistics} sich die Statistiken zu den Tests 
anschauen. Dabei werden folgende Informationen zu jeder Test Funktion gezeigt:
Die gesamte Anzahl der Durchläufe mit der Anzahl der fehlgeschlagenen und der 
Anzahl der invaliden Tests, die ungefähre Laufzeit des Tests, wie viel Prozent 
davon für die Datengenerierung aufgewendet wurde, weshalb der Test beendet 
wurde sowie alle aufgetretenen Events von \lstinline{hypothesis}.

Ein Event kann allerdings auch vom Entwickler genau so, wie \lstinline{note()} 
genutzt werden, mit dem Unterschied, dass ein Event immer zu einem Output 
führt, wenn die Zeile im Code ausgeführt wird (sofern \lstinline{pytest} und 
\lstinline{--hypothesis-show-statistics} genutzt werden).

Mit diesem Wissen lässt sich \lstinline{hypothesis} bereits für die einfachsten 
Dinge nutzen. Bestehen etwas Spezifischere Anforderungen, so muss etwas tiefer
in die Materie eingestiegen werden. Angenommen, eine Funktion soll mit einem
\lstinline{Integer} getestet werden, der durch sich selbst teilbar sein muss.
Dies könnte mit einem \lstinline{if} Statement geprüft werden, würde aber die 
Anzahl der relevanten Tests erheblich senken. Standessen ist es möglich auf 
Strategien Filter an zu wenden. Filter müssen dabei Funktionen sein, die einen 
Wert filtern und zurück geben. Für das eben genannte Beispiel würde die 
\Gls{annotation} dann so aussehen:
\lstinline{@given(integers().filter(lambda i: i % i == 0))}.

Gibt \lstinline{hypothesis} einem Test Daten, die nicht gefiltert werden können 
oder sollen, so ist es möglich mit \lstinline{hypothesis.assume()} eine Annahme 
auf zu stellen. Ist die Annahme falsch, so wird dieser Test übersprungen. 
Allerdings kann dies zu Fällen führen, bei denen \lstinline{hypothesis} keine 
validen Daten finden kann, was zu einer Exception und zum fehlschlagen des 
Tests führt.

\lstinline{hypothesis} verfügt noch über weit mehr Funktionalität als hier 
genannt werden kann, wie zum Beispiel das Verketten von Strategien. Die 
Dokumentation zu allem, was \lstinline{hypothesis} ohne Erweiterungen 
erschaffen kann, kann 
\href{https://hypothesis.readthedocs.io/en/latest/data.html}{online}\footnote{https://hypothesis.readthedocs.io/en/latest/data.html}
eingesehen werden.

Auch wenn die Standard Bibliothek von \lstinline{hypothesis} bereits viele 
Strategien enthält, so gibt es dennoch einiges, was mithilfe von Erweiterungen 
dazu gewonnen werden kann. So werden von \lstinline{hypothesis} selbst 
First-party Erweiterungen zur Verfügung gestellt. Dies dient vor allem dazu die 
Abhängigkeiten von der Standard Bibliothek so gering wie möglich zu halten 
(eine zusätzliche Abhängigkeit) und zum anderen um Kompatibilitätsprobleme zu 
verhindern. Aus diesem Grund können diese zusätzlichen Erweiterungen extra 
installiert werden, dazu finden sich in der Dokumentation drei Seiten, eine 
Generelle\footnote{https://hypothesis.readthedocs.io/en/latest/extras.html}, 
eine für 
Django\footnote{https://hypothesis.readthedocs.io/en/latest/django.html} und 
eine für Wissenschaftliche 
Module\footnote{https://hypothesis.readthedocs.io/en/latest/numpy.html}.
Zusätzlich zu den First-party Erweiterungen, gibt es auch noch Erweiterungen 
der Community, eine kleine Liste dazu kann in der  
Dokumentation\footnote{https://hypothesis.readthedocs.io/en/latest/strategies.html}
gefunden werden oder durch das Explizite suchen einer Strategie im Internet.

Ein kleines Beispiel einer Anwendung von \lstinline{hypothesis} war bereits in 
Listing \ref{listing:hypothesis:given} zu sehen, für weitere Beispiele kann in 
der Dokumentation unter
"`\href{https://hypothesis.readthedocs.io/en/latest/quickstart.html}{Quick
start
guide}"'\footnote{https://hypothesis.readthedocs.io/en/latest/quickstart.html}
und unter
"`\href{https://hypothesis.readthedocs.io/en/latest/examples.html}{Some more
examples}"'\footnote{https://hypothesis.readthedocs.io/en/latest/examples.html}
nachgesehen werden.

Da \lstinline{hypothesis} das derzeit einzige aktiv entwickelte Open Source 
Tool ist, müssen Entwickler, die \gls{fuzz} betreiben wollen, auf dieses Tool 
zurück greifen. Für das Testen von Funktionen und Methoden reicht dies aus. 
Selbstverständlich kommt es hier stets auf den Anwendungsfall an, denn nicht 
jede Funktion oder Methode lässt sich mit zufälligen Werten wirklich effektiv 
testen. Jedoch bietet \lstinline{hypothesis} dort, wo es sinnvoll ist, alles
um ausgiebig zu testen. Sollten die in der Standard Bibliothek enthaltenen 
Strategien nicht reichen, so bietet \lstinline{hypothesis} mit seinen 
First-party Erweiterungen, Erweiterungen die zu 100\% mit 
\lstinline{hypothesis} funktionieren. Dadurch ist die Standardbibliothek 
besonders schlank.

\lstinline{hypothesis} selbst lässt den Entwickler sehr schnell einsteigen, da 
nicht sonderlich viel Vorarbeit notwendig ist um \lstinline{hypothesis} an zu 
wenden. Allerdings kommt mit den steigenden Anforderungen an die Daten auch die 
steigende Vorarbeit, die nötig ist um \gls{fuzz}-testing zu betreiben.

Die Komplexität von \lstinline{hypothesis} ist allerdings sehr hoch. So viel wie
\lstinline{hypothesis} bietet, so komplex kann es sein, dieses an zu wenden. 
Hat man spezielle Anforderungen abseits der Standardwerte, wie \lstinline{int} 
oder \lstinline{str} ist mehr Arbeit und Verständnis notwendig. Dies kann dazu 
führen, dass man als Entwickler etwas Zeit investieren muss, um die 
Spezifikationen auf eine Strategie an zu wenden. Je nach Komplexität der 
erforderlichen Daten ist es also leicht bis schwer \lstinline{hypothesis}
zu verwenden. Bei komplexeren Anforderungen kann es dazu führen, dass der 
geschriebene Code unübersichtlich wird, wenn er nicht richtig dokumentiert 
wurde.

Auch wenn \lstinline{hypothesis} als Erweiterung in dieser Arbeit behandelt 
wird, so bietet es Erweiterungen für sich selbst. Dies liegt daran, dass man 
den Entwicklern nicht standardmäßig alle Funktionalität an die Hand gibt, die 
vermutlich gar nicht verwendet wird. Und auch wenn \lstinline{hypothesis} über 
sehr viele Strategien verfügt, so hat die Community dennoch weitere Strategien 
entwickelt die zusätzlich genutzt werden können.

% !TeX root = ../bachelor.tex
\subsection{unittest}\label{python-tools:unittest}

Blubb

