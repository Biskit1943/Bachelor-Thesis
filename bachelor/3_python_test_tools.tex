% !TeX root = ../bachelor.tex
\section{Python Test-Tools}\label{python-tools}

Dieses Kapitel befasst sich mit den von der STDLIB bereitgestellten Test-Tools
sowie denen aus externen Paketen.
Diese werden unter \ref{python-tools:stdlib} und \ref{python-tools:extlib}
zusammengefasst, wobei diese unterteilt sind in unit-testing -,
\gls{mock}-testing - und \gls{fuzz} Tools.

Die unit-testing Tools sind Tools die Funktionalität zum testen bereitstellen.
Jedoch wird für TDD weit mehr als nur Tests benötigt, aus diesem Grund werden
\gls{mock}-testing - und \gls{fuzz} Tools zusätzlich behandelt. Dabei soll aber
zwischen einer reinen Erweiterung eines Tools und der Erweiterung von allen Tools
unterschieden werden.
Ist zum Beispiel ein Tool nur zusammen mit einem anderen Tool, das die
Funktionalität bereit stellt Tests aus zu führen, so ist dieses Tool als
Erweiterung zu sehen und wird nicht Analysiert. Bietet ein Tool allerdings
Funktionalität zum erweitern von verschiedenen Test Tools, so wird es 
hier zur Analyse verwendet. Diese Unterscheidung wird verwendet um Duplikationen
in den einzelnen vergleichen zu vermeiden und verhindert, dass Tools mit sehr
vielen Erweiterungen den Rahmen dieser Arbeit sprengen.Sollten sich für ein Tool
besonders interessante Erweiterungen finden, so werden diese in der Analyse des
jeweiligen Tools erwähnt und verlinkt.
\newline
\\
Jedes Tool wird anhand folgender Aspekte untersucht:
\begin{itemize}
    \item \underline{Anwendbarkeit}:\newline
    Bietet das Tool alles, um TDD betreiben zu können? (\Glspl{fixture} und \Glspl{mock})
    Mit wie vielen Paketen muss das Tool betrieben werden? (Mehr Abhängigkeiten führen zu
    mehr externe Entwicklern auf die man sich verlassen muss.)
    \begin{itemize}
        \item Bei \gls{mock}-testing- und \gls{fuzz}-testing-Tools wird hier auf die
        Features die das jeweilige Tool bietet geprüft.
    \end{itemize}
    
    \item \underline{Effizienz}:\newline
    Wie viel lässt sich mit diesem Tool möglichst einfach und
    schnell erreichen? Ist besonders viel Vorarbeit notwendig um die Tests
    auf zu setzen oder kann sofort mit dem Schreiben der Tests begonnen
    werden?
    \newline
    Genauso stellt sich die Frage, wie Effizient der Entwickler die Tests
    auswerten kann (Nur bei Tools mit test-runner).
    
    \item \underline{Komplexität}:\newline
    Wie komplex ist das Tool? Das heißt, wie viel Funktionalität
    bietet das Tool dem Entwickler von Haus aus, aber auch wie schwer
    ist es einen Code zu schreiben oder wie schnell wird ein Code unübersichtlich, da
    das Tool viel Code abseits der Tests benötigt.
    
    \item \underline{Erweiterbarkeit}:\newline
    Wie leicht lässt sich das Tool mit anderen Tools erweitern?
    Gibt es vielleicht Erweiterungen der Community für dieses Tool, die sehr
    hilfreich sind?
    \begin{itemize}
        \item Dieser Punkt wird bei \gls{mock}-testing - und \gls{fuzz}-testing Tools ignoriert,
        da diese selbst Erweiterungen darstellen.
    \end{itemize}
\end{itemize}
\noindent
Zum Vergleich der einzelnen unit-testing Tools untereinander werden diese auf den in Listing
\ref{listing:base:my_module} abgebildeten Code angewandt, da sich diese Arbeit auch mit
\gls{mock}-testing Tools beschäftigt, wurde auch für diese Code geschrieben der zu testen ist.
Der Code dazu befindet sich in Listing \ref{listing:base:my_mock_module}. Durch diesen Code
lassen sich die unterschiedlichen Anforderungen zwischen den Arten der Tools besser vergleichen.

Das Modul aus Listing \ref{listing:base:my_module} für die unit-testing Tools
enthält eine selbst geschriebene Implementierung der Funktion \lstinline|pow()|,
welche eine Zahl \lstinline|a| mit einer Zahl \lstinline|b| quadriert. Um die
Komplexität zu erhöhen wurde eine in-memory Datenbank implementiert, welche
eine Items Tabelle enthält. Jedes Item hat eine ID (\lstinline|id|), einen Namen
(\lstinline|name|), einen Lager Platz (\lstinline|storage_location|) und eine
Anzahl der vorrätigen Items \lstinline|amount|. Dazu besitzt jedes Item eine
Methode \lstinline|do_something()| welche eine externe Funktion/Methode aufruft,
die jedoch noch nicht existiert \lstinline|do_something_which_does_not_exist()|.

Für die \gls{mock}-testing Tools wurde eine Klasse geschrieben, welche Funktionen besitzt
die in ihrem Namen ausdrücken welche Aktion sie ausführen, jedoch führt nicht jede Methode
das gewünschte aus. Die \gls{mock}-testing Tools sollen hier die Methoden so \gls{mock}en, dass
sie das jeweilige ausführen. Selbstverständlich ist realer Code viel komplexer als in diesem
Listing (\ref{listing:base:my_mock_module}) dargestellt, jedoch reicht dieser Code aus um viele
Tools an Ihre grenzen zu bringen. So muss in \lstinline{call_internal_functin_n_times} überprüft
werden ob sie \lstinline{n} mal aufgerufen wurde, auch \lstinline{call_helper_help} wird nicht
ohne weiteres funktionieren, da die Methode der Klasse \lstinline{Helper} noch nicht implementiert wurde. Auch die Methode \lstinline{return_false_filepath}
bringt das ein oder andere Tool zum schwitzen, da eine externe Methode aus der STDLIB ersetzt
werden muss, wobei dies nicht global geschehen darf.

Für die Fuzz-testing Tools werden keine Beispiele geschrieben werden, da diese meist auf Komplexen
Beispielen basieren um Sinn zu ergeben, stattdessen werden im jeweiligen Kapitel kleinere Beispiel
genannt, wie das Tool ein zu setzen ist oder es wird auf Beispiele aus der jeweiligen Dokumentation
verwiesen.

Verfügt ein Tool über keinen test-runner, so wird der von der STDLIB gestellte runner
\lstinline{unittest} verwendet.

% !TeX root = ../bachelor.tex
\subsection{Tools der Standard Bibliothek}\label{python-tools:stdlib}

Die Standard Bibliothek von Python bietet zwei verschiedene Test-Tools (\cite{wiki.python:PythonTestingToolsTaxonomy}).
Zum einen ist dies
\href{http://pyunit.sourceforge.net/pyunit.html}{unittest}\footnote{http://pyunit.sourceforge.net/pyunit.html}
und zum anderen
\href{https://docs.python.org/3/library/doctest.html}{doctest}\footnote{https://docs.python.org/3/library/doctest.html}.
Diese beiden Tools reichen sind im ihrem Umfang bereits so vielseitig dass, es einfach ist eine hohe Test-Abdeckung eines Programms oder eine Bibliothek zu erreichen.
\newline
\\
Beide Tools zählen zu den "`Unit Testing Tools"' (\cite{wiki.python:PythonTestingToolsTaxonomy})
auf Deutsch Modul Test-Tools, mit deren Hilfe die einzelnen Module eines Programms
getestet werden können. In einem Programm oder einer Bibliothek wären dies die einzelnen
Funktionen und Methoden.

% !TeX root = ../bachelor.tex
\subsubsection{unittest}\label{python-tools:unittest}

Das von JUnit inspirierte (\cite{docs.python:unittest}) Tool unittest, ist bestand
der Python Standardbibliothek und bietet seit jeher seinen Nutzern ein umfangreiches
Repertoire an Funktionen zum testen von Python Code.

Die Funktionen von unittest lassen sich unter folgenden Punkten beschreiben:
\begin{itemize}
    \item \Gls{fixture}, zum präparieren der Tests.
    \item Test Fälle, zum gliedern einzelner Tests.
    \item Testumgebungen, zum gliedern von zusammengehörigen Tests.
    \item Test runner, zum ausführen von Testumgebungen oder Test Fällen.
\end{itemize}

Mithilfe der genannten Punkte ist es dem Entwickler möglich eine Stabile Test Umgebung
auf zu bauen. Jedoch bietet unittest alleine nicht alles um TDD betreiben zu können.

Bei TDD werden zuerst die Tests und dann die Funktionalitäten geschrieben, daher
muss es möglich sein andere Module(units) zu \gls{mock}en auf denen ein Test basiert.
Durch die \Glspl{fixture} ist es bereits möglich den Test oder die Tests so vor zu
bereiten, dass diese funktionieren, jedoch bieten \Glspl{mock} einfachere und
schnellere Möglichkeiten Funktionen, Methoden, Klassen usw. zu imitieren.

Jedoch gibt es in der STDLIB eine Erweiterung zu unittest mit dem Namen
\lstinline|unittest.mock| welche unter eben diesem importiert werden kann um die \gls{mock}ing Funktionalität
zu bekommen.

Das Tool bietet des weiteren einen CLI, mit welchem es dem Benutzer möglich ist
seine Tests gebündelt aus zu führen und aus zu werten. Mit dem CLI ist es auch
auch möglich automatisch Tests in einem Ordner zu "`entdecken"'(discover) und
aus zu führen. Dadurch ist es sehr leicht neue Tests in ein bestehendes Test
System ein zu führen und diese ohne Veränderungen am bestehenden System aus zu
führen.
\newline
\\
Mit Hilfe von unittest lässt sich sehr einfach und schnell ein Test schreiben,
so würde das der Code aus Listing \ref{listing:unittest:example} bereits unsere
quadrat-Funktion testen, einmal mit unserem selbst berechneten Wert und
einmal gegen den wert der quadrat-Funktion aus der STDLIB.

\lstinputlisting[
    language=Python,
    label=listing:unittest:example,
    caption=unittest einfaches Beispiel
]{analyse/unittest_/example.py}

Die basis-Funktionalität von unittest ist schnell verstanden und setzt sich
eigentlich nur aus \lstinline|self.assert{irgendwas}(...)| zusammen. Hat man die
richtige assert Funktion gefunden lässt sich eigentlich jede unabhängige
Funktion testen.

Möchte man allerdings fortgeschrittenere Tests schreiben so muss man sich der
Dokumentation bedienen, welche unter \url{https://docs.python.org/3/library/unittest.html}
zu finden ist. Würde man die Seite als PDF herunterladen so wären dies 58
Seiten Fließtext. Möchte man nun zum Beispiel vor den Tests etwas vorbereiten
oder präparieren so lässt sich mit \lstinline|setUp()| und \lstinline|tearDown()| dies realisieren,
diese zwei Methoden überschreiben die Methoden aus \lstinline|unittest.TestCase| und
werden vor jeder Funktion ausgeführt. Das Gleiche gibt es auch für den Test Fall, bei dem
\lstinline|setUp()| und \lstinline|tearDown()| allerdings nur beim eintritt der Klasse und beim
austritt ausgeführt werden. Diese Methoden sind die sogenannten \Glspl{fixture}.

Das folgende Listing zeigt wie
% !TeX root = ../bachelor.tex
\subsubsection{doctest}\label{python-tools:doctest}

\begin{itemize}
    \item keine Abhängigkeiten, da STDLIB
    \item gleicht der Python shell
    \item integrierbar mit unittest
    \item Tests als Dokumentation
    \item Kein assert, Output wird überprüft
\end{itemize}

% !TeX root = ../bachelor.tex
\subsection{Tools abseits der Standard Bibliothek}\label{python-tools:extlib}
Abseits der Standard Bibliothek gibt es einige Tools deren Nutzung von
Vorteil gegenüber der STDLIB ist. Diese werden unter diesem Punkt aufgeführt.
\newline
\\
Auf \url{https://wiki.python.org/moin/PythonTestingToolsTaxonomy} werden viele
externe Tools gelistet, jedoch scheinen viele inaktiv zu sein da ihre
\glspl{commit} teilweise mehr als ein Jahr zurück liegen. Dies ist weder für
eine Bibliothek noch für ein Tool ein gutes Zeichen, da sich die Anforderungen
stetig ändern und niemals alle Bugs gefixt sind.
\newline
\\
Manche der dort aufgelisteten Tools sind bereits oder werden gerade in andere Tools
integriert. Dies kann zum einem sein, da ein Tool eine Erweiterung für ein anderes
war und die Entwickler die Änderungen angenommen haben und zum anderen um die Tools
zu verbessern und mehr Entwickler zur Verfügung zu haben. Eventuell sind auch
andere Gründe dafür verantwortlich, jedoch war dies der Grund bei zum Beispiel
\href{https://github.com/Yelp/Testify/}{Testify}
von Yelp \footnote{https://github.com/Yelp/Testify/}.
\newline
\\
Da sowohl Software als auch Programmiersprachen sich stetig weiter entwickeln
werden in dieser Arbeit nur jene Tools behandelt die sich diesen Entwicklungen
Anpassen, sei dies durch das unterstützen der aktuellsten Version von Python als
auch durch neue Innovationen, sowie \Gls{bug}-fixes. Aus diesem Grund werden
Tools deren letzter \Gls{commit} weiter als ein Jahr zurück liegt hier nicht
behandelt.
\newline
\\
Lässt man zusätzlich die Erweiterungen von Tool zunächst außen vor, so bleiben
folgende Modul Test-Tools zur Verfügung
(\cite{wiki.python:PythonTestingToolsTaxonomy}):
\begin{itemize}
    \item \href{https://github.com/pytest-dev/pytest/}{pytest}\footnote{https://github.com/pytest-dev/pytest/}
\end{itemize}
Tools wie \href{https://www.reahl.org/docs/4.0/devtools/tofu.d.html}{reahl.tofu}\footlabel{reahl.tofu}{https://www.reahl.org/docs/4.0/devtools/tofu.d.html} oder
\href{https://pypi.org/project/zope.testing/}{zope.testing}\footlabel{zope.testing}{https://pypi.org/project/zope.testing/} sind zwar mehr oder weniger aktiv, da sie
beide auf die neusten Python Versionen patchen, jedoch bieten sie sonst keinen Mehrwert in ihren Updates.
\href{https://www.reahl.org/docs/4.0/devtools/tofu.d.html}{reahl.tofu}\footref{reahl.tofu} selbst ist
auch nur eine Erweiterung für bestehende Test Tools, wie zum Beispiel pytest, weshalb es hier nicht behandelt wird. So wird
\href{https://pypi.org/project/zope.testing/}{zope.testing}\footref{zope.testing} auch nicht behandelt, da dieses Tool
wie bereits beschrieben nicht aktiv weiter entwickelt wird und es eher dafür gemacht wurde \href{http://www.zope.org/en/latest/}{zope}\footnote{http://www.zope.org/en/latest/}
Applikationen zu testen und nicht Python Applikationen.

Der Test-Runner \href{https://pypi.org/project/nose/1.3.7/}{nose}\footnote{https://pypi.org/project/nose/1.3.7/}, der eine Erweiterung
zu \nameref{python-tools:unittest} bietet ist nicht mehr in Entwicklung, dennoch haben sich ein paar Liebhaber des Tools zusammengeschlossen
und \href{https://pypi.org/project/nose2/}{nose2}\footlabel{nose2}{https://pypi.org/project/nose2/} geschrieben.
Da \href{https://pypi.org/project/nose2/}{nose2}\footref{nose2} wie sein Vorfahre eine Erweiterung zu \nameref{python-tools:unittest}
darstellt wird es hier nicht analysiert.

Auch sehr bekannt ist \href{https://pypi.org/project/testtools/}{testtools}\footnote{https://pypi.org/project/testtools/}, welches jedoch auch als
Erweiterung zu \nameref{python-tools:unittest} gesehen werden muss. Jedoch findet dort auch keine wirkliche weiter Entwicklung statt.

Die Quelle aus dem Offiziellen Python Wiki beschreibt sonst keine weiteren
Tools. Auch die suche auf einschlägigen Suchmaschinen liefert sonst keine
weiteren relevanten Tools.
\newline
\\
Als weitere Kategorie werden \Gls{mock}-Tools geführt. Auch wenn fast alle
unittest-Tools integriertes \gls{mock}ing haben, so lässt sich mit diesen
Tools meist mehr erreichen. Es wurden die gleichen Filter-Kriterien verwendet
wie bei den unit-testing Tools(\cite{wiki.python:PythonTestingToolsTaxonomy}).
\begin{itemize}
    \item \href{https://www.reahl.org/docs/4.0/devtools/stubble.d.html}{stubble}\footnote{https://www.reahl.org/docs/4.0/devtools/stubble.d.html}
    \item \href{https://github.com/timbertson/mocktest/tree/master}{mocktest}\footnote{https://github.com/timbertson/mocktest/tree/master}
    \item \href{https://github.com/bkabrda/flexmock}{flexmock}\footnote{https://github.com/bkabrda/flexmock}
    \item \href{https://bitbucket.org/DavidVilla/python-doublex}{python-doublex}\footnote{https://bitbucket.org/DavidVilla/python-doublex}
    \item \href{https://github.com/ionelmc/python-aspectlib}{python-aspectlib}\footnote{https://github.com/ionelmc/python-aspectlib}
\end{itemize}
Auch hier lassen sich sonst keine weiteren relevanten Tools finden.
\newline
\\
Als letzte Kategorie werden hier die \Gls{fuzz} Tools behandelt, da diese eine
gute Möglichkeit bieten Code ausgiebig zu testen. Das wohl umfangreichste und nach den obigen Kriterien
einzige Tool ist \href{https://github.com/HypothesisWorks/hypothesis}{hypothesis}\footnote{https://github.com/HypothesisWorks/hypothesis} (\cite{wiki.python:PythonTestingToolsTaxonomy}).

% !TeX root = ../bachelor.tex
\subsubsection{pytest}\label{python-tools:pytest}\mbox{}
\newline
Das am 04. August 2009 in Version
1.0.0\footnote{https://github.com/pytest-dev/pytest/releases/tag/1.0.0}
veröffentlichte Tool \lstinline{pytest} (auch py.test genannt)
ist ein sehr umfangreiches und weit entwickeltes Tool. Seit 2009 wird das Tool
stets weiter entwickelt und vorangetrieben, wodurch es eine Menge an Features
gewonnen hat.
\noindent
Die Basis Features von \lstinline{pytest} sind folgende(\cite{docs.pytest.org:4.4}):
\begin{itemize}
    \item Simple \lstinline{assert} statements.
    \subitem Kein \lstinline{self.assert}
    \subitem Error Überprüfung mit \Gls{context}
    \item Informativer Output in Farbe
    \subitem Der gesamte Output kann angepasst werden
    \item Feature reiche \Glspl{fixture}
    \subitem Vordefinierte \Glspl{fixture} von \lstinline{pytest}
    \subitem Geteilte \Glspl{fixture} unter Tests
    \subitem Globale \Glspl{fixture} zwischen Modulen
    \subitem Parametrisierung von Test als \Glspl{fixture}
    \item Überprüfung von \lstinline{stdout} und \lstinline{stderr}
    \item Gliedern von Test in Fällen
    \subitem ... durch Markierung
    \subitem ... durch Nodes (Auswahl der Modul Abhängigkeit)
    \subitem ... durch String Abgleich der Funktions-Namen
\end{itemize}
\noindent
Wie anhand der Basis Features erkennbar ist, bietet \lstinline{pytest} einiges
um Tests zu schreiben und aus zu führen. So ist mit den gebotenen
\Glspl{fixture} bereits eine Voraussetzung für TDD erfüllt, da
\lstinline{pytest} viele verschiedene Arten bietet diese zu benutzen.
\Glspl{fixture} werden in \lstinline{pytest} allerdings nicht mit
\lstinline{setUp} und \lstinline{tearDown} geschrieben, sondern werden mithilfe
eines \glspl{decorator} markiert und den Test Funktionen als Parameter
übergeben. Um die \lstinline{setUp} und \lstinline{tearDown} Funktionalität zu
bekommen muss lediglich das keyword \lstinline{yield} verwendet werden, die
\Gls{fixture} wird dann den Code bis zum \lstinline{yield} ausführen und nach
der Funktion den Rest nach \lstinline{yield} ausführen.
\newline
\\
Diese lassen sich durch zusätzliche Parameter weiter anpassen. Diese
Funktionen sind in Kapitel fünf nach \cite{docs.pytest.org:4.4} beschrieben.

Des weiteren bietet \lstinline{pytest} auch \gls{mock}ing, so lässt sich beispielsweise mit
\lstinline{monkeypatch.setattr()} der return Wert einer Funktion ersetzen. Dies wird
in \lstinline{pytest} automatisch am ende der Funktion rückgängig gemacht, wodurch der Entwickler
sich mehr auf das eigentliche Testen konzentrieren kann.
\newline
\\
Auch hier bietet \lstinline{pytest} weitere Möglichkeiten \Glspl{mock} zu verwenden, dazu hat
\cite{docs.pytest.org:4.4} in Kapitel sieben einige Worte geschrieben.

Ein weiteres sehr praktisches Feature von \lstinline{pytest} ist die Gliederung in Fälle. Dies ist bei \lstinline{pytest}
sehr fein einstellbar, so lässt sich wie in den Basis Features bereits beschrieben ein Test mit
einer oder mehreren Markierungen versehen wodurch eine Gliederung nach Markierung entsteht. Auch
durch das selektieren bestimmten Wörtern lassen sich Tests nach ihrem Namen gliedern, so entsteht
einerseits eine Gliederung zur Ausführung der Tests und andererseits eine Gliederung für die
Entwickler die sie selbst im Code sehen können. Als letzte alternative lassen sich Tests anhand
ihrer Module ausführen, dies geschieht durch die Angabe der Module. So würde
\lstinline{test_datei.py::TestKlasse::test_methode} den Test \lstinline{test_methode} der Klasse
\lstinline{TestKlasse} in der Datei \lstinline{test_datei.py} ausführen.
\newline
\\
Diese Features werden in Kapitel sechs von \cite{docs.pytest.org:4.4} beschrieben.

Selbstverständlich bietet \lstinline{pytest} noch weitere Features jedoch sind diese nicht zwangsläufig notwendig
um TDD zu betreiben und sind mehr ein nice to have Feature als wirklich benötigt. Für eine
Vollständige Auflistung und Erklärung aller Features kann jederzeit unter
\url{https://docs.pytest.org/en/latest/} die aktuellste Version der Dokumentation abgefragt werden.

Da \lstinline{pytest} nicht in der STDLIB ist, muss es mit einem \Gls{paket} installiert werden.
Dabei werden für \lstinline{pytest 4.4.0} die in Listing \ref{listing:pytest:requirements}
gezeigten Abhängigkeiten installiert. Demnach benötigt \lstinline{pytest} sechs externe Abhängigkeiten
zusätzlich zu sich selbst.
Demnach bietet \lstinline{pytest} alles um TDD anwenden zu können und die sieben zusätzlichen Abhängigkeiten
die ein Entwickler bei der Nutzung von \lstinline{pytest} eingeht sollten keinen Entwickler davon abhalten
\lstinline{pytest} und seine Features zu genießen.
\newline
\\
Da \lstinline{pytest}, wie bereits erwähnt keine neuen assert Methoden hinzufügt lässt sich sehr schnell und
einfach ein Test schreiben. Selbst die Nutzung von \Glspl{fixture} ist in \lstinline{pytest} sehr einfach, da
lediglich eine Funktion geschrieben werden muss die mit \lstinline{@pytest.fixture} markiert wurde
und der Test Funktion oder Methode als Parameter übergeben wird. Den Rest erledigt \lstinline{pytest} selbst
im Hintergrund. Genauso einfach gestaltet sich die Nutzung von \Glspl{mock}.

Die Effizienz, die bei der Nutzung von \lstinline{pytest} entsteht ist demnach sehr hoch. Denn der Entwickler
muss nicht wirklich etwas neues dazu lernen um Tests zu schreiben oder zu präparieren. Genauso
leicht kann ein Entwickler den Output von \lstinline{pytest} auswerten, da dieser erstens, in Farbe ist, was
die Lesbarkeit deutlich erhöht, zweitens sehr gut gegliedert ist und Wichtige Strukturen klar
darstellt und drittens nach den Wünschen der Entwickler sich gestalten und verbessern lässt.
\newline
\\
Durch die eben genannten Features im Bezug auf das schreiben von Tests mit assert, sowie
\Glspl{fixture} und \Glspl{mock} lässt sich sagen das \lstinline{pytest} sehr viel Funktionalität bietet
wobei es trotzdem Struktur im Code bietet und komplexe Features einfach ermöglicht. Demnach
kann ein Entwickler mit wenig Code viel Testing Funktionalität erstellen.
\newline
\\
Sollten dem Entwickler die Features von \lstinline{pytest} nicht ausreichen, so findet man unter
\url{http://plugincompat.herokuapp.com/} eine Liste von 618 (Stand: 2. April 2019) Erweiterungen
für \lstinline{pytest 4.3.0}. \lstinline{pytest} selbst kann allerdings auch als Erweiterung zu unittest
genutzt werden, indem ein Entwickler \lstinline{pytest} zum ausführen der unittests verwendet. Dadurch bietet
sich dem Entwickler ein verbesserter Output des Test Ergebnisses.
\newline
\\
Im Listing \ref{listing:pytest:advanced} und \ref{listing:pytest:advanced_output} befindet sich der
Code und der Output zum Test von dem in Listing \ref{listing:base:my_module} definierten Modul.
% !TeX root = ../bachelor.tex
\subsubsection{Mocking Tools}\label{python-tools:extlib:mock}

Blubb

% !TeX root = ../bachelor.tex
\paragraph{stubble}\label{python-tools:stubble}\mbox{}
\newline
Stubble zählt zwar zu den \gls{mock}ing Tools, jedoch werden hier nur
\Glspl{stub} als Funktionalität gegeben. Demnach ist es zwar möglich Objekte
zu ersetzen und Rückgabewerte zu fälschen, jedoch lässt sich nicht verfolgen
wie oft und mit welchen Parametern ein Objekt aufgerufen wurde.

Angenommen ein Objekt \lstinline{class Original} hat eine Abhängigkeit zu dem zu
testenden Objekt. Mit \lstinline{stubble} lässt sich ein Objekt
\lstinline{class Fake} erstellen, welches einen \gls{decorator} besitzt
\lstinline{@stubclass(Original)}. Dadurch stellt \lstinline{stubble} sicher,
dass alle Methoden aus \lstinline{class Fake} der Signatur derer aus
\lstinline{class Original} entsprechen. Zusätzlich kann \lstinline{class Fake}
von \lstinline{class Original} erben, wodurch nur die Funktionen überschrieben
werden, die in \lstinline{class Fake} definiert wurden.

Um Helfer Methoden zu \lstinline{class Fake} hinzu zu fügen, müssen diese mit
\lstinline{@exempt} annotiert werden. Ist das Objekt von
\lstinline{class Original} bereits initialisiert, so lässt es sich dieses an
\lstinline{class Fake} übergeben, wenn diese von
\lstinline{reahl.stubble.Delegate} erbt. Dadurch ist es möglich zur Laufzeit
ein Objekt durch einen \Gls{stub} ersetzen. Als weiteres Feature wird die
Möglichkeit geboten \Glspl{stub} mit \lstinline{setuptools} zu verwenden um
Installationen zu \gls{stub}ben.
\newline
\\
Die Hauptfeatures von stubble sind jedoch die vorgefertigten \Glspl{stub}, So
werden dem Entwickler folgende \Glspl{stub} geboten:
\begin{itemize}
    \item \href{https://www.reahl.org/docs/4.0/devtools/stubble.d.html\#systemoutstub}{SystemOutStub}\footnote{https://www.reahl.org/docs/4.0/devtools/stubble.d.html\#systemoutstub}
    \subitem Ein \Gls{stub}, der den Standard Out ersetzt
    \item \href{https://www.reahl.org/docs/4.0/devtools/stubble.d.html\#callmonitor}{CallMonitor}\footnote{https://www.reahl.org/docs/4.0/devtools/stubble.d.html\#callmonitor}
    \subitem Ein \Gls{stub}, zum überprüfen welche Methoden wann, wie und wie oft aufgerufen wurde
\end{itemize}
\noindent
Diese \Glspl{stub} lassen sich als \glspl{context} einsetzen, wodurch es
leichter wird Code mit diesen zu schreiben. Ein weiterer \Glspl{context} ist
\lstinline{replaced}, dieser lässt den Entwickler eine Methode oder Funktion
durch eine andere ersetzen wodurch, solange der \glspl{context} aktiv ist, die
neue Methode/Funktion genutzt wird.
\newline

Des weiteren bietet stubble einige Experimentelle Features die, jedoch eher
weniger für den praktischen Einsatz geeignet sind (\cite{reahl.stubble:4.0}),
aber dennoch interessante Features darstellen. So ist es möglich
\lstinline{class Fake} von \lstinline{reahl.stubble:Impostor} erben zu lassen,
wodurch jede Instanz von \lstinline{class Fake} eine Instanz von
\lstinline{class Original} ist. Das Beispiel dazu ist in Listing
\ref{listing:reahl.stubble:Impostor} zu finden.

Ein weiteres experimentelles Feature ist das ersetzen eines bereits bestehenden
Objekts (\lstinline{Delegation}). Dies wurde bereits in den Hauptfeatures
beschrieben, da es als sehr praktisch an zu sehen ist. Lediglich die Instanz
Variablen eines Objektes machen hierbei Probleme, da diese nicht mit einem
\Gls{stub} ersetzt werden können. So werden auch Objekt variablen die von
Originalen Methoden gesetzt oder verändert werden nicht im \Gls{stub} gesetzt.
Aus diesem Grund wird \lstinline{Delegation} als Experimentelles Features
gesehen, da ein Bug, der durch dieses Feature entstanden ist schwer zu finden
ist (\cite{reahl.stubble:4.0}).
\newline

Der Code für den Vergleichstest von stubble ist in Listing
\ref{listing:reahl.stubble:example} zu finden. Mit stubble alleine lässt sich
dieser in Listing \ref{listing:base:my_mock_module} definierte Code nicht zu 100\%
optimal testen.
So ist die nicht existierende Methode \lstinline{Helper.help(self)} nicht
ohne Probleme zu ersetzen, lediglich wenn in der \Gls{stub} Klasse die Methode
\lstinline{help(self)} definiert und sie mit \lstinline{@exempt} annotiert wird.
Jedoch verliert man dadurch die Funktionalität von stubble, welche überprüft ob
die Signatur der Funktionen übereinstimmen. Alternativ hätte man auch einfach
eine Klasse nehmen können die ohne \lstinline{@stubclass} auskommt. Ansonsten
Zeigt der Code wie ein recht simples Tool viel erreichen kann, wenn auch mit
viel Code abseits der Tests. Dabei lies sich \lstinline{os.path.abspath(path)}
ohne Komplikationen mithilfe von \lstinline{reahl.stubble.replaced} ersetzen und
das abfangen des STDOUTs mithilfe der vorgefertigten \Glspl{stub} erwies sich
auch als unkompliziert.
\newline

Im Bezug auf die Anwendbarkeit für TDD ist stubble demnach ein Tool das durchaus
in Betracht gezogen werden kann, auch die Abhängigkeiten halten sich mit einer
Abhängigkeit
(\href{https://github.com/benjaminp/six}{six}\footnote{https://github.com/benjaminp/six})
in grenzen.

Die Effizienz mit der ein Entwickler stubble einsetzen kann ist hoch, denn abgesehen von einem
\gls{decorator} benötigt der Entwickler nichts weiter um Objekte zu ersetzen.

Lediglich die Komplexität von stubble ist nicht sehr hoch, zwar wird dem Entwickler alles geboten um
ein Objekt zu ersetzen, jedoch aber auch nicht mehr. Zusätzlich bietet stubble mit seinen
vorgefertigten \Glspl{stub} eine gute schnelle Möglichkeit den Stdout zu ersetzen oder ein Objekt
zu Monitoren.
Um jedoch selbst ein Objekt zu ersetzen wird vergleichsweise viel Code benötigt, da
nicht einfach nur der Rückgabe Wert neu gesetzt wird sondern die Methode neu geschrieben werden muss.
Dies kann allerdings bei Komplexeren Anforderungen wiederum zu Vorteil werden weshalb sich das ganze
wiederum ausgleicht.
Durch die \Glspl{context} ist der Code der geschrieben wird sehr übersichtlich und gut gegliedert.
% !TeX root = ../bachelor.tex
\paragraph{mocktest}\label{python-tools:mocktest}\mbox{}
\newline
\lstinline{mocktest} ist ein \gls{mock}ing Tool, dass von
\href{http://rspec.info/}{\lstinline{rspec}}\footlabel{rspec}{http://rspec.info/}
inspiriert wurde, dabei soll \lstinline{mocktest} kein
Port von \lstinline{rspec}\footref{rspec} sein, sondern eine kleine leichtere
Version in Python(\cite{mocktest:doc}).
Da sich \lstinline{mocktest} derzeit in der Version \lstinline{0.7.3} (Stand 16.
April 2019) befindet, sind noch nicht alle Features vollkommen entwickelt und
ausgereift, dennoch lässt sich das Tool bereits Produktiv einsetzen, da die
Basis Funktionalität aus \lstinline{rspec}\footref{rspec} bereits implementiert
wurde.
\newline

Das Hauptfeature von \lstinline{mocktest} ist die Test Isolation, diese
verhindert, dass \Gls{mock} Objekte außerhalb eines Test Falles weiterhin
Veränderungen vornehmen und so andere Tests beeinflussen. Demnach werden Tests
immer innerhalb einer \lstinline{mocktest.MockTransaction} ausgeführt, sofern
diese einen oder mehrere \Glspl{mock} nutzen.

Innerhalb dieses Kontextes stehen dem Entwickler verschiedene Möglichkeiten zur
Verfügung Tests aus zu führen. So lässt sich Beispielsweise überprüfen ob eine
Funkion oder Methode aufgerufen wurde, dabei spiel es keine Rolle ob es sich
hierbei um ein Globales Objekt handelt oder ein Lokales.
Des weiteren lassen sich auch \Glspl{stub} nutzen, mit deren Hilfe Funktionen
und Methoden präpariert werden können. Wie auch beim überprüfen des Aufrufs
spielt es hier keine Rolle ob es sich um ein Lokales oder Globales Objekt
handelt.
Die Präparation lässt sich nach bedarf auch anpassen, so kann der Entwickler
beispielsweise festlegen, dass nur bei einem bestimmten Aufruf mit bestimmten
Parametern das \Gls{mock} Objekt aufgerufen wird. Andernfalls wird das Originale
Objekt genutzt.
\newline

\lstinline{mocktest} nutzt \lstinline{unittest} als Basis für seine
Testumgebung, \lstinline{mocktest.TestCase} erbt von
\lstinline{unittest.TestCase}, wodurch die gesamte Funktionalität von
\lstinline{unittest} geerbt wird.

\noindent
\lstinline{mocktest.TestCase} führt dabei in seiner \lstinline{setUp()} und
\lstinline{tearDown()} Methode Code aus, der benötigt wird um
\lstinline{mocktest} Nutzen zu können. Zusätzlich wird mit \lstinline{mocktest}
unittest um eine \lstinline{assert} Methode erweitert,
\lstinline{assertRaises()} verfügt über verbesserte Funktionalität zum
überprüfen auf Exceptions. Ist es jedoch erwünscht unittest nicht zu verwenden,
so ist es möglich \lstinline{mocktest.MockTransaction} als Kontextmanager mit
\lstinline{with MockTransaction:} zu nutzen.
\lstinline{MockTransaction.__enter__()} und
\lstinline{MockTransaction.__exit__()} sind die Methoden die in
\lstinline{setUp()} und \lstinline{tearDown()} automatisch aufgerufen werden,
weshalb es möglich ist \lstinline{mocktest} mit jedem Test-runner zu verwenden,
der \Glspl{fixture} unterstützt.
\newline

Zum überprüfen von \Glspl{mock} kann entweder \lstinline{when(obj)} oder
\lstinline{expect(obj)} verwendet werden. Der unterschied hier ist lediglich,
dass \lstinline{when(obj)} nicht überprüft ob eine Methode aufgerufen wurde oder
nicht. Wird hingegen aufgestellt, dass \lstinline{expect(os).system}, dann wird
ein Fehler geworfen, wenn \lstinline{os.system} nicht aufgerufen wurde, was bei
\lstinline{when(os).system} nicht der Fall ist. Möchte man später überprüfen wie
oft und mit was ein Objekt aufgerufen wurde, ist es möglich
\lstinline{received_calls} ab zu prüfen, dies ist eine Liste von allen
getätigten aufrufen und Ihren Parametern.
\newline

\noindent
Eine Auflistung der Features von \lstinline{mocktest} (\cite{mocktest:doc}):
\begin{itemize}
    \item Ergebnisvariation mit \lstinline{.and_return(erg1, erg2, ergX)},
    alternativ ist \lstinline{.then_return()} die gleiche Methode.
    
    \item Mit \lstinline{.at_least(n)}, \lstinline{.at_most(n))},
    \lstinline{.between(a, b)} und \lstinline{.exact(n)} lässt sich festlegen
    wie viele aufrufe erfolgt sein müssen.
    
    \begin{itemize}
        \item Als alias wird \lstinline{.never()} und \lstinline{.once()} geboten für nie und einmal aufgerufen.
    \end{itemize}

    \item Durch \lstinline{.then_call(func)} wird die Funktion oder Methode
    \lstinline{func} ausgeführt anstatt der eigentlichen.
    
    \item Properties lassen sich mit \lstinline{.with_children(**kwargs)}
    setzen, wobei
    
    \lstinline{mock('mein_mock').with_children(x=1)} den Wert \lstinline{x}  für
    \lstinline{mein_mock} gleich \lstinline{1} setzt.
    
    \item Das gleiche ist mit \lstinline{.with_method(**kwargs)} möglich, nur für Methoden anstatt Properties.
\end{itemize}

\noindent
Diese Features wurden anhand des in Listing \ref{listing:base:my_mock_module}
definierten Codes, mit \lstinline{mocktest} getestet. Der Code dazu ist in
Listing \ref{listing:mocktest:example} zu finden. Bei der Implementierung der
Tests gab es keine Komplikationen mit dem Tool. Da \lstinline{mocktest} keine
\lstinline{setUp()} Methode benötigt um zu funktionieren kann hier ein paar
Zeilen Code gespart werden. Durch \lstinline{expect()} und \lstinline{when()}
ist es möglich jedes beliebige Objekt zu nehmen und zu ersetzen. Da die
Implementierung keine Schwierigkeiten barg wird hier auf eine weitere Analyse
verzichtet.
\newline

Um TDD zu betreiben bietet \lstinline{mocktest} alles was ein Entwickler zum
\gls{mock}en benötigt. So ist es möglich bestehende Objekte gänzlich oder
Teilweise zu ersetzen, oder neue Objekte zu erstellen die, die Signatur eines
anderen Objektes haben. Auch Globale Objekte lassen sich für den Zeitraum eines
Tests ersetzen, dadurch ist alles an Funktionalität geboten, was ein
\gls{mock}-testing Tool bieten muss.

Bezüglich der Effizienz kann \lstinline{mocktest} problemlos eingesetzt werden,
da quasi kein Vorwissen von Nöten ist um Objekte erfolgreich zu \gls{mock}en.
Die Vorarbeit die geleistet werden muss hält sich je nach Anwendungsfall in
grenzen oder ist so gering, dass sie kein Hindernis darstellt. Wird die von
\lstinline{mocktest} zur Verfügung gestellte Klasse
\lstinline{mocktest.TestCase} für einen Testfall genutzt, so ist bereits alles
fertig und direkt einsetzbar. Selbstverständlich kann
\lstinline{mocktest.TestCase} je nach den Bedürfnissen um Funktionalität
erweitert werden, jedoch sollte sich dies als unkompliziert erweisen.

\lstinline{mocktest} lässt den Entwickler Code schreiben der sich wie gesprochen
liest. Durch die Methoden \lstinline{when(obj)} und \lstinline{expect(obj)}
lässt sich ein \Gls{mock} erstellen der mit Methoden angepasst werden
kann, die wenn man sie aneinander reiht, sich wie ein Satz lesen. Dies ist zum
einen durch die verschiedenen Aliase möglich, wie zum Beispiel \lstinline{.once()}
welches \lstinline{.exact(1)} ausführt als auch mit \lstinline{.then_return()}.
Abseits der eigentlichen Funktionalität die mit \lstinline{mocktest} genutzt wird
fällt kein Code an der geschrieben werden muss um Tests ausführbar zu machen.
% !TeX root = ../bachelor.tex
\newpage
\paragraph{flexmock}\label{python-tools:flexmock}\mbox{}
\newline
\lstinline{flexmock} ist ein weiteres Tool, dass von der Ruby Community
inspiriert wurde. Dabei diente das gleichnamige Tool
\href{https://github.com/jimweirich/flexmock}{flexmock}\footnote{https://github.com/jimweirich/flexmock}
als Inspiration. "`Es ist jedoch nicht das Ziel von Python flexmock, ein
Klon der Ruby-Version zu sein. Stattdessen liegt der Schwerpunkt auf der
vollständigen Unterstützung beim Testen von Python-Programmen und der möglichst
unauffälligen Erstellung von gefälschten Objekten."'\footnote{Übersetzt aus dem Englischen}
(\cite{flexmock:docs:0.10.3}). Die Features von \lstinline{flexmock} sind denen
von \nameref{python-tools:mocktest} sehr ähnlich. So bildet \lstinline{flexmock}
seine Tests, \Glspl{stub} und \Glspl{mock} wie ausgesprochen dar.
\newline

\lstinline{flexmock} bietet dem Nutzer einiges an Funktionalität, auch wenn es
sich erst in der Version \lstinline{0.10.4} (Stand 19. Juli 2019) befindet. So
lässt sich bereits ein \Gls{stub} für Klassen, Module und Objekte mithilfe von
\lstinline{flexmock()} einrichten. Damit ist es möglich alles Notwendige für
ein erfolgreiches Durchlaufen der Tests zu erstellen.

Auch \Gls{mock}ing ist kein Problem mit \lstinline{flexmock}. Mit der gleichen
Funktion, mit der \Glspl{stub} erstellt werden, lassen sich
\Glspl{mock} erstellen. Dadurch ist es möglich, einen \Gls{stub} als einen
\Gls{mock} und umgekehrt zu verwenden. Daraus resultierend besteht für den
Entwickler die Möglichkeit, zu überprüfen wie oft etwas aufgerufen wurde, welche
Parameter verwendet wurden, welcher Rückgabe Wert zu erwarten ist und/oder
welche Exception zu geworfen werden soll.

Zusätzlich ist es möglich, originale Funktionen eines Objektes zu
\gls{mock}en. Dadurch enthält der Entwickler die Möglichkeit
zu überprüfen, ob eine unveränderte Funktion oder Methode bestimmten
Anforderungen entspricht. Das Interface dafür ist das Selbe, wie bei einem
\Gls{stub}.
\lstinline{flexmock} bietet die Möglichkeit einen leeren \Gls{mock} zu
erstellen und diesem eine beliebige Methoden hinzu zu fügen. Dies ist jedoch
nicht bei Objekten möglich, deren Methoden bereits fest stehen. Deswegen kann
eine neue Methode zu einem gemockten Objekt nicht hinzugefügt werden.
\newline

Ein besonders interessantes Feature ist die Überprüfung des Ergebnisses nach
Typ. Dabei wird \lstinline{.and_return()} nur der Typ mitgegeben, der erwartet
wird. So würde
\\% Code geht über Zeile hinaus
\lstinline{.and_return((int, str, None)} jedes \lstinline{Tuple}
zulassen, das als ersten Wert einen \lstinline{int} hat, als zweiten einen
String und als drittes None. Dabei kann selbstverständlich auf jede beliebige
Instanz überprüft werden, so auch auf eigene Klassen. Dies ist allerdings nur
bei \lstinline{.should_call()} und bei \lstinline{.should_receive()} nicht
möglich.

Das letzte in der Dokumentation erwähnte Feature ist die Ersetzung von Klassen
noch vor ihrer Instanziierung (\cite{flexmock:docs:0.10.3}). Dabei gibt es
mehrere Ansätze, die verschiedene Ausgänge haben. Der erste Ansatz ist
auf Modul Level. Dabei wird im Modul die Klasse mit einem Objekt, das
entweder ein \lstinline{flexmock} oder ein beliebig anderes sein kann, ersetzt.
Nachteil dieser Methode ist, dass durch das Ersetzen der Klasse durch eine
Funktion, Probleme herbeigeführt werden. Um dem entgegen zu wirken, kann
alternativ \lstinline{.new_instance(obj)} verwendet werden, welches beim
Erstellen das Objekt zurück gibt, welches \lstinline{.new_instnce()} übergeben
wurde. Ein weiterer Lösungsweg könnte sein, die \lstinline{__new__()} Methode
der Klasse mit \lstinline{.should_receive('__new__').and_return(obj)} zu
ersetzen, was im Endeffekt das gleiche ist, nur verständlich ausgeschrieben.
\newline

Der Code zu Listing \ref{listing:base:my_mock_module} wurde mit Hilfe des Codes
von \lstinline{flexmock} getestet. Dabei ergaben sich die eben aufgelisteten
Eigenschaften (vgl. Listing \ref{listing:flexmock:example}).
\lstinline{flexmock} überzeugt dabei mit seiner Einfachheit des Aufsetzens. In
der \lstinline{setUp()} wird global im Modul 
\lstinline{my_package.my_mock_module} die Klasse \lstinline{NotMocked} ersetzt.
Dadurch besteht die Möglich in jedem Testfall eine Annahme zu tätigen, die
auf alle Objekte der Klasse \lstinline{NotMocked} zutreffen. Der Aufwand diesen
Code zu schreiben war demnach sehr gering. Lediglich das Erstellen einer
Methode, die im originalen Objekt nicht existiert, war unmöglich. Aus diesem
Grund ist der Code in \lstinline{test_call_helper_help()} nicht korrekt und
wirft eine Exception. Diese kann in Listing \ref{listing:flexmock:exception}
eingesehen werden. Das Problem wurde mit Hilfe des Erstellers des Tools
versucht zu lösen, jedoch konnte bis zur Abgabe dieser Arbeit keine
funktionierende Lösung erarbeitet werden.
\newline

Die Anwendbarkeit von \lstinline{flexmock} ist ausgezeichnet, da es zunächst
keine weiteren Abhängigkeiten, außer sich selbst, installiert und mit allen
Test-runnern kompatibel ist. Im Bezug auf TDD ist \lstinline{flexmock} sehr zu
empfehlen. Dem Entwickler werden zahlreiche Funktionalität geboten,
\Glspl{stub} oder \Glspl{mock} zu erstellen und zu verwenden. Dabei lässt sich
von der Aufrufanzahl bis zu den verwendeten Parametern, alles überprüfen.
Lediglich das Setzen einer nicht implementieren Methode war mit
\lstinline{flexmock} nicht möglich.

Im Aspekt Effizienz ist \lstinline{flexmock} ein Parade-Beispiel für den 
benötigten Aufwand, Tests zu implementieren. Der Entwickler muss selbst wenig
von \gls{mock}ing, verstehen um die Funktionalität von \lstinline{flexmock}
nutzen zu können, da sich der Code wie eine Aussage liest. Das Gleiche gilt
natürlich auch für die \Glspl{stub}.

Genauso ist die Komplexität von \lstinline{flexmock} ausgezeichnet.
\lstinline{flexmock} bieten für den Entwickler zahlreiche sinnvolle Aspekte und
darüber hinaus  ein Interface, welches das Schreiben von übersichtlichen Codes
ermöglicht. Dies liegt vor allem an der deskriptiven Programmierung, die von
\lstinline{flexmock} geboten wird. Auch die Unübersichtlichkeit des Codes hält 
sich dabei in Grenzen.
% !TeX root = ../bachelor.tex
\paragraph{python-doublex}\label{python-tools:doublex}\mbox{}
\newline
Mit \lstinline{python-doublex} wird dem Entwickler ein Tool an die Hand gelegt,
dessen Funktionalität sich voll und ganz um doubles dreht. Ein double ist je nach
Anwendung ein \Gls{mock}, ein \Gls{stub} oder ein Spy(Spion), der eine Klasse oder
ein Objekt imitiert.

\lstinline{python-doublex} bietet drei Verschiedene Interfaces (vier zählt man die
Unterklasse von Spy dazu), Stub, Spy (und ProxySpy) und Mock. Da die Dokumentation
hier sehr schöne und treffende Beschreibungen nimmt, werden diese nun hier zitiert.
\begin{itemize}
    \item "`Stubs sagen dir was du hören willst."'\footlabel{doublex:doc:cite}{\cite{doublex:docs:1.8.1}} \footlabel{doublex:doc:trans}{Übersetzt aus dem Englischen}
    \item "`Spies erinnern sich an alles was Ihnen passiert."'\footref{doublex:doc:cite} \footref{doublex:doc:trans}
    \item "`Proxy spies leiten aufrufe an Ihre originale Instanz weiter."'\footref{doublex:doc:cite} \footref{doublex:doc:trans}
    \item "`Mock erzwingt das vordefinierte Skript."'\footref{doublex:doc:cite} \footref{doublex:doc:trans}
\end{itemize}

Mit diesen Beschreibungen der einzelnen Interfaces kann ein Entwickler bereits entscheiden
welches von Ihnen für Ihn in Frage kommt, ohne den Code dazu kennen zu müssen. Um dennoch
einen erweiterten Einblick zu schaffen werden die einzelnen Funktionen hier noch ein mal
genauer beschrieben. Wichtig ist hierbei zu beachten, dass \lstinline{python-doublex}
lediglich doubles, also Duplikate anbietet, die das gleiche Interface wie eine Klasse habe,
jedoch keine Instanz dieser Klasse sind. Dadurch bleiben manche Möglichkeiten dem
Entwickler verwehrt, wie zum Beispiel das Nutzen einer implementierten Methode aus
der original Klasse. Die Duplikate sollen als Ersatz gelten für nicht implementierte
Methoden, deren Funktionalität und deswegen der Rückgabe Wert bekannt ist.

Das Stub Interface ist, wie der Name bereits vermuten lässt, ein \Gls{stub}
mit dem es dem Entwickler möglich ist Rückgabe Wert von Funktionen zu setzen oder ähnliche
Aktionen fest zu definieren. Dazu ist es dem Entwickler möglich zwischen verschiedenen
Parametern zu unterscheiden oder gar auf alle aufrufe zu prüfen. Das Interface bietet auch
die Möglichkeit Parameter von einem Objekt zu ändern um den ausgang von Methoden zu verändern.
Das Interface unterscheiden zwischen einem Stub und einem "`free"' Stub. Ersterer nimmt eine
Klasse als Parameter und ersetzt diese, zweiterer nimmt keine Parameter und fungiert als
generelles Objekt. Dabei ist der wichtigste unterschied, dass der normale Stub nur 
Methoden ab ändern kann die, die originale Klasse auch besitzt, der "`free"' Stub 
ist in dieser Hinsicht ungebunden und kann alles sein was der Entwickler Ihm zuweist,
also ist er frei.

Als zweites Interface wird von \cite{doublex:docs:1.8.1} in der Dokumentation das Spy Interface
erwähnt. Dieses bietet dem Nutzer die Möglichkeit Klassen zu überwachen, dabei legt der Entwickler
fest welche Methode wie oft und mit welchen Parametern aufgerufen werden muss. Werden die 
Erwartungen nicht getroffen wird eine Exception geworfen. Das überprüfen der Parameter ist dabei
möglich mithilfe des exakten wertes, einer Regex oder mit dem von \lstinline{python-doublex}
definierten Schlüsselwort \lstinline{ANY_ARG}, welches, wie der Name sagt, jedes Argument
validiert. Zum Spy gibt es auch wieder einen "`free"' Spy, welcher von keiner Klasse
abhängig ist. Dadurch ist es möglich jede beliebige Methode auf diesem aus zu führen ohne
dass dies zu einer Exception führen würde, danach kann dann überprüft werden mit welchen
Parametern und wie oft der Spy aufgerufen wurde.

Zusätzlich zum Spy gibt es noch einen \lstinline{ProxySpy}, welcher das zu überwachende
Objekt aufruft und nicht ersetzt. Aus diesem Grund erhält der \lstinline{ProxySpy} ein
Objekt als Parameter und kein Klasseninterface. Sämtliche Methoden werden auf das originale
Objekt ausgeführt wodurch keine Veränderung an diesem Vorgenommen werden kann, in Form von
einem \Gls{stub}.

Das letzte Interface, ist das Mock Interface. Mit diesem lässt sich vor dem Test festlegen
welche Methode wann und wie aufgerufen werden muss, damit der Test erfolgreich wird. Die
Reihenfolge spielt dabei eine Rolle, außer man nutzt \lstinline{any_order_verify()}.
Das Mock Objekt lässt sich dabei wie jedes andere Objekt verändern und wie ein Stub
präparieren. Auch der "`free"' Mock ist wie bei den anderen Interfaces verfügbar und
bietet die gleichen Möglichkeiten der freien Gestaltung des Objekts.

Zusammenfassen lässt sich sagen, das jedes Interface die Möglichkeit bietet eine Klasse
zu verändern ausgenommen, der \lstinline{ProxySpy} der nur auf einer Instanz arbeiten kann.
Zu jedem Interface ist ein freies Interface verfügbar, welches mit beliebigen Methoden
aufgerufen werden kann ohne Fehler zu werfen. Zum Abschluss lässt sich auch noch sagen, dass
jedes Interface, die Möglichkeit bietet stubbing zu betreiben, wobei nur das Stub Interface
sonst keine weitere Funktionalität bietet.


\lstinline{python-doublex} bietet dem Entwickler viel, aber nicht alles für effizientes TDD.
Wie der Name des Tools bereits vermuten lässt handelt es sich um Duplikate von Objekten, deren
Funktionalität aber vom Entwickler festgelegt werden muss. So würde eine Klasse mit einer Methode
\lstinline{return_input(input)} die den übergebenen Parameter zurück gibt standardmäßig
\lstinline{None} zurück geben, solange nichts anderes im Duplikat festgelegt wurde. Lediglich
die Signartur des Aufrufs wird überprüft, so würde \lstinline{stub.reuturn_input()} eine Exception
werfen. Zwar lässt sich das standardverhalten von \lstinline{python-doublex} verändern, jedoch
kann man hier nur einen festen Wert setzen der für alle nicht ersetzten Methoden gilt. Auch
wenn man Globale Objekte oder Module ersetzen möchte ist dies nicht möglich. Hinzu kommt das,
dass Tool drei Abhängigkeiten benötigt und selbst nicht in stabilem Zustand ist. Währen der
Analyse sind zwei \lstinline{DeprecationWarning}s aufgetaucht, wovon eine bereits mit Python 3.0
ausgelaufen ist. Dies weißt auf eine nicht sehr aktive Entwicklung des Tools hin. Selbstverständlich
wurde die Fehler auf GitHub gemeldet, sodass man sich darum kümmern kann. (Stand 23. April 2019)

Auch die Effizienz lässt etwas zu wünschen übrig. Dadurch, dass das Duplikat nicht die originalen
Methoden aufruft, sofern verfügbar, muss jede Methode mit einem Return Wert überschrieben werden.
Dies mag für manche Tests vollkommen ausreichen, macht aber die Entwicklung mit TDD sehr schwer,
da Tests von Zeit zu Zeit ausgeführt werden müssen und dort die Teil Implementierung selbst
verständlich genutzt werden soll. Hat man allerdings eine Externe Klasse auf die kein Einfluss ist
kann dieses Tool durchaus nützlich werden.

An Komplexität fehlt es \lstinline{python-doublex} etwas, so kann der \Gls{stub} lediglich festlegen
welche Funktion ausgeführt, welcher Rückgabewert zurück gegeben oder welche Exception geworfen wird.
Das Spy Interface hingegen ist leicht zu nutzen und bietet fast alles was ein Entwickler brauchen
kann um zu überprüfen ob und wie etwas aufgerufen wurde. Jedoch scheitert es hier auch an der
Implementieren von echten Objekten, so fungiert der ProxySpy zwar als Spy auf einem Objekt, jedoch
kann er nur verzeichnen welche Methode auf Ihm aufgerufen wurde. So würde \lstinline{spy.call_other()}
nicht \lstinline{obj.other()} registrieren, wodurch es nicht möglich ist zu überprüfen ob
\lstinline{other()} nun aufgerufen wurde oder nicht. Das gleiche gilt mit Mock, lediglich Methoden
die auf dem Mock-Objekt aufgerufen werden werden registriert. Aus diesem Grund ist es auch hier nicht
optimal nutzbar für TDD.

Auch hier wurde der Code aus Listing \ref{listing:base:my_mock_module} mit \lstinline{python-doublex}
getestet. Der Code dazu befindet sich in Listing \ref{listing:doublex:example}. Wie dort zu erkennen
ist, konnte nicht jeder Test ohne Probleme validiert werden. So war es zwar möglich die ersten drei
Tests erfolgreich zu validieren, jedoch fragt man sich dort ob es wirklich etwas bringt ein Objekt
zu testen welches einfach das zurück gibt was man erwartet. Beim Test, der eine interne Methode
aufrufen soll kommt das Spy Interface an seine grenzen, da es nicht überprüfen kann ob die
interne Methode aufgerufen wurde. Der Error wurde in einem Kommentar in der Teste Methode
festgehalten. Das Testen ob \lstinline{Helper.help()} aufgerufen wurde, war teilweise erfolgreich,
da es dank des Duplikates ein Objekt mit dem Richtigen Interface bekommen hat, dennoch ist es leider
mit \lstinline{python-doublex} nicht möglich vor zu täuschen eine andere Klasse zu sein. Auch hier
wurde der Fehler in einem Kommentar festgehalten. Zuletzt sollte getestet werden ob das \lstinline{os}
Modul ausgetauscht werden kann, jedoch ist dies mit \lstinline{python-doublex} leider nicht möglich.
Der Fehler dazu wurde in einem Kommentar innerhalb der Funktion festgehalten, sowie eine mögliche,
wenn auch umständliche Lösung. Die würde das \lstinline{os} Objekt duplizieren und der Methode als
Parameter übergeben, was im Praktischen Einsatz aber nicht gemacht werden sollte.
% !TeX root = ../bachelor.tex
\paragraph{python-aspectlib}\label{python-tools:aspectlib}\mbox{}
\newline
Blubb
% !TeX root = ../bachelor.tex
\subsubsection{Fuzz-testing Tools}\label{python-tools:extlib:fuzz}

Zusätzlich zu den hier behandelten Test runnern und den \gls{mock}ing Tools 
werden Fuzz-testing Tools behandelt. Unter Fuzz-testing 
versteht man das Testen eines Programms oder einem Modul mit zufälligen Werten. 
Diese Werte können komplett zufällig sein, aber auch einem gewissen Format 
oder einem Typ entsprechen.

Fuzz-testing funktioniert dabei anders als die gewohnten Methoden des Testens. 
Normalerweise werden im Test Daten präpariert, die im Test ausgeführt und 
danach validiert werden, sodass das Ergebnis richtig ist. Mit Fuzz-testing 
sieht das ganze etwas anders aus: zuerst wird festgelegt, welchem Schema die 
Daten entsprechen müssen (Beispiel: Es wird eine IP erwartet, 
\lstinline|[1-9]{1,3}.[1-9]{1,3}.[1-9]{1,3}.[1-9]{1,3}|), danach wird mit den 
Daten der Test ausgeführt und schlussendlich validiert 
(\cite{hypothesis:doc:4.18.0}).

Der Unterschied ist, dass der Entwickler sich selbst keine Daten ausdenken 
muss. Im genannten Beispiel müsste der Entwickler sich verschiedene IP Adressen 
ausdenken, die entweder richtig oder falsch sind und danach diese überprüfen. 
Fuzz-testing übernimmt das Ausdenken der Daten und sorgt dafür, dass diese der 
Spezifikation entsprechen. So kann eine viel größere Anzahl an Tests mit 
erheblich weniger Aufwand auf einem Modul oder Programm ausgeführt werden.

Schlägt ein Test mit Daten, die per  Spezifikation richtig sind, fehl, so wird 
der Wert der Daten für später gespeichert und, sofern möglich, versucht mit 
ähnlichen Daten dieser Fehler zu erreichen. Dadurch hat ein Entwickler am Ende 
der Tests Daten, die fehlgeschlagen sind. Mit diesen kann er die weiteren Tests 
befüllen um so  zu verhindern, dass diese Daten zu Fehlern führen. Im 
Normalfall übernimmt dies allerdings das Tool, wodurch der Entwickler sich nur 
darum kümmern muss, dass die Tests erfolgreich verlaufen.

Von Zeit zu Zeit entsteht dadurch ein Katalog an Daten für verschiedene Tests, 
der vorgibt bei welchen Daten ein Test fehl geschlagen ist. Dadurch wird 
verhindert, dass bei der Änderung des Codes dieser Fehler wieder auftritt. 
Führt man  dies über den gesamten Entwicklungsprozess hinweg durch, so erhält 
man am Ende ein sehr ausgiebig getestetes Programm.

Diese Methode des Testens wird auch "`property based testing"', also 
Eigenschaftsbasierte Tests genannt. In dieser Arbeit, wird allerdings weiterhin 
der englische Begriff Fuzztesting genutzt.

% !TeX root = ../bachelor.tex
\paragraph{hypothesis}\label{python-tools:hypothesis}\mbox{}
\newline
Das größte und am weitesten verbreitete Tool für Fuzz-testing mit Python ist
\lstinline{hypothesis}. Recherchen im Internet weisen auf kein weiteres aktuelles
Tool hin, dass Open Source ist, demnach ist \lstinline{hypothesis} das einzige
Tool welches zum Fuzz-testing mit Python derzeit genutzt werden kann (Python 3.7).

Bevor man anfangen kann mit \lstinline{hypothesis} zu testen, muss man verstehen
wie \lstinline{hypothesis} funktioniert. Möchte man Daten haben die einer
Spezifikation entsprechen, benötigt man \lstinline{hypothesis.strategies}. In
\lstinline{hypothesis} sind Strategien Spezifikationen für die Generierung von
Daten. Wäre eine Spezifikation also, dass die Daten \lstinline{Integer} sind, dann
würde man \lstinline{hypothesis.strategies.integers()} verwenden. Diese Strategie
würde verschiedene Daten die \lstinline{Integer} sind ausgeben. Möchte man nur
Werte zwischen A und B, so wäre die Strategie
\lstinline{.integers(min_value=A, max_value=B)}. Dies kann man mit beliebig
vielen Daten Typen und Spezifikationen durchführen, dabei verfügt
\lstinline{hypothesis} über so viele verschiedene Spezifikationen, dass es schwer
sein wird alle zu kennen. Die Basics lassen sich allerdings in der
\href{https://hypothesis.readthedocs.io/en/latest/data.html}{Dokumentation finden}\footnote{https://hypothesis.readthedocs.io/en/latest/data.html}.

Ist die Spezifikation bekannt und die Strategie gefunden werden Tests mithilfe einer
\Gls{annotation} der Test Funktion übergeben, diese \Gls{annotation} nennt sich
\lstinline{hypothesis.given()}. Ein kleines Beispiel dazu kann in Listing
\ref{listing:hypothesis:given} gefunden werden. \lstinline{.given()} unterstützt
dabei die \lstinline{*args} als auch die \lstinline{**kwargs} Übergabe der Strategien.

Hat man nun einen Fehler gefunden und möchte diesen für alle weiteren Tests testen,
so kann man mit \lstinline{hypothesis.example()} den Test annotieren und die Werte
bei denen der Test fehl geschlagen ist dort übergeben. Möchte der Entwickler etwas
mehr Information darüber weshalb ein Test fehl geschlagen ist, so ist es Ihm möglich
mit \lstinline{hypothesis.note()} vor einer Assertion einen Text aus zu geben, der
Ihm diese Informationen gibt. \lstinline{note()} wird allerdings nur aus gegeben,
wenn der Test fehl schlägt.

Möchte man ein paar Statistiken darüber was \lstinline{hypothesis} gemacht hat und man
nutzt \lstinline{pytest} als Test runner, so kann man mit \lstinline{--hypothesis-show-statistics}
sich die Statistiken zu den Tests anschauen. Dabei werden folgende Informationen 
zu jeder Test Funktion gezeigt:
\begin{itemize}
    \item Gesamte Anzahl der Durchläufe mit der Anzahl der fehlgeschlagenen und der Anzahl der invaliden Tests.
    \item Die ungefähre Laufzeit des Tests.
    \item Wie viel Prozent davon für die Datengenerierung aufgewendet wurde.
    \item Wieso der Test beendet wurde.
    \item Alle aufgetretenen Events von \lstinline{hypothesis}
\end{itemize}

Ein Event kann allerdings auch vom Entwickler genau so wie \lstinline{note()} genutzt
werden, mit dem Unterschied, dass ein Event immer zu einem Output führt wenn die
Zeile im Code ausgeführt wird (Sofern \lstinline{pytest} und
\lstinline{--hypothesis-show-statistics} genutzt werden).

Mit diesem Wissen lässt sich \lstinline{hypothesis} bereits für die einfachsten Dinge
Nutzen, hat man allerdings etwas Spezifischere Anforderungen, so muss etwas tiefer
in die Materie eingestiegen werden. Angenommen, eine Funktion soll mit einem
\lstinline{Integer} getestet werden, dieser muss aber durch sich selbst teilbar sein.
Dies könnte mit einem \lstinline{if} Statement geprüft werden, würde aber die Anzahl
der relevanten Tests erheblich senken. Standessen ist es möglich auf Strategien filter
an zu wenden. Filter müssen dabei Funktionen sein, die einen Wert filtern und Ihn zurück
geben. Für das eben genannte Beispiel würde die \Gls{annotation} dann so aussehen:
\lstinline{@given(integers().filter(lambda i: i % i == 0))}.

Gibt \lstinline{hypothesis} einem Test Daten die nicht gefiltert werden können oder sollen,
so ist es möglich mit \lstinline{hypothesis.assume()} eine Annahme auf zu stellen. Ist die
Annahme falsch, so wird dieser Test übersprungen. Allerdings kann dies zu Fällen führen bei
denen \lstinline{hypothesis} keine validen Daten finden kann, was zu einer Exception und
zum fehlschlagen des Tests führt.

\lstinline{hypothesis} verfügt noch über weit mehr Funktionalität als hier genannt werden kann,
wie zum Beispiel das verketten von Strategien. Die Dokumentation zu allem, was
\lstinline{hypothesis} ohne Erweiterungen erschaffen kann, kann
\href{https://hypothesis.readthedocs.io/en/latest/data.html}{online}\footcite{https://hypothesis.readthedocs.io/en/latest/data.html}
eingesehen werden.

\textbf{Erweiterungen erklären}
\textbf{Code Beispiel von simplen sachen}

% !TeX root = ../bachelor.tex
\section{Zusammenfassung}\label{zusammenfassung}

Hier werden in kurzen Punkten die Vor- und Nachteile der einzelnen Tools zusammengefasst,
um dem Leser einen schnellen Überblick zur Verfügung zu stellen.
% !TeX root = ../bachelor.tex
\section{Vergleich der Tools}\label{vergleich}

Blubb

% !TeX root = ../bachelor.tex
\section{Kombinierung von Tools}\label{kombinierung}

Blubb


