% !TeX root = ../bachelor.tex
\paragraph{doublex}\label{python-tools:doublex}\mbox{}
\newline
Mit \lstinline{doublex} wird dem Entwickler ein Tool an die Hand gelegt,
dessen Funktionalität sich voll und ganz um doubles dreht. Ein double ist je
nach Anwendung ein \Gls{mock}, ein \Gls{stub} oder ein Spy(Spion), der eine
Klasse oder ein Objekt imitiert (\cite{doublex:docs:1.8.1}).
\newline

\lstinline{doublex} bietet drei verschiedene Interfaces (vier zählt man die
Unterklasse von Spy dazu), \Gls{stub}, Spy (ProxySpy) und Mock. Da die
Dokumentation hier sehr schöne und treffende Beschreibungen nimmt, werden diese 
nun zitiert. "`\Glspl{stub} sagen dir, was du hören 
willst."'\footlabel{doublex:doc:trans}{Übersetzt aus dem Englischen} 
\footlabel{doublex:doc:cite}{\cite{doublex:docs:1.8.1}}. "`Spies erinnern sich 
an alles was Ihnen passiert."'\footref{doublex:doc:trans} 
\footref{doublex:doc:cite}. "`Proxy spies leiten Aufrufe an ihre originale 
Instanz weiter."'\footref{doublex:doc:trans} \footref{doublex:doc:cite}. "`Mock 
erzwingt das vordefinierte Skript."'\footref{doublex:doc:trans} 
\footref{doublex:doc:cite}.

Mit diesen Beschreibungen der einzelnen Interfaces kann ein Entwickler bereits
entscheiden, welches der Objekte relevant für die zu erledigende Arbeit ist, 
ohne
den Code kennen zu müssen. Um dennoch einen erweiterten Einblick zu
schaffen, werden die einzelnen Funktionen nun genauer beschrieben.

Wichtig hierbei ist zu beachten, dass \lstinline{doublex} lediglich
doubles, also Duplikate anbietet, die das gleiche Interface wie eine Klasse
haben, jedoch keine Instanz dieser Klasse sind. Dadurch bleiben manche
Möglichkeiten, wie zum Beispiel das Nutzen einer
implementierten Methode aus der original Klasse, dem Entwickler verwehrt.
Die Duplikate sollen als Ersatz für Objekte gelten, die in einer bestimmten
Umgebung ausgeführt werden müssen oder Änderungen vornehmen, die während
eines Tests unerwünscht sind.
\newline

Das \Gls{stub} Interface ist ein \Gls{stub} mit dem es dem Entwickler möglich 
ist, Rückgabewerte von Funktionen zu setzen oder ähnliche Aktionen zu 
definieren. Dabei ist es dem Entwickler möglich, zwischen verschiedenen 
Parametern zu unterscheiden oder gar auf alle Aufrufe zu prüfen.

Das Interface bietet die Möglichkeit Parameter von einem Objekt zu modifizieren
um den Ausgang von Methoden zu verändern. Hierbei wird zwischen einem \Gls{stub}
und einem "`free"' \Gls{stub} unterschieden. Ersterer nimmt eine Klasse als
Parameter und ersetzt diese, zweiterer nimmt keine Parameter und fungiert als
generelles Objekt. Dabei ist der wichtigste Unterschied, dass der normale
\Gls{stub} nur Methoden abändern kann, welche die originale Klasse auch 
besitzt. Der "`free"' \Gls{stub} hingegen ist in dieser Hinsicht ungebunden und 
kann alles sein, was der Entwickler programmiert.
\newline

Als zweites Interface wird in der Dokumentation das Spy Interface erwähnt
(\cite{doublex:docs:1.8.1}). Dieses bietet dem Nutzer die Möglichkeit Klassen zu
überwachen. Dabei legt der Entwickler fest, welche Methode wie oft und mit
welchen Parametern aufgerufen werden muss. Werden die Erwartungen nicht
getroffen, wird eine Exception geworfen. Mit Hilfe des Überprüfen der Parameter 
und mit Hilfe des exakten Wertes ist es möglich, einer Regex oder mit dem von
\lstinline{doublex} definierten Schlüsselwort \lstinline{ANY_ARG} zu erstellen, 
welches, wie der Name sagt, jedes Argument validiert.

Zusätzlich gibt es einen "`free"' Spy, welcher von keiner
Klasse abhängig ist. Hierdurch ist es möglich, jede beliebige Methode auf
diesem aus zu führen, ohne dass dies zu einer Exception führen würde.
\newline

Zusätzlich zum Spy und "`free"' Spy, gibt es noch einen \lstinline{ProxySpy},
welcher das zu überwachende Objekt aufruft und nicht ersetzt. Aus diesem Grund
erhält der \lstinline{ProxySpy} ein Objekt als Parameter und kein
Klasseninterface. Sämtliche Methoden werden auf das originale Objekt ausgeführt,
wodurch keine Veränderung an diesem vorgenommen werden kann.
\newline

Das letzte Interface, ist das Mock Interface. Mit diesem lässt sich vor dem Test
festlegen, welche Methode wann und wie aufgerufen werden muss, damit der Test
erfolgreich ist. Die Reihenfolge spielt dabei eine wichtige Rolle, außer es wird
\lstinline{any_order_verify()} genutzt. Das Mockobjekt lässt sich, wie
jedes andere Objekt, verändern und wie ein \Gls{stub} präparieren.

Auch der "`free"' Mock ist, wie bei den anderen Interfaces, verfügbar und bietet
die gleichen Möglichkeiten der freien Gestaltung des Objekts.
\newline

Zusammenfassend lässt sich sagen, dass jedes Interface die Möglichkeit bietet, 
eine
Klasse zu verändern, ausgenommen der \lstinline{ProxySpy}, der nur auf einer
Instanz arbeiten kann.

Zu jedem Interface ist ein freies verfügbar, welches mit beliebigen
Methoden aufgerufen werden kann ohne Fehler zu werfen. Abschließend lässt sich
sagen, dass jedes Interface, die Möglichkeit bietet,
\gls{stub}ing zu betreiben, wobei nur das \Gls{stub}interface sonst keine
weitere Funktionalität bietet.
\newline

Die Features von \lstinline{doublex} wurden auf den in Listing
\ref{listing:base:my_mock_module} definierten Code angewandt (vgl. Listing
\ref{listing:doublex:example}). Wie dort zu erkennen ist, konnte nicht jeder
Test ohne Probleme validiert werden. So war es zwar möglich die ersten drei
Tests erfolgreich zu validieren, jedoch ist es fraglich, ob dieser Code
wirklich etwas bringt, da die getesteten Objekte einfach nur zurück geben, was
erwartet wird und dabei nicht einmal das originale Objekt sind.

Bei dem Test, der eine interne Methode aufrufen soll, kommt das Spy Interface an
seine Grenzen, da es nicht überprüfen kann, ob die interne Methode aufgerufen
wurde. Stadtessen wird eine Exception geworfen, welche in einem
Kommentar in der Testmethode festgehalten wurde.

Das Testen, ob \lstinline{Helper.help()} aufgerufen wurde, war teilweise
erfolgreich, da es Dank des Duplikates ein Objekt mit dem richtigen Interface
bekommen hat. Es konnte jedoch nicht vortäuschen, die gewünschte Klasse zu sein.
Auch hier wurde der Fehler in einem Kommentar festgehalten.

Zuletzt sollte getestet werden, ob das \lstinline{os} Modul ausgetauscht werden
kann. Dies ist mit \lstinline{doublex} jedoch nicht möglich. Der Fehler
dazu wurde in einem Kommentar innerhalb der Funktion festgehalten, sowie eine
mögliche, wenn auch umständliche Lösung.
Diese würde das \lstinline{os} Objekt duplizieren und der Methode als
Parameter übergeben, was in der Praxis nicht gemacht werden
sollte, da dazu das Interface der Methode geändert werden müsste.
\newline

\lstinline{doublex} bietet dem Entwickler viel, aber nicht alles Notwendige für 
effizientes
TDD. Wie der Name des Tools verrät, handelt es sich um Duplikate von Objekten,
deren Funktionalität aber vom Entwickler festgelegt werden muss. So würde eine
Klasse mit einer Methode \lstinline{return_input(input)} die den übergebenen
Parameter zurück gibt, standardmäßig \lstinline{None} zurück geben, solange
nichts anderes im Duplikat festgelegt wurde. Lediglich die Signatur des Aufrufs
wird überprüft. Deswegen würde \lstinline{stub.reuturn_input()} eine Exception
werfen.

Zwar lässt sich das Standardverhalten von \lstinline{doublex} verändern, jedoch
kann nur ein fester Wert gesetzt werden, der für alle nicht ersetzten Methoden
gilt. Globale Objekte oder Module lassen sich ebenso nicht ersetzen. Hinzu 
kommt, dass das Tool drei Abhängigkeiten benötigt und selbst noch nicht in 
stabilem Zustand ist. Während der Analyse sind zwei 
\lstinline{DeprecationWarning}s aufgetaucht, wovon eine bereits mit Python 3.0 
ausgelaufen ist. Diese weisen auf eine nicht sehr aktive Entwicklung des Tools 
hin.

Selbstverständlich wurden die Fehler auf GitHub gemeldet, sodass der Entwickler
sich in Zukunft darum kümmern kann (Stand 23. April 2019). Jedoch wurden diese 
bis zum Abschluss dieser Arbeit nicht gefixed (Stand 19. Juli 2019).

Auch die Effizienz ist nicht optimal. Dadurch, dass das Duplikat nicht die
originalen Methoden aufruft, muss jede Methode mit einem Returnwert
überschrieben werden. Dies mag für manche Tests vollkommen ausreichen, macht
aber die Entwicklung mit TDD sehr schwer, da Tests von Zeit zu Zeit ausgeführt
werden müssen und dort die original Implementierung ab und zu genutzt
werden soll. Hat man allerdings eine externe unbeeinflussbare Klasse
kann dieses Tool durchaus nützlich werden.

An Komplexität fehlt es \lstinline{doublex} ein wenig. So kann der \Gls{stub}
lediglich festlegen, welche Funktion ausgeführt, welcher Rückgabewert zurück
gegeben oder welche Exception geworfen wird.

Das Spy Interface hingegen ist leicht zu nutzen und bietet fast alles, was ein
Entwickler brauchen kann, um zu überprüfen, ob und wie etwas aufgerufen wurde.
Jedoch scheitert es hier auch an der Implementierung von echten Objekten. So
fungiert der ProxySpy zwar als Spy auf einem Objekt, jedoch kann er nur
verzeichnen, welche Methode auf ihm aufgerufen wurde.

\lstinline{spy.call_other()} würde nicht \lstinline{obj.other()} registrieren,
wodurch es unmöglich ist zu überprüfen, ob \lstinline{other()} nun aufgerufen
wurde oder nicht. Das gleiche gilt für \Gls{mock}. Lediglich Methoden, die auf
dem Mock-Objekt aufgerufen werden, werden registriert. Aus diesem Grund ist es
auch hier nicht optimal nutzbar für TDD.