% !TeX root = ../bachelor.tex
\paragraph{python-doublex}\label{python-tools:doublex}\mbox{}
\newline
Mit \lstinline{python-doublex} wird dem Entwickler ein Tool an die Hand gelegt,
dessen Funktionalität sich voll und ganz um doubles dreht. Ein double ist je nach
Anwendung ein \Gls{mock}, ein \Gls{stub} oder ein Spy(Spion), der eine Klasse oder
ein Objekt imitiert.

\lstinline{python-doublex} bietet drei Verschiedene Interfaces (vier zählt man die
Unterklasse von Spy dazu), Stub, Spy (und ProxySpy) und Mock. Da die Dokumentation
hier sehr schöne und treffende Beschreibungen nimmt, werden diese nun hier zitiert.
\begin{itemize}
    \item "`Stubs sagen dir was du hören willst."'\footlabel{doublex:doc:cite}{\cite{doublex:docs:1.8.1}} \footlabel{doublex:doc:trans}{Übersetzt aus dem Englischen}
    \item "`Spies erinnern sich an alles was Ihnen passiert."'\footref{doublex:doc:cite} \footref{doublex:doc:trans}
    \item "`Proxy spies leiten aufrufe an Ihre originale Instanz weiter."'\footref{doublex:doc:cite} \footref{doublex:doc:trans}
    \item "`Mock erzwingt das vordefinierte Skript."'\footref{doublex:doc:cite} \footref{doublex:doc:trans}
\end{itemize}