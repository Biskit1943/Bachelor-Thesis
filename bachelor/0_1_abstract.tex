% !TeX root = ../bachelor.tex
\begin{center}
    \vspace{0.9cm}
    \textbf{Abstract}
\end{center}

\begin{abstract}
Da die Anforderungen an Software im Laufe der Zeit stark angestiegen sind,
haben sich einige Technologien entwickelt, die dazu verwendet werden diese
Anforderungen zu gewährleisten. Eine dieser Technologien ist das Test-driven
development. Diese Arbeit befasst sich mit dieser Technologie im Bezug zu den
aktuellen Test-Tools der Programmiersprache Python und analysiert diese auf
ihre Tauglichkeit im Bezug zum Test-driven development.
\newline

Dabei werden zuerst verschiedenen Arten von Tools untersucht, die unit-testing-,
mock-testing- und fuzz-testing Tools. Diese werden anhand ihrer
Anwendbarkeit, Effizient, Komplexität und Erweiterbarkeit beschrieben und
analysiert, um mit Hilfe eines Vergleichs derer einen Leitfaden für das am
besten für Test-driven development geeignete Tool zu bieten. Das daraus
resultierende Ergebnis soll aufzeigen, welche Toolkombinationen die beste Wahl
anhand der vorgegebenen Kriterien darstellen. Diese sind Anwendbarkeit,
Effizienz, Komplexität und Erweiterbarkeit.
\newline

Die dabei entstandenen Ergebnisse zeigen auf, dass nicht alle Tools optimal
für Test-driven development geeignet sind. Schlussendlich ergaben sich vier
Tools, welche gemeinsam eingesetzt, das optimale Toolset für die allgemeine
Anwendung bieten. Dabei ist jedoch zu beachten, dass dies immer vom
Anwendungsfall, sowie den Anforderungen an die Software und an die Tools
abhängig ist. Dabei ergab sich, dass \lstinline{pytest} zusammen mit
\lstinline{mocktest} sowie \lstinline{doctest} und \lstinline{hypothesis}
gemeinsam eine sehr gute Wahl darstellen.
\end{abstract}
