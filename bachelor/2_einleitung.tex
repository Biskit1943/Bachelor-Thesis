% !TeX root = ../bachelor.tex
\section{Einleitung}\label{einleitung}

TDD wird in der heutigen Softwareentwicklung immer verbreiteter und beliebter.
Die Ansprüche an Software sind in den letzten Jahren immer weiter gestiegen.
Dies liegt vor allem an der Reichweite, die Software heute hat. So besitzen
im Jahr 2018 bereits circa 66\% aller Menschen ein Smartphone
(\cite{FraukeSchobelt:Smartphone}). Im Arbeitsleben ist ein PC meist gar nicht
mehr weg zu denken. Doch mit den steigenden Nutzerzahlen steigen auch die
Anforderungen, welche die Nutzer an die Software stellen. Somit wird die Anzahl
der gefundenen \Glspl{bug} dementsprechend größer.

Im weltweiten Markt gibt es viele große Unternehmen, die gegenseitig um die
Nutzer kämpfen. Selbstverständlich präferieren die Nutzer denjenigen Anbieter,
welcher die bessere Software bietet. Dies kann sich heute jedoch stetig ändern.
Mit der steigenden Anzahl an Bugs, die gefunden werden, steigt auch die Anzahl
der Nutzer, die von diesen Bugs betroffen sind. Diese Bugs sollen natürlich
schnellstmöglich gefixt werden um so zu verhindern, dass die Nutzer die Software
wechseln.

So schwer es ist seine Nutzer zu halten, umso schwerer ist es, zum Start einer
Software Nutzer zu akquirieren. Es gibt bereits Software, die ähnliche Services
anbieten. So ist es noch schwerer dem Markt beizutreten. Die Anforderungen
werden durch die bereits am Mark tätigen Firmen gesetzt, damit sollten Fehler,
die bereits gelöst wurden nicht mehr auftauchen.

Für Unternehmen sind diese Anforderungen meist schwer zu meistern, weshalb
Software meist mit Fehlern released wird, um diese dann von den Nutzern
aufdecken zu lassen und zu fixen.

Sowohl als Entwickler als auch als Arbeitgeber muss man sich bei der Wahl der
Programmiersprache Gedanken darüber ob und wie einfach eine Sprache für
Test-driven development ein zu setzen ist. Der wichtigste Aspekt bei diesem
Prozess ist die Auswahl und die Qualität der von der Sprache zur Verfügung
gestellten Tools.

% !TeX root = ../bachelor.tex
\subsection{Was ist Python}\label{einleitung:python}

Blubb

% !TeX root = ../bachelor.tex
\subsection{Test-driven development}\label{einleitung:tdd}

Test-driven development ist eine Ausrichtung der Entwicklung nach einem definierten Schema.
Bei TDD dreht sich in der Entwicklung alles um die Tests die ausgeführt werden sollen.

Der Basis Ansatz dabei ist, dass bevor der Tatsächliche Code geschrieben wird, die Anforderungen
an den Code definiert sind. Je genauer die Definition ist, desto besser lässt sich TDD betreiben.
Aus der Definition heraus lassen sich die einzelnen Units (Funktionen und Methoden) und deren
angestrebte Funktionalität abstrahieren. Ist dies geschehen werden Tests geschrieben, die die
Funktionalität der nicht vorhanden Units prüft. Dabei sollte jede Spezifikation getestet werden
die vorher fest gelegt wurde. Je nach Auslegung werden erst alle Tests und dann der Code geschrieben
oder jeweils der Test und dann der Code, dabei ist es nicht von Nöten, dass der Entwickler der den
Test geschrieben hat auch den Code schreibt.

Die Implementierung wird dabei stets so vorgenommen, dass ein Test nach dem Anderem erfolgreich ist,
wobei Tests, die bereits Erfolgreich waren, immer Erfolgreich bleiben müssen. Bei diesem Prozess ist
es normal, dass zunächst alle Tests fehl schlagen und von Zeit zu Zeit weniger werden. Der Ablauf ist
demnach immer, Code schreiben, Tests ausführen, wenn Erfolgreich Code aufräumen und nächste Unit
schrieben und wenn nicht Erfolgreich den Code verbessern. Dieser Prozess wird immer und immer
wiederholt bis alle Tests erfolgreich durchgelaufen sind.

Sollten während der Test Phase Units auf anderen Units basieren können diese mithilfe von
\gls{mock}ing ersetzt werden um so die Funktionalität unabhängig testen zu können. Wird dies
nicht getan, so ist nicht mehr von Unit Tests die rede, sondern von Integration Tests.

Sollte dem Programm nun während der Entwicklung weitere Units hinzugefügt werden, so ist hier auch
stets zuerst der Test zu schreiben und dann der Code. Das gleiche gilt auch wenn die Funktionalität
einer Unit erweitert werden soll.

Neben dieser allgemeinen Herangehensweise gibt es auch Definitionen, wie die von Kent Beck in
"`Test Driven Development: By Example"' geschilderte. Demnach sieht der Entwicklungsprozess
folgendermaßen aus (\cite{beck:tdd}):
\begin{enumerate}
    \item Schreiben eines Tests, der neue Funktionalität oder Verbesserung zu einer bestehenden Unit
    hinzufügt. Dieser Test sollte möglichst kurz und aussagekräftig sein. Dazu muss der Entwickler
    die Spezifikationen und Anforderungen des Features genau verstehen um den Test wirklich effektiv
    schreiben zu können.
    \item Alle Tests durchführen und sehen, ob der neue Test fehlschlägt. Dies bestätigt, dass der
    neue Test die anderen Tests nicht beeinflusst und auch ohne neuen Code nicht bestanden wurde.
    Dadurch kann der Entwickler ausschließen, dass ein Test immer Korrekt ist, wodurch das Vertrauen
    des Entwicklers in den Test gesteigert wird.
    \item Schreiben des Code, dabei ist unwichtig wie unschön der Code ist, Hauptsache, der Test wird
    bestanden. Die Code Qualität spielt hier keine Rolle, da sie in Punkt Fünf überarbeitet wird.
    Der Entwickler darf in diesem Prozess auch keinen Weiteren Code hinzufügen, der nicht notwendig
    ist um den Test zu bestehen.
    \item Alle Test durchführen und überprüfen ob alle Tests erfolgreich waren. Sollte dies nicht der
    Fall sein muss Punkt Drei wiederholt werden.
    \item Der neue Code muss aufgeräumt werden. Dieser Punkt ist sehr wichtig, da in Punkt Drei der
    Code nur Funktionieren muss kann es sein, dass hier Code entstanden ist, der nicht den Anforderungen
    entspricht. Sämtliche Objekte sollten einen aussagekräftigen Namen erhalten und der Code sollte,
    sofern dies nötig ist an einen Ort verlegt werden, der seiner Logischen Aufgabe entspricht und
    Duplikationen sollten entfernt werden. Nach jeder Aufräumaktion sollten die Tests noch einmal
    ausgeführt werden, um zu verifizieren, dass alles noch so Funktioniert wie es sollte.
\end{enumerate}
\noindent
Dieser Vorgang soll für jedes Feature wieder holt werden bis alle Anforderungen an das Programm
erfüllt sind. Dabei sollten die einzelnen Tests und Features möglichst klein sein.

