% !TeX root = ../bachelor.tex
\section{Einleitung}\label{einleitung}

TDD wird in der heutigen Softwareentwicklung immer verbreiteter und beliebter.
Die Ansprüche an Software sind in den letzten Jahren immer weiter gestiegen. Dies
liegt vor allem an der Reichweite, die Software heute hat. So besitzen nach \cite{FraukeSchobelt:Smartphone}
im Jahr 2018 bereits circa 66\% aller Menschen ein Smartphone. Im Arbeitsleben
ist ein PC meist gar nicht mehr weg zu denken. Doch mit den steigenden
Nutzerzahlen steigen auch die Anforderungen, welche die Nutzer an die Software stellen.
Somit wird die Anzahl der gefundenen Bugs dementsprechend größer.

Im weltweiten Markt gibt es viele große Unternehmen, die gegenseitig um die
Nutzer kämpfen. Selbstverständlich präferieren die Nutzer denjenigen Anbieter, welcher die bessere
Software bietet. Dies kann sich heute jedoch stetig ändern. Mit der steigenden Anzahl
an Bugs, die gefunden werden, steigt auch die Anzahl der Nutzer, die von diesen Bugs betroffen
sind. Diese Bugs sollen natürlich schnellstmöglich gefixt werden um so zu verhindern,
dass die Nutzer die Software wechseln.

So schwer es ist seine Nutzer zu halten, umso schwerer ist es, zum Start einer Software
Nutzer zu akquirieren. Es gibt bereits Software, die ähnliche Services anbieten.
So ist es noch schwerer dem Markt beizutreten. Die Anforderungen werden durch die
bereits am Mark tätigen Firmen gesetzt, damit sollten Fehler, die bereits gelöst
wurden nicht mehr auftauchen.

Für Unternehmen sind diese Anforderungen meist schwer zu meistern, weshalb Software
meist mit Fehlern released wird, um diese dann von den Nutzern aufdecken zu lassen und
zu fixen.

Sowohl als Entwickler als auch als Arbeitgeber muss man sich bei der Wahl der
Programmiersprache Gedanken darüber ob und wie einfach eine Sprache für
Test-driven development ein zu setzen ist. Der wichtigste Aspekt bei diesem
Prozess ist die Auswahl und die Qualität der von der Sprache zur Verfügung
gestellten Tools.

% !TeX root = ../bachelor.tex
\subsection{Die Programmiersprache Python}\label{einleitung:python}

Blubb

% !TeX root = ../bachelor.tex
\subsection{Test-driven development}\label{einleitung:tdd}

In diesem Kapitel wird das Thema testen und Test-driven development behandelt
um ein Grundlegendes Verständnis von TDD zu schaffen mit welchem diese Arbeit
besser zu verstehen ist.

