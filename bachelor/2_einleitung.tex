% !TeX root = ../bachelor.tex
\section*{Einleitung}\label{einleitung}
\addcontentsline{toc}{section}{Einleitung}

Die Ansprüche an Software sind in den letzten Jahren immer weiter gestiegen.
Dies liegt vor allem an der Reichweite, die Software heute hat, so besitzen
im Jahr 2018 bereits circa 66\% aller Menschen ein Smartphone
(\cite{FraukeSchobelt:Smartphone}) und im Arbeitsleben ist ein PC meist nicht
mehr weg zu denken. Die dadurch benötigte Software muss bestimmte
Anforderungen, welche die Nutzer an diese stellen gewährleisten und
sicherstellen. Mit steigenden Nutzer Zahlen, steigen oft auch die Anforderungen
an eine Software sowie die gefundenen \Glspl{bug} in dieser.

Im weltweiten Markt gibt es viele große Unternehmen, die gegenseitig um ihre
Nutzer kämpfen. Die Anwender präferieren meist denjenigen Anbieter, welcher die
bessere Software bietet. An dieser Stelle ist der Software-Architekt einer
jeden Anwendung gefragt, welche Technologie eingesetzt wird um die Qualitativen
Anforderungen an diese zu gewährleisten. Eine dieser Technologien ist das
Test-driven development, welches sich durch eine hohe resultierende
Test-Abdeckung ihren Namen gemacht hat. Bei dieser Methode werden zuerst die
Tests und dann der Code geschrieben, um so später sicher stellen zu können,
dass die Software alle Anforderungen erfüllt und keine \Glspl{bug} enthält.

Gerade zum eintritt einer Neuen Software ist es für Unternehmen schwer Nutzer
zu akquirieren. Andere Anwendungen, die bereits auf dem Markt sind, haben
schon viele ihrer \Glspl{bug} gefixed und verfügen über eine durch den
Anwender ausgiebig getestete Anwendung. Um eine neue Applikation in diesen
Markt zu bringen, ist es erforderlich, dass diese zum Start ohne Fehler läuft.
Dies kann beispielsweise mit Test-driven development sicher gestellt werden.

Eine der Programmiersprachen mit der Test-driven development möglich ist, ist
Python. Python wird heutzutage von vielen Entwicklern immer häufiger eingesetzt
und ist Mittlerweile die am zweit meisten geliebte und am vierthäufigsten
genutzte Sprache Weltweit (\cite{stackoverflow:2019}).

Aufgrund der Zunehmenden Beliebtheit, sowie meiner Eigenen Vorliebe für die
Programmiersprache Python (Die Bezeichnung dafür ist \Gls{Pythonista}) habe ich
mich dazu entschieden meine Bachelorarbeit dieser zu widmen. Da ich beginnend
mit dem Praxissemester und danach als Werkstudent in einer Firma an der
Entwicklung der Hausinternen Test-Umgebung mitgewirkt habe, wurde ich immer
interessierter an dem Thema Testen. Durch die Beschäftigung mit Tests in meiner
Freizeit bin ich irgendwann auf das Test-driven development gestoßen und war
sofort überzeugt, das dies eine Technologie ist, die ich anwenden möchte. Doch
leider war die Auswahl der Tools, welche für Test-driven development in Frage
kamen so hoch, das ich Angst hatte etwas verpassen zu können, da ich ein Tool
einem anderen vorzog. Beim Analysieren der Tools viel mir auf, dass viele der
Tools veraltet oder ungeeignet waren, was einen Einsatz für Test-driven
development ausschloss. Die übrig gebliebenen Tools waren in Ihrem Umfang so
mächtig, dass ich darüber lustig machte, dass es möglich sei eine ganze Arbeit
darüber zu schreiben, weshalb ich das Projekt erst einmal zur Seite legte.

Kurze Zeit später musste ich mich für meine Bachelorarbeit bereit machen und
kam wieder auf das Thema zurück. Da ich bereits ein fundiertes Wissen bezüglich
Test-driven development besaß, entschied ich mich dazu dieses Wissen aufs
Papier zu bringen um anderen Entwicklern die Arbeit zu ersparen die einzelnen
Tools Analysieren zu müssen.

Das Ziel dieser Arbeit ist es, verschiedene Aktuelle Test-Tools der
Programmiersprache Python auf ihre Tauglichkeit für den Einsatz von Test-driven
development zu untersuchen und zu vergleichen. Dabei wird zuerst jedes Tool und
seine Funktionalitäten beschrieben. Im Anschluss wird ein jedes auf die
verschiedene Anforderungen untersucht, welche es erfüllen muss um für
Test-driven development in Frage zu kommen. Dies geschieht anhand eines
erstellten Codes Beispiels, welche gewisse Funktionalität liefert (oder nicht).
Dieses Beispiel wird mithilfe der Verschiedenen Tools auf die gleiche Art
getestet. Die dabei entstehenden Komplikationen sowie der benötigte Aufwand
ergeben die Ergebnisse, welche für die Analyse benötigt werden.

Das Ergebnis der Arbeit soll aufzeigen Welche Tools zusammen die optimale
Verbindung bieten um Test-driven development durch zu führen. Dabei erwarte ich
mir auch tiefere Einblicke in das Test-driven development zu erhalten um so
heraus zu finden ob diese Technologie für mich relevant ist.

Dazu wird in Kapitel \ref{definition} zunächst geklärt, welche Eigenschaften
die Programmiersprache Python hat, sowie die Definition von Test-driven
development. Die genutzte Methodik wird in Kapitel \ref{methodik} beschrieben
und anschließend in Kapitel \ref{python-tools} angewendet. Die Analyse ist in
vier Unterpunkte gegliedert, in Punkt \ref{python-tools:stdlib} werden zunächst
die Tools der Standardbibliothek analysiert. Anschließend werden unter Punkt
\ref{python-tools:extlib} die Tools, welche nicht in der Standardbibliothek
sind analysiert. Darauffolgend werden unter Punkt \ref{python-tools:vergleich}
die analysierten Tool verglichen um dann unter Punkt
\ref{python-tools:kombination} mögliche Kombinationen dieser auf zu Zeigen.
Bevor in Kapitel \ref{fazit} das Fazit gezogen wird, werden die Ergebnisse der
Arbeit in Kapitel \ref{diskussion} diskutiert.