% !TeX root = ../bachelor.tex
\subsubsection{Python Test-Tools}\label{python-tools:extlib:python}

Auf \url{https://wiki.python.org/moin/PythonTestingToolsTaxonomy} werden viele
externe Tools gelistet, jedoch scheinen viele inaktiv zu sein da ihre \glspl{commit}
teilweise mehr als ein Jahr zurück liegen. Dies ist weder für eine Bibliothek noch
für ein Tool ein gutes Zeichen, da sich die Anforderungen stetig ändern und niemals
alle Bugs gefixt sind.
\newline
\\
Manche der dort aufgelisteten Tools sind bereits oder werden gerade in andere Tools
integriert. Dies kann zum einem sein, da ein Tool eine Erweiterung für ein anderes
war und die Entwickler die Änderungen angenommen haben und zum anderen um die Tools
zu verbessern und mehr Entwickler zur Verfügung zu haben. Eventuell sind auch
andere Gründe dafür verantwortlich, jedoch war dies der Grund bei zum Beispiel
\href{https://github.com/Yelp/Testify/}{Testify}
von Yelp \footnote{https://github.com/Yelp/Testify/}.
\newline
\\
Lässt man die Erweiterungen von Tool zunächst außen vor und ignoriert Tools deren
letzter \gls{commit} älter als ein Jahr ist, so bleiben nach
\cite{wiki.python:PythonTestingToolsTaxonomy} folgende Modul Test-Tools zur Verfügung:
\begin{itemize}
    \item \href{http://pytest.org/latest/}{py.test}\footnote{http://pytest.org/latest/}
    \item \href{http://pytest.org/latest/}{nose}\footnote{http://pytest.org/latest/}
    \item \href{https://www.reahl.org/docs/4.0/devtools/tofu.d.html}{tofu}\footnote{https://www.reahl.org/docs/4.0/devtools/tofu.d.html}
    \item \href{https://pypi.org/project/zope.testing/}{zope.testing}\footnote{https://pypi.org/project/zope.testing/}
\end{itemize}
\noindent
Als weitere Kategorie werden \Gls{mock}-Tools geführt. Zumeist erweitern diese
bereits bestehende Tools wie zum Beispiel 
