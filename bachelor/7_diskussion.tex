% !TeX root = ../bachelor.tex
\section{Diskussion}\label{diskussion}
Um für die Anwendung von TDD die passenden Tools zu finden, wurden verschiedene
Tool der Programmiersprache Python analysiert und auf ihre Tauglichkeit im
Bezug auf TDD geprüft. Daraus entstand eine Empfehlung für vier Tools die
gemeinsam für TDD genutzt werden können, um so die optimale Funktionalität zum
schreiben von Tests, sowie zu einer optimalen Testabdeckung zu gewähren. Die 
dabei erhaltenen Ergebnisse zeigen auf, dass ein Tool alleine zwar ausreichend 
sein kann für die Anwendung von TDD, jedoch ist die Verwendung von mehreren 
Tools zusammen für den Entwickler und für das Produkt besser.

Im Rahmen dieser Arbeit wurden lediglich Tools analysiert, deren letzte
Aktivität nicht weiter als ein Jahr zurück liegt. Tools die eine Erweiterung zu
einem anderen Tool darstellen wurden nicht behandelt, da dies den Rahmen der
Arbeit gesprengt hätte. Aus diesem Grund kann es sein, dass ein Beliebtes oder
Bekanntes Tool in dieser Arbeit nicht behandelt wurde. Das bedeutet nicht, dass
diese Tool zwangsläufig nicht geeignet ist für TDD. Die Ergebnisse dieser
Arbeit sind also nicht repräsentativ für alle Tools die es für die
Programmiersprache Python gibt und würden eventuell anders ausfallen, wenn
andere Tools behandelt worden würden.

%% !TeX root = ../bachelor.tex
\subsection{Stärken von Test-driven development}\label{diskussion:staerken}

Blubb

%% !TeX root = ../bachelor.tex
\subsection{Schwächen von Test-driven development}\label{diskussion:schwaechen}

Dieses Kapitel soll die Schwächen von TDD Diskutieren, dabei wird in \nameref{diskussion:schwaechen:umgehen}
beschrieben welche dieser Schwächen umgangen werden können und in \nameref{diskussion:schwaechen:akzeptieren}
werden jene Schwächen die nicht vermeidbar sind analysiert.

% !TeX root = ../bachelor.tex
\subsubsection{Welche Möglichkeiten gibt es diese Schwächen zu umgehen?}\label{diskussion:schwaechen:umgehen}

Blubb

% !TeX root = ../bachelor.tex
\subsubsection{Unumgängliche Schwächen}\label{diskussion:schwaechen:akzeptieren}

Unumgängliche Schwächen

%% !TeX root = ../bachelor.tex
\subsection{Welche wirtschaftlichen Aspekte müssen bei Test-driven development beachtet werden?}\label{diskussion:wirtschaft}

Blubb

%% !TeX root = ../bachelor.tex
\subsection{Zusammenfassung}\label{diskussion:zusammenfassung}

Blubb

