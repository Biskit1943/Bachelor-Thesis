% !TeX root = ../bachelor.tex
\paragraph{mocktest}\label{python-tools:mocktest}\mbox{}
\newline
\lstinline{mocktest} ist ein \gls{mock}ing Tool, das von
\href{http://rspec.info/}{\lstinline{rspec}}\footlabel{rspec}{http://rspec.info/}
inspiriert wurde. Dabei soll \lstinline{mocktest} kein Port von
\lstinline{rspec}\footref{rspec} sein, sondern eine kleine leichtere
Version in Python(\cite{mocktest:doc}). Da sich \lstinline{mocktest} derzeit in
der Version \lstinline{0.7.3} (Stand 16. April 2019) befindet, sind noch nicht
alle Features vollkommen entwickelt und ausgereift, dennoch lässt sich das Tool
bereits produktiv einsetzen, da die Basisfunktionalität aus
\lstinline{rspec}\footref{rspec} bereits implementiert wurde.
\newline

Das Hauptfeature von \lstinline{mocktest} ist die Testisolation. Diese
verhindert, dass \Gls{mock}objekte außerhalb eines Test Falles weiterhin
Veränderungen vornehmen und so andere Tests beeinflussen. Demnach werden Tests
immer innerhalb einer \lstinline{mocktest.MockTransaction} ausgeführt, sofern
diese einen oder mehrere \Glspl{mock} nutzen.

Innerhalb dieses Kontextes stehen dem Entwickler verschiedene Möglichkeiten zur
Verfügung Tests aus zu führen. So lässt sich beispielsweise überprüfen, ob eine
Funkion oder Methode aufgerufen wurde. Dabei ist es nebensächlich, ob es sich
hierbei um ein globales oder lokales Objekt handelt. Des Weiteren lassen sich
auch \Glspl{stub} nutzen, mit deren Hilfe Funktionen und Methoden präpariert
werden können. Wie auch beim Überprüfen des Aufrufs ist es ebenfalls irrelevant,
ob es sich um ein Lokales oder Globales Objekt handelt.
Die Präparation lässt sich nach Bedarf auch anpassen, so kann der Entwickler
beispielsweise festlegen, dass nur bei einem bestimmten Aufruf mit bestimmten
Parametern das \Gls{mock} Objekt aufgerufen wird. Andernfalls wird das originale
Objekt genutzt.
\newline

\lstinline{mocktest} nutzt \lstinline{unittest} als Basis für seine
Testumgebung. \lstinline{mocktest.TestCase} erbt von
\lstinline{unittest.TestCase}, wodurch die gesamte Funktionalität von
\lstinline{unittest} geerbt wird. \lstinline{mocktest.TestCase} führt dabei in
seiner \lstinline{setUp()} und \lstinline{tearDown()} Methode Code aus, der
benötigt wird, um \lstinline{mocktest} nutzen zu können. Zusätzlich wird mit
\lstinline{mocktest} unittest um eine \lstinline{assert} Methode erweitert.
\lstinline{assertRaises()} verfügt über eine verbesserte Funktionalität für die
Überprüfung auf Exceptions. Ist es jedoch erwünscht unittest nicht zu verwenden,
so ist es möglich \lstinline{mocktest.MockTransaction} als Kontextmanager mit
\lstinline{with MockTransaction:} zu nutzen.
\lstinline{MockTransaction.__enter__()} und
\lstinline{MockTransaction.__exit__()} sind die Methoden, die in
\lstinline{setUp()} und \lstinline{tearDown()} automatisch aufgerufen werden,
weshalb es möglich ist, \lstinline{mocktest} mit jedem Test-runner zu verwenden,
der \Glspl{fixture} unterstützt.
\newline

Für die Überprüfung von \Glspl{mock} kann entweder \lstinline{when(obj)} oder
\lstinline{expect(obj)} verwendet werden. Der Unterschied hier ist lediglich,
dass \lstinline{when(obj)} nicht überprüft ob eine Methode aufgerufen wurde oder
nicht. \lstinline{expect(obj)} hingegen überprüft ob die Methode aufgerufen
wurde. So würde Beispielsweise \lstinline{expect(os).system} einen Fehler
werfen, wenn \lstinline{os.system} nicht aufgerufen wurde. Bei
\lstinline{when(os).system} wäre dies nicht der Fall. Ist es zu einem späteren
Zeitpunk gewünscht zu überprüfen wie oft und mit welchen Parametern ein Objekt
aufgerufen wurde, ist es möglich \lstinline{received_calls} ab zu prüfen. Diese
Variable ist eine Liste von allen getätigten Aufrufen und ihren Parametern.
\newline

Folgende Features werden von \lstinline{mocktest} unterstützt:
Rückgabelvariation der \Glspl{stub} mit
\lstinline{.and_return(erg1, erg2,ergX)} oder alternativ
\lstinline{.then_return()}, Festlegen der erwarteten Aufrufe einer
Funktion/Methode mit \lstinline{.at_least(n)}, \lstinline{.at_most(n))},
\lstinline{.between(a, b)} und \lstinline{.exact(n)}, \gls{stub}en von
Methoden/Funktionen mit einer anderen Funktion/Methode mit Hilfe von
\lstinline{.then_call(ersatz_func)}, setzen von Klassen-variablen mit
\lstinline{.with_children(**kwargs)} und das setzen von neuen Klassen-Methoden
mit Hilfe von \lstinline{.with_method(**kwargs)} (\cite{mocktest:doc}).
\newline

Diese Features wurden anhand des in Listing \ref{listing:base:my_mock_module}
definierten Codes, mit \lstinline{mocktest} getestet (vgl. Listing
\ref{listing:mocktest:example}). Bei der Implementierung der Tests gab es keine
Komplikationen mit dem Tool. Da \lstinline{mocktest} keine \lstinline{setUp()}
Methode benötigt um zu funktionieren, können hier ein paar Zeilen Code gespart
werden. Durch \lstinline{expect()} und \lstinline{when()} ist es möglich jedes
beliebige Objekt zu nehmen und zu ersetzen. Da die Implementierung keine
Komplikationen aufwies wird hier auf eine weitere Analyse verzichtet.
\newline

Um TDD zu betreiben bietet \lstinline{mocktest} alles, was ein Entwickler zum
\gls{mock}en benötigt. So ist es möglich bestehende Objekte gänzlich oder
teilweise zu ersetzen, oder neue Objekte zu erstellen, welche die Signatur eines
anderen Objektes haben. Auch globale Objekte lassen sich für den Zeitraum eines
Tests ersetzen. Dadurch ist alles an Funktionalität geboten, was ein
\gls{mock}-testing Tool bieten muss.

Bezüglich der Effizienz kann \lstinline{mocktest} problemlos eingesetzt werden,
da quasi kein Vorwissen von Nöten ist um Objekte erfolgreich zu \gls{mock}en.
Die Vorarbeit, die geleistet werden muss, hält sich je nach Anwendungsfall in
Grenzen oder ist so gering, dass sie kein Hindernis darstellt. Wird die von
\lstinline{mocktest} zur Verfügung gestellte Klasse
\lstinline{mocktest.TestCase} für einen Testfall genutzt, so ist bereits alles
fertig und direkt einsetzbar. Selbstverständlich kann
\lstinline{mocktest.TestCase} je nach Bedürfnis um Funktionalität erweitert
werden, jedoch ist dies als unkompliziert zu betrachten.

\lstinline{mocktest} lässt den Entwickler Code schreiben, der sich wie
gesprochen liest. Durch die Methoden \lstinline{when(obj)} und
\lstinline{expect(obj)} lässt sich ein \Gls{mock} erstellen, der mit Methoden
angepasst werden kann, die wenn man sie aneinander reiht, sich wie ein Satz
lesen. Dies ist zum einen durch die verschiedenen Aliase möglich, wie zum
Beispiel \lstinline{.once()} welches \lstinline{.exact(1)} ausführt, als auch
mit \lstinline{.then_return()}. Abseits der eigentlichen Funktionalität, die
mit \lstinline{mocktest} genutzt wird, fällt kein Code an, der geschrieben
werden muss um Tests ausführbar zu machen.