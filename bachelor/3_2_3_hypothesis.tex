% !TeX root = ../bachelor.tex
\paragraph{hypothesis}\label{python-tools:hypothesis}\mbox{}
\newline
Das größte und am weitesten verbreitete Tool für Fuzz-testing mit Python ist
\lstinline{hypothesis}. Recherchen im Internet weisen auf kein weiteres aktuelles
Tool hin, dass Open Source ist, demnach ist \lstinline{hypothesis} das einzige
Tool welches zum Fuzz-testing mit Python derzeit genutzt werden kann (Python 3.7).

Bevor man anfangen kann mit \lstinline{hypothesis} zu testen, muss man verstehen
wie \lstinline{hypothesis} funktioniert. Möchte man Daten haben die einer
Spezifikation entsprechen, benötigt man \lstinline{hypothesis.strategies}. In
\lstinline{hypothesis} sind Strategien Spezifikationen für die Generierung von
Daten. Wäre eine Spezifikation also, dass die Daten \lstinline{Integer} sind, dann
würde man \lstinline{hypothesis.strategies.integers()} verwenden. Diese Strategie
würde verschiedene Daten die \lstinline{Integer} sind ausgeben. Möchte man nur
Werte zwischen A und B, so wäre die Strategie
\lstinline{.integers(min_value=A, max_value=B)}. Dies kann man mit beliebig
vielen Daten Typen und Spezifikationen durchführen, dabei verfügt
\lstinline{hypothesis} über so viele verschiedene Spezifikationen, dass es schwer
sein wird alle zu kennen. Die Basics lassen sich allerdings in der
\href{https://hypothesis.readthedocs.io/en/latest/data.html}{Dokumentation}
\footnote{https://hypothesis.readthedocs.io/en/latest/data.html} finden.

Ist die Spezifikation bekannt und die Strategie gefunden werden Tests mithilfe einer
\Gls{annotation} der Test Funktion übergeben, diese \Gls{annotation} nennt sich
\lstinline{hypothesis.given()}. Ein kleines Beispiel dazu kann in Listing
\ref{listing:hypothesis:given} gefunden werden. \lstinline{.given()} unterstützt
dabei die \lstinline{*args} als auch die \lstinline{**kwargs} Übergabe der Strategien.

Hat man nun einen Fehler gefunden und möchte diesen für alle weiteren Tests testen,
so kann man mit \lstinline{hypothesis.example()} den Test annotieren und die Werte
bei denen der Test fehl geschlagen ist dort übergeben. Alternativ übernimmt dies
\lstinline{hypothesis} durch das speichern des Wertes im Cache, allerdings kann dieser
verloren gehen, oder durch umbenennen einer Funktion invalide werden. Um zu verhindern,
dass der Cache von Entwickler zu Entwickler unterschiedlich ist, lässt sich der Cache
in der VCS einbinden, dazu sollte alles in \lstinline{.hypothesis/examples/} hinzu
gefügt werden. Alle anderen Ordnern in \lstinline{.hypothesis} sind nicht dafür
ausgelegt in das VCS integriert zu werden, da sie zu Merge-Konflikten führen können.

Möchte der Entwickler etwas mehr Information darüber weshalb ein Test fehl geschlagen
ist, so ist es Ihm möglich mit \lstinline{hypothesis.note()} vor einer Assertion
einen Text aus zu geben, der Ihm diese Informationen gibt. \lstinline{note()} wird
allerdings nur aus gegeben, wenn der Test fehl schlägt.

Möchte man ein paar Statistiken darüber was \lstinline{hypothesis} gemacht hat und man
nutzt \lstinline{pytest} als Test runner, so kann man mit \lstinline{--hypothesis-show-statistics}
sich die Statistiken zu den Tests anschauen. Dabei werden folgende Informationen 
zu jeder Test Funktion gezeigt:
\begin{itemize}
    \item Gesamte Anzahl der Durchläufe mit der Anzahl der fehlgeschlagenen und der Anzahl der invaliden Tests.
    \item Die ungefähre Laufzeit des Tests.
    \item Wie viel Prozent davon für die Datengenerierung aufgewendet wurde.
    \item Wieso der Test beendet wurde.
    \item Alle aufgetretenen Events von \lstinline{hypothesis}
\end{itemize}

Ein Event kann allerdings auch vom Entwickler genau so wie \lstinline{note()} genutzt
werden, mit dem Unterschied, dass ein Event immer zu einem Output führt wenn die
Zeile im Code ausgeführt wird (Sofern \lstinline{pytest} und
\lstinline{--hypothesis-show-statistics} genutzt werden).

Mit diesem Wissen lässt sich \lstinline{hypothesis} bereits für die einfachsten Dinge
Nutzen, hat man allerdings etwas Spezifischere Anforderungen, so muss etwas tiefer
in die Materie eingestiegen werden. Angenommen, eine Funktion soll mit einem
\lstinline{Integer} getestet werden, dieser muss aber durch sich selbst teilbar sein.
Dies könnte mit einem \lstinline{if} Statement geprüft werden, würde aber die Anzahl
der relevanten Tests erheblich senken. Standessen ist es möglich auf Strategien filter
an zu wenden. Filter müssen dabei Funktionen sein, die einen Wert filtern und Ihn zurück
geben. Für das eben genannte Beispiel würde die \Gls{annotation} dann so aussehen:
\lstinline{@given(integers().filter(lambda i: i % i == 0))}.

Gibt \lstinline{hypothesis} einem Test Daten die nicht gefiltert werden können oder sollen,
so ist es möglich mit \lstinline{hypothesis.assume()} eine Annahme auf zu stellen. Ist die
Annahme falsch, so wird dieser Test übersprungen. Allerdings kann dies zu Fällen führen bei
denen \lstinline{hypothesis} keine validen Daten finden kann, was zu einer Exception und
zum fehlschlagen des Tests führt.

\lstinline{hypothesis} verfügt noch über weit mehr Funktionalität als hier genannt werden kann,
wie zum Beispiel das verketten von Strategien. Die Dokumentation zu allem, was
\lstinline{hypothesis} ohne Erweiterungen erschaffen kann, kann
\href{https://hypothesis.readthedocs.io/en/latest/data.html}{online}\footnote{https://hypothesis.readthedocs.io/en/latest/data.html}
eingesehen werden.

Auch wenn die Standard Bibliothek von \lstinline{hypothesis} bereits sehr viele Strategien enthält,
so gibt es dennoch einiges was mithilfe von Erweiterungen dazu gewonnen werden kann. So werden
von \lstinline{hypothesis} selbst First-party Erweiterungen zur Verfügung gestellt. Dies dient
vor allem dazu die Abhängigkeiten von der Standard Bibliothek so gering wie möglich zu halten
(eine zusätzliche Abhängigkeit) und zum anderen um Kompatibilitätsprobleme zu verhindern. Aus
diesem Grund können diese zusätzlichen Erweiterungen extra installiert werden, dazu finden sich
in der Dokumentation drei Seiten, eine Generelle\footnote{https://hypothesis.readthedocs.io/en/latest/extras.html},
eine für Django\footnote{https://hypothesis.readthedocs.io/en/latest/django.html} und eine für
Wissenschaftliche Module\footnote{https://hypothesis.readthedocs.io/en/latest/numpy.html}.
Zusätzlich zu den First-party Erweiterungen, gibt es auch noch Erweiterungen der Community,
eine kleine Liste dazu kann in der Dokumentation\footnote{https://hypothesis.readthedocs.io/en/latest/strategies.html}
gefunden werden oder durch das Explizite suchen einer Strategie im Internet.

Ein kleines Beispiel einer Anwendung von \lstinline{hypothesis} war bereits in Listing
\ref{listing:hypothesis:given} zu sehen, für weitere Beispiele kann in der Dokumentation unter
\href{https://hypothesis.readthedocs.io/en/latest/quickstart.html}{"'Quick start guide"'}
\footnote{https://hypothesis.readthedocs.io/en/latest/quickstart.html} und unter
\href{https://hypothesis.readthedocs.io/en/latest/examples.html}{"'Some more examples"'}
\footnote{https://hypothesis.readthedocs.io/en/latest/examples.html} nachgesehen werden.

Da \lstinline{hypothesis} das derzeit einzige aktiv Entwickelte Open Source Tool ist, müssen
Entwickler, die \gls{fuzz}-testing betreiben wollen auf dieses Tool zurück greifen. Doch das
ist nichts Negatives, \lstinline{hypothesis} bietet einem Entwickler alles um seine einzelnen
Funktionen und Methoden auf Herz und Nieren zu testen. Selbstverständlich kommt es hier stets
auf den Anwendungsfall an, denn nicht jede Funktion oder Methode lässt sich mit zufälligen Werten
wirklich Effektiv testen, jedoch bietet \lstinline{hypothesis} dort wo es sinnvoll ist alles
um ausgiebig zu testen. Sollten die in der Standard Bibliothek enthaltenen Strategien nicht
reichen, so bietet \lstinline{hypothesis} mit seinen First-party Erweiterungen, Erweiterungen die
zu 100\% mit \lstinline{hypothesis} Funktionieren. Dadurch ist die Standard Bibliothek besonders
schlank.

\lstinline{hypothesis} selbst lässt den Entwickler sehr schnell einsteigen, da nicht sonderlich
viel Vorarbeit notwendig ist um \lstinline{hypothesis} an zu wenden. Allerdings kommt mit den
steigenden Anforderungen an die Daten auch die steigende Vorarbeit, die nötig ist um
\gls{fuzz}-testing zu betreiben.

Die Komplexität von \lstinline{hypothesis} ist allerdings sehr hoch, so viel wie
\lstinline{hypothesis} bietet, so komplex kann es auch werden es an zu wenden. Hat man spezielle
Anforderungen abseits der Standard Werte wie \lstinline{int} oder \lstinline{str}
ist mehr Arbeit und Verständnis Notwendig. Dies kann schon mal dazu führen das man als
Entwickler etwas Zeit investieren muss um die Spezifikationen auf eine Strategie an zu wenden.
Je nach Komplexität der erforderlichen Daten ist es also leicht bis schwer \lstinline{hypothesis}
zu verwenden. Bei Komplexeren Anforderungen kann es auch dazu führen das der geschriebene Code
unübersichtlich wird, wenn er nicht richtig dokumentiert wurde.

Auch wenn \lstinline{hypothesis} als Erweiterung in dieser Arbeit behandelt wird, so bietet es selbst
allerdings auch Erweiterungen für sich selbst. Dies liegt aber daran, dass man den Entwicklern
nicht standardmäßig alle Funktionalität an die Hand gibt, die vermutlich gar nicht verwendet wird.
Und auch wenn \lstinline{hypothesis} über sehr viele Strategien verfügt, so hat die Community dennoch
weitere Strategien entwickelt die zusätzlich genutzt werden können.