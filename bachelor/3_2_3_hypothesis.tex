% !TeX root = ../bachelor.tex
\paragraph{hypothesis}\label{python-tools:hypothesis}\mbox{}
\newline
Das größte und am weitesten verbreitete Tool für Fuzz-testing mit Python ist
\lstinline{hypothesis}. Recherchen im Internet weisen auf kein weiteres aktuelles
Tool hin, dass Open Source ist, demnach ist \lstinline{hypothesis} das einzige
Tool welches zum Fuzz-testing mit Python derzeit genutzt werden kann (Python 3.7).

Bevor man anfangen kann mit \lstinline{hypothesis} zu testen, muss man verstehen
wie \lstinline{hypothesis} funktioniert. Möchte man Daten haben die einer
Spezifikation entsprechen, benötigt man \lstinline{hypothesis.strategies}. In
\lstinline{hypothesis} sind Strategien Spezifikationen für die Generierung von
Daten. Wäre eine Spezifikation also, dass die Daten \lstinline{Integer} sind, dann
würde man \lstinline{hypothesis.strategies.integers()} verwenden. Diese Strategie
würde verschiedene Daten die \lstinline{Integer} sind ausgeben. Möchte man nur
Werte zwischen A und B, so wäre die Strategie
\lstinline{.integers(min_value=A, max_value=B)}. Dies kann man mit beliebig
vielen Daten Typen und Spezifikationen durchführen, dabei verfügt
\lstinline{hypothesis} über so viele verschiedene Spezifikationen, dass es schwer
sein wird alle zu kennen. Die Basics lassen sich allerdings in der
\href{https://hypothesis.readthedocs.io/en/latest/data.html}{Dokumentation finden}\footnote{https://hypothesis.readthedocs.io/en/latest/data.html}.

Ist die Spezifikation bekannt und die Strategie gefunden werden Tests mithilfe einer
\Gls{annotation} der Test Funktion übergeben, diese \Gls{annotation} nennt sich
\lstinline{hypothesis.given()}. Ein kleines Beispiel dazu kann in Listing
\ref{listing:hypothesis:given} gefunden werden. \lstinline{.given()} unterstützt
dabei die \lstinline{*args} als auch die \lstinline{**kwargs} Übergabe der Strategien.

Hat man nun einen Fehler gefunden und möchte diesen für alle weiteren Tests testen,
so kann man mit \lstinline{hypothesis.example()} den Test annotieren und die Werte
bei denen der Test fehl geschlagen ist dort übergeben. Möchte der Entwickler etwas
mehr Information darüber weshalb ein Test fehl geschlagen ist, so ist es Ihm möglich
mit \lstinline{hypothesis.note()} vor einer Assertion einen Text aus zu geben, der
Ihm diese Informationen gibt. \lstinline{note()} wird allerdings nur aus gegeben,
wenn der Test fehl schlägt.

Möchte man ein paar Statistiken darüber was \lstinline{hypothesis} gemacht hat und man
nutzt \lstinline{pytest} als Test runner, so kann man mit \lstinline{--hypothesis-show-statistics}
sich die Statistiken zu den Tests anschauen. Dabei werden folgende Informationen 
zu jeder Test Funktion gezeigt:
\begin{itemize}
    \item Gesamte Anzahl der Durchläufe mit der Anzahl der fehlgeschlagenen und der Anzahl der invaliden Tests.
    \item Die ungefähre Laufzeit des Tests.
    \item Wie viel Prozent davon für die Datengenerierung aufgewendet wurde.
    \item Wieso der Test beendet wurde.
    \item Alle aufgetretenen Events von \lstinline{hypothesis}
\end{itemize}

Ein Event kann allerdings auch vom Entwickler genau so wie \lstinline{note()} genutzt
werden, mit dem Unterschied, dass ein Event immer zu einem Output führt wenn die
Zeile im Code ausgeführt wird (Sofern \lstinline{pytest} und
\lstinline{--hypothesis-show-statistics} genutzt werden).

Mit diesem Wissen lässt sich \lstinline{hypothesis} bereits für die einfachsten Dinge
Nutzen, hat man allerdings etwas Spezifischere Anforderungen, so muss etwas tiefer
in die Materie eingestiegen werden. Angenommen, eine Funktion soll mit einem
\lstinline{Integer} getestet werden, dieser muss aber durch sich selbst teilbar sein.
Dies könnte mit einem \lstinline{if} Statement geprüft werden, würde aber die Anzahl
der relevanten Tests erheblich senken. Standessen ist es möglich auf Strategien filter
an zu wenden. Filter müssen dabei Funktionen sein, die einen Wert filtern und Ihn zurück
geben. Für das eben genannte Beispiel würde die \Gls{annotation} dann so aussehen:
\lstinline{@given(integers().filter(lambda i: i % i == 0))}.

Gibt \lstinline{hypothesis} einem Test Daten die nicht gefiltert werden können oder sollen,
so ist es möglich mit \lstinline{hypothesis.assume()} eine Annahme auf zu stellen. Ist die
Annahme falsch, so wird dieser Test übersprungen. Allerdings kann dies zu Fällen führen bei
denen \lstinline{hypothesis} keine validen Daten finden kann, was zu einer Exception und
zum fehlschlagen des Tests führt.

\lstinline{hypothesis} verfügt noch über weit mehr Funktionalität als hier genannt werden kann,
wie zum Beispiel das verketten von Strategien. Die Dokumentation zu allem, was
\lstinline{hypothesis} ohne Erweiterungen erschaffen kann, kann
\href{https://hypothesis.readthedocs.io/en/latest/data.html}{online}\footcite{https://hypothesis.readthedocs.io/en/latest/data.html}
eingesehen werden.

\textbf{Erweiterungen erklären}
\textbf{Code Beispiel von simplen sachen}