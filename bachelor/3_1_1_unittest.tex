% !TeX root = ../bachelor.tex
\subsubsection{unittest}\label{python-tools:unittest}

Das von JUnit inspirierte (\cite{docs.python:unittest}) Tool
\lstinline{unittest}, ist Bestandteil der Python Standardbibliothek und bietet
seit jeher seinen Nutzern ein umfangreiches Repertoire an Funktionen zum Testen
von Python Code.
\noindent
Die Funktionalität von \lstinline{unittest} lässt sich mit folgenden Punkten
beschreiben:\Gls{fixture}, zum Präparieren der Tests, Testfälle zum Gliedern
einzelner Tests, Testumgebungen zum Gliedern von zusammengehörigen Tests und
Test runner zum Ausführen von Testumgebungen oder Testfällen.
Diese Funktionalität von \lstinline{unittest} ermöglicht es Anwendern eine
komplette Testumgebung zu erstellen und zu nutzen.

Dazu wird die Testumgebung, eines der Hauptfeatures, von \lstinline{unittest}
verwendet, mit der es möglich ist, Tests in logische Klassen zu gliedern. Es
besteht die Möglichkeit ohne die Testumgebung \lstinline{unittest} ein zu
setzen, jedoch ermöglichen diese eine strukturierte Arbeitsumgebung, sowie die
Verwendung von \Glspl{fixture}.

Jede Testumgebung verfügt über Testfälle, welche einen Test repräsentieren.
Ist eine \lstinline{setUp()} und/oder \lstinline{tearDown()} Methode für eine
Testumgebung definiert, so werden diese vor und nach jedem Test ausgeführt.
Alternativ kann dies aber auch nur bei einer Initialisierung der Testumgebung
und beim Verlassen dieser geschehen. Allerdings sind die Tests danach nicht
mehr untereinander unabhängig, weshalb sie dann "`Integration Tests"' genannt
werden.

Testfälle bestehen aus den jeweiligen Tests, die mit Hilfe der verschiedenen
\lstinline{assert} Methoden von \lstinline{unittest} geprüft werden. Eine
\lstinline{assert} Methode stellt eine Behauptung auf, die getroffen werden
muss, damit ein Test als bestanden angesehen wird. Die von \lstinline{unittest}
bereit gestellten Methoden unterscheiden sich ein wenig von der
\lstinline{assert} Methode aus der STDLIB. Eine Überprüfung auf
\lstinline{True} würde mit der aus der STDLIB stammenden Funktion so aussehen:
\lstinline{assert xy is True}, während \lstinline{unittest} eine Methode bereit
stellt, mit der es etwas kürzer und leichter zu lesen ist,
\lstinline{assertTrue(xy))}. \lstinline{unittest} stellt noch viele weitere
dieser Methoden zur Verfügung. Die Dokumentation verweist auf diese
\footnote{https://docs.python.org/3/library/unittest.html}.

Ein kleines Beispiel zu den \lstinline{assert} Methoden ist in Listing
\ref{listing:unittest:example} zu finden. Dabei wurde die Funktion
\lstinline{my_pow()} aus \lstinline{my_module} (Listing
\ref{listing:base:my_module}) getestet. Dabei ist zu erkennen, wie problemlos
es möglich ist, eine Funktion auf einen Wert zu prüfen. Dies geschieht
einerseits anhand eines selbst errechneten Wertes und andererseits anhand der
Quadratfunktion aus der STDLIB.

Um mit TDD Units unabhängig testen zu können, müssen abhängige Units mit einem
\Gls{mock} oder einem \Gls{stub} ersetzt werden. Durch die \Glspl{fixture} ist
es bereits möglich den Test oder die Tests so vor zu
bereiten, dass diese funktionieren. Jedoch bieten \Glspl{mock} einfachere und
schnellere Möglichkeiten Funktionen, Methoden, Klassen usw. zu imitieren.

Jedoch gibt es in der STDLIB eine Erweiterung zu unittest mit dem Namen
\lstinline{unittest.mock}, welche unter eben diesem importiert werden kann, um 
die \gls{mock}ing Funktionalität zu bekommen. Diese Erweiterung, auch als
\lstinline{submodule} bezeichnet, ist Teil der STDLIB seit Python 3.3 wie in
PEP417 definiert wurde (\cite{python.org:PEP417}).

\lstinline{unittest} bietet des Weiteren ein CLI, mit welchem es dem Benutzer 
möglich ist, seine Tests gebündelt aus zu führen und aus zu werten. Mit dem CLI 
ist es auch möglich, automatisch Tests in einem Ordner zu 
"`entdecken"'(discover) und aus zu führen. Dadurch ist es sehr leicht, neue 
Tests in ein bestehendes Test System ein zu führen und diese ohne Veränderungen 
am bestehenden System aus zu führen.

Die folgenden Listings \ref{listing:unittest:example_output_success} und
\ref{listing:unittest:example_output_failure} zeigen wie ein erfolgreicher und 
ein misslungener Test aussehen können. Beide Outputs stammen aus dem Test
der \lstinline{my_pow()} Funktion aus Listing \ref{listing:base:my_module}.
In beiden Listings ist gut zu erkennen, ob und was bei einem Test 
fehlgeschlagen ist. Der erfolgreiche Test zeigt dem Nutzer sofort wie viele 
Test in wie vielen Sekunden gelaufen sind. Falls gewünscht kann der Nutzer mit  
\lstinline|--verbose| sich noch mehr Informationen anzeigen lassen.
Bei misslungenen Tests ist im Output immer der \gls{stacktrace} abgebildet um 
so den Fehler bis zur Wurzel zurück verfolgen zu können. Auch der Fehler selbst 
wird in den Output geschrieben, wie in Zeile 8 in Listing 
\ref{listing:unittest:example_output_failure} zu erkennen ist. Am Ende wird 
auch noch einmal angezeigt, wie viele Tests fehlgeschlagen sind.

Um einen Vergleich zwischen den unit-testing Tools zu schaffen, wurde die in
Listing \ref{listing:base:my_module} definierte Klasse \lstinline{Item} mit
\lstinline{unittest} getestet. Der Code dazu ist in Listing
\ref{listing:unittest:advanced} zu finden.
Die Tests wurden in einer Testumgebung gegliedert und verfügen über eine
\lstinline{setUp()} und \lstinline{tearDown()} \Gls{fixture}. Die
\lstinline{setUp()} Methode startet eine Datenbank \lstinline{session} auf der
gearbeitet werden kann, sowie ein Objekt \lstinline{self.item}, mit welchem in
den jeweiligen Tests gearbeitet wird. Die \lstinline{tearDown()} Methode
sorgt dafür, dass die Datenbank nach jedem Test geschlossen wird, um sicher zu 
stellen, dass ungenutzte Datenbank Sessions offen sind. Im ersten Test wird 
überprüft, ob ein der Datenbank hinzugefügtes Objekt auch wirklich in dieser 
ist. Im zweiten Test wird überprüft, ob es möglich ist, eine nicht existierende 
externe Funktion zu ersetzen, sodass die Tests erfolgreich verlaufen.
\newline

Im Aspekt Anwendbarkeit bietet \lstinline|unittest| alles um als Tool für TDD in
Frage zu kommen. Dies kann jedoch nur unter Einbezug der Erweiterung
\lstinline|unittest.mock|. Da sich das Tool in der STDLIB befindet, sind keine
weiteren Pakete zur Verwendung von \lstinline{unittest} nötig.

Die Effizienz Tests aus zu werten ist hoch. Ein Entwickler kann mit ein paar
Zeilen Code eine Testumgebung, die \Glspl{fixture} besitzt, sowie verschiedene 
Testfälle, die von ihnen abhängig sind, aufsetzen. Der Code, der vor den 
eigentlichen Tests geschrieben werden muss, hält sich also in Grenzen. Die
Effizienz mit der ein Entwickler Tests auswerten kann ist hingegen nicht 
optimal. Bei mehreren Fehlern und langem \gls{stacktrace} wird der Output 
schnell für das Terminal unübersichtlich. Ein farbiger Output 
würde hier 
ein wenig Abhilfe schaffen.

\lstinline{unittest} bietet einiges an Funktionalität zum Schreiben von Tests.
Durch die verschiedenen \lstinline{assert} Methoden, die bereit gestellt werden,
ist es möglich übersichtliche und lesbare Tests zu schreiben, die dennoch Ihre
Funktion erfüllen. Die gebotene Funktionalität sollte für die meisten Nutzer, 
abgesehen von ein paar Edge cases, bei denen spezielle Anforderungen  gestellt 
sind, ausreichend sein.

\lstinline{unittest} verfügt über einige interessante Erweiterungen, welche das 
Tools mit neuen Funktionalitäten auffrischen. Die folgende Auflistung zeigt ein 
paar der bekannteren Erweiterungen: 
\href{https://github.com/nose-devs/nose}{nose}\footnote{https://github.com/nose-devs/nose}(veraltet),
\href{https://github.com/nose-devs/nose2}{nose2}\footnote{https://github.com/nose-devs/nose2},
\href{https://github.com/twisted/twisted}{twisted}\footnote{https://github.com/twisted/twisted}
und
\href{https://launchpad.net/testtools}{testtools}\footnote{https://launchpad.net/testtools}.