% !TeX root = ../bachelor.tex
\subsubsection{unittest}\label{python-tools:unittest}

Das von JUnit inspirierte (\cite{docs.python:unittest}) Tool \lstinline|unittest|, ist bestand
der Python Standardbibliothek und bietet seit jeher seinen Nutzern ein umfangreiches
Repertoire an Funktionen zum testen von Python Code.
\noindent
Die Funktionen von unittest lassen sich unter folgenden Punkten beschreiben:
\begin{itemize}
    \item \Gls{fixture}, zum präparieren der Tests.
    \item Test Fälle, zum gliedern einzelner Tests.
    \item Testumgebungen, zum gliedern von zusammengehörigen Tests.
    \item Test runner, zum ausführen von Testumgebungen oder Test Fällen.
\end{itemize}
\noindent
Mithilfe der genannten Punkte ist es dem Entwickler möglich eine Stabile Test Umgebung
auf zu bauen. Jedoch bietet \lstinline{unittest} alleine nicht alles um TDD betreiben zu können.

Bei TDD werden zuerst die Tests und dann die Funktionalitäten geschrieben, daher
muss es möglich sein andere Module(units) zu \gls{mock}en auf denen ein Test basiert.
Durch die \Glspl{fixture} ist es bereits möglich den Test oder die Tests so vor zu
bereiten, dass diese funktionieren, jedoch bieten \Glspl{mock} einfachere und
schnellere Möglichkeiten Funktionen, Methoden, Klassen usw. zu imitieren.

Jedoch gibt es in der STDLIB eine Erweiterung zu unittest mit dem Namen
\lstinline|unittest.mock| welche unter eben diesem importiert werden kann um die 
\gls{mock}ing Funktionalität zu bekommen. Diese Erweiterung, auch als \lstinline|submodule|
bezeichnet ist Teil der STDLIB seit Python 3.3 wie in PEP417 von \cite{python.org:PEP417}
definiert wird.

Das Tool bietet des weiteren einen CLI, mit welchem es dem Benutzer möglich ist
seine Tests gebündelt aus zu führen und aus zu werten. Mit dem CLI ist es auch
auch möglich automatisch Tests in einem Ordner zu "`entdecken"'(discover) und
aus zu führen. Dadurch ist es sehr leicht neue Tests in ein bestehendes Test
System ein zu führen und diese ohne Veränderungen am bestehenden System aus zu
führen.

Im Aspekt Anwendbarkeit bietet \lstinline|unittest| alles um als Tool für TDD in
Frage zu kommen, jedoch nur unter Einbezug der Erweiterung \lstinline|unittest.mock|.
\newline
\\
Mit Hilfe von unittest lässt sich sehr einfach und schnell ein Test schreiben,
so würde das der Code aus Listing \ref{listing:unittest:example} bereits unsere 
selbst geschriebene quadrat-Funktion aus Listing \ref{listing:base:my_module} testen,
einmal mit unserem selbst berechneten Wert und einmal gegen den wert der quadrat-Funktion
aus der STDLIB.

Mit \lstinline{self.assertEqual} wird überprüft ob der erste Wert dem zweiten Wert
gleicht. Im ersten \lstinline{assert} wird auf einen im Kopf ausgerechneten Wert
geprüft und im zweiten wird die von Python gegebene Methode zum überprüfen verwendet.

Die basis-Funktionalität von unittest ist schnell verstanden und setzt sich
eigentlich nur aus \lstinline|self.assert{irgendwas}(...)| zusammen. Hat man die
richtige assert Funktion gefunden lässt sich eigentlich jede unabhängige
Funktion testen.

Möchte man allerdings fortgeschrittenere Tests schreiben so muss man sich der
Dokumentation bedienen, welche unter \url{https://docs.python.org/3/library/unittest.html}
zu finden ist. Würde man die Seite als PDF herunterladen so wären dies 58
Seiten Fließtext. Möchte man nun zum Beispiel vor den Tests etwas vorbereiten
oder präparieren so lässt sich mit \lstinline|setUp()| und \lstinline|tearDown()| dies realisieren,
diese zwei Methoden überschreiben die Methoden aus \lstinline|unittest.TestCase| und
werden vor jeder Funktion ausgeführt. Das Gleiche gibt es auch für den Test Fall, bei dem
\lstinline|setUp()| und \lstinline|tearDown()| allerdings nur beim eintritt der Klasse und beim
austritt ausgeführt werden. Diese Methoden sind die sogenannten \Glspl{fixture}.

Die folgenden Listings \ref{listing:unittest:example_output_success} und
\ref{listing:unittest:example_output_failure} zeigen wie ein Erfolgreicher Test
und wie ein Misslungener Test aussehen.

In Beiden Listings ist gut zu erkennen ob und was schief gegangen ist bei einem Test.
Der erfolgreiche Test zeigt dem Nutzer sofort wie viele Test in wie vielen Sekunden gelaufen
sind, und ist dies erwünscht kann der Nutzer mit \lstinline|--verbose| sich noch mehr
Informationen anzeigen lassen.

Bei misslungenen Tests ist im Output immer der \gls{stacktrace} abgebildet um so den Fehler
bis zur Wurzel zurück verfolgen zu können. Auch der Fehler selbst wird in den Output geschrieben,
wie in Zeile 8 in Listing \ref{listing:unittest:example_output_failure} zu erkennen ist. Am
Ende wird auch noch einmal angezeigt wie viele Tests schief gelaufen sind.

Die Effizient Tests aus zu werten ist also sehr hoch, jedoch wird der Output bei Mehreren Fehlern
und langem \gls{stacktrace} sehr schnell sehr unübersichtlich für das Terminal. Ein Farbiger Output
würde hier Abhilfe schaffen.

Um die Komplexität von \lstinline{unittest} dar zu stellen wurde mithilfe des in Listing \ref{listing:base:my_module} definierten Codes ein Fortgeschrittener Test
geschrieben, welcher die Basis Features von \lstinline{unittest} umfasst. Dieser Code
befindet sich in Listing \ref{listing:unittest:advanced}.
\newline
\\
Unittest verfügt über einige interessante Erweiterungen, welche das Tools mit neuen Funktionalitäten auffrischen.
Die folgende Auflistung zeigt ein paar dieser Erweiterungen.
\begin{itemize}
    \item \href{https://github.com/nose-devs/nose}{nose}\footnote{https://github.com/nose-devs/nose}(veraltet)
    \item \href{https://github.com/nose-devs/nose2}{nose2}\footnote{https://github.com/nose-devs/nose2}
    \item \href{https://github.com/twisted/twisted}{twisted}\footnote{https://github.com/twisted/twisted}
    \item \href{https://launchpad.net/testtools}{testtools}\footnote{https://launchpad.net/testtools}
\end{itemize}