% !TeX root = ../bachelor.tex
\section*{Einleitung}\label{einleitung}
\addcontentsline{toc}{section}{Einleitung}

Die Ansprüche an Software sind in den letzten Jahren immer weiter gestiegen.
Dies liegt vor allem an der Reichweite, die Software heute hat. So besitzen
im Jahr 2018 bereits circa 66\% aller Menschen ein Smartphone
(\cite{FraukeSchobelt:Smartphone}) und im Arbeitsleben ist ein PC meist nicht
mehr weg zu denken. Die dadurch benötigte Software muss bestimmte
Anforderungen, welche die Nutzer an diese stellen, gewährleisten und
sicherstellen. Mit steigenden Nutzerzahlen, steigen oft auch die Anforderungen
an eine Software, sowie die gefundenen \Glspl{bug} in dieser.

Im weltweiten Markt gibt es viele große Unternehmen, die gegenseitig um ihre
Nutzer kämpfen. Die Anwender präferieren meist denjenigen Anbieter, welcher die
bessere Software bietet. An dieser Stelle ist der Software-Architekt einer
jeden Anwendung gefragt, welche Technologie eingesetzt wird um die qualitativen
Anforderungen an diese zu gewährleisten. Eine dieser Technologien ist das
Test-driven development (TDD), welches sich durch eine hohe resultierende
Testabdeckung ihren Namen gemacht hat. Bei dieser Methode werden zuerst die
Tests und dann der Code geschrieben, um so später sicher stellen zu können,
dass die Software alle Anforderungen erfüllt und keine \Glspl{bug} enthält.

Gerade zum Release einer neuen Software ist es für Unternehmen schwer, Nutzer
zu akquirieren. Andere Anwendungen, die bereits auf dem Markt sind, haben
schon viele ihrer \Glspl{bug} gefixed und verfügen über eine durch den
Anwender ausgiebig getestete Software. Um eine neue Applikation auf diesen
Markt zu bringen, ist es erforderlich, dass diese zum Start ohne Fehler läuft.
Dies kann mit Test-driven development sicher gestellt werden.

Python wird heutzutage von vielen Entwicklern immer häufiger eingesetzt
und ist Mittlerweile die am zweit meisten beliebte und am vierthäufigsten
genutzte Sprache weltweit (\cite{stackoverflow:2019}).

Aufgrund der zunehmenden Beliebtheit, sowie meiner eigenen Vorliebe für die
Programmiersprache Python (auch \Gls{Pythonista} genannt) habe ich
mich dazu entschieden meine Bachelorarbeit dieser zu widmen. Da ich beginnend
mit dem Praxissemester und danach als Werkstudent in einer Firma an der
Entwicklung der hausinternen Test-Umgebung mitgewirkt habe, wurde ich immer
interessierter an dem Thema Testen. Durch die Beschäftigung mit Tests in meiner
Freizeit bin ich auf das Test-driven development gestoßen und war
sofort überzeugt, dass dies eine Technologie ist, die ich anwenden möchte. Doch
leider war die Auswahl der Tools, welche für Test-driven development in Frage
kamen so hoch, das ich Angst hatte etwas verpassen zu können, da ich ein Tool
einem anderen vorzog. Beim Analysieren der Tools viel mir auf, dass viele der
Tools veraltet oder ungeeignet waren, was einen Einsatz für Test-driven
development ausschloss. Die übrig gebliebenen Tools waren so umfangreich, dass
ich das Projekt erst einmal zur Seite legte.

Die Bachelorarbeit erschien mir als der passende Rahmen für die intellektuelle
und praktische Beschäftigung mit dieser Thematik. Zum einen konnte ich so meine
Privaten Pläne diesbezüglich verwirklichen und zum anderen erschien mir das
Thema relevant für meine späteren Beruflichen Perspektiven.

Das Ziel dieser Arbeit ist es, verschiedene aktuelle Test-Tools der
Programmiersprache Python auf ihre Tauglichkeit für den Einsatz von Test-driven
development zu untersuchen und zu vergleichen. Dabei wird zuerst jedes Tool und
seine Funktionalitäten beschrieben. Im Anschluss wird ein jedes auf die
verschiedene Anforderungen untersucht, welche es erfüllen muss um für
Test-driven development in Frage zu kommen. Dies geschieht anhand eines
erstellten Codebeispiels, welches gewisse Funktionalität liefern sollte.
Dieses Beispiel wird mithilfe der verschiedenen Tools auf die gleiche Art
getestet. Die dabei entstehenden Komplikationen, sowie der benötigte Aufwand
zeigen auf, welche Tools für die Analyse benötigt werden.

Die Arbeit soll aufzeigen welche Kombination der verschiedenen Tools die
optimale Verbindung bieten um Test-driven development durch zu führen. Dabei
erwarte ich mir tiefere Einblicke in das Test-driven development zu erhalten,
um heraus zu finden ob diese Technologie für mich und für meine weiteren
Arbeiten relevant ist.

Dazu wird in Kapitel \ref{definition} zunächst geklärt, welche Eigenschaften
die Programmiersprache Python hat, sowie die Definition von Test-driven
development. Die genutzte Methodik wird in Kapitel \ref{methodik} beschrieben
und anschließend in Kapitel \ref{python-tools} angewendet. Die Analyse ist in
vier Unterpunkte gegliedert, in Punkt \ref{python-tools:stdlib} werden zunächst
die Tools der Standardbibliothek analysiert. Diese sind \lstinline{unittest}
und \lstinline{doctest}. Zusammengefasst unter Punkt \ref{python-tools:extlib}
sind die Tools, welche nicht in der Standardbibliothek zu finden sind. Diese
setzen sich aus \lstinline{pytest}, \lstinline{stubble}, \lstinline{mocktest},
\lstinline{flexmock}, \lstinline{doublex} und \lstinline{hypothesis}.
Darauffolgend werden unter Punkt \ref{python-tools:vergleich} die analysierten
Tool verglichen um unter Punkt \ref{python-tools:kombination} mögliche
Kombinationen dieser auf zu Zeigen. Bevor in Kapitel \ref{fazit} das Fazit
gezogen wird, werden die Ergebnisse der Arbeit in Kapitel \ref{diskussion}
diskutiert.