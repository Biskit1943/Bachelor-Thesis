% !TeX root = ../bachelor.tex
\subsubsection{doctest}\label{python-tools:doctest}

"`Das doctest Modul sucht nach Textstücken, die wie interaktive Python-Sitzungen aussehen, und führt diese Sitzungen dann aus, um sicherzustellen, dass sie genau wie gezeigt funktionieren."' (\cite{docs.python:doctest}). Diese Textstücke müssen sich in Kommentaren befinden, da sie
sonst von Python als Code interpretiert werden.

Wie \nameref{python-tools:unittest}, ist auch doctest in der STDLIB, wodurch keine externen
Abhängigkeiten geladen werden müssen.

Doctest bietet dem Nutzer eine Möglichkeit mithilfe von einem Test den Code zu dokumentieren.
Da doctest lediglich Code ausführt wie in einer interaktiven Python-Sitzung, ist auch nur
das möglich aus zu führen was im Code geschrieben ist. Test Fälle oder gar \Glspl{mock} sind 
hierbei nicht möglich, \Glspl{fixture} hingegen sind teilweise realisierbar in Form von Code der
vor dem Test ausgeführt wird.

Doctest selbst nutzt keine Assert Funktionen oder Methoden, stattdessen wird der Output eines
Befehls überprüft. Dabei spielt der Typ auch eine Rolle, so ergibt eine Funktion
\lstinline{return_3(s=None, i=None)} bei \lstinline{s=True} eine \lstinline{'3'} und bei
\lstinline{i=True} eine \lstinline{3}. Dieses Beispiel ist in Listing \ref{listing:doctest:example}
zu sehen, um einen Output zu erzeugen wurde noch ein fehlerhafter Test hinzu gefügt. Dieser Output
ist in Listing \ref{listing:doctest:example_output} zu sehen.

Möchte der Entwickler einen etwas größeren Test schreiben, so kann er sich einer Funktionalität
von doctest bedienen, die es Ihm ermöglicht Tests in eine externe Datei zu verlagern. Dabei
wird die Datei als ein \Gls{docstring} behandelt, wodurch sie keine \lstinline{""""} benötigt.
Um jedoch Code in diese Datei ausführen zu können muss das zu testende Modul importiert werden.

Das Listing \ref{listing:doctest:advanced} zeigt den Code aus \ref{listing:base:my_module} mit
\Glspl{docstring} versehen. Dieser Test wurde mithilfe der \lstinline{main} Funktion von Python
ausgeführt. Der in Listing \ref{listing:doctest:advanced_text} ausgelagerte Test, wurde mithilfe
des CLIs von doctest ausgeführt. Da beide Test keine Fehler werfen existiert auch kein Output
für diese Tests.

Die einzige Möglichkeit hier einen Output zu bekommen wäre mit \lstinline{-v} im CLI oder
\lstinline{verbose=True} im Funktionsaufruf. Die dort dargestellten Informationen sind
lediglich welche Kommandos ausgeführt wurden, was erwartet wurde und das, dass erwartete
eingetroffen ist. Ein kleines Beispiel ist in Listing \ref{listing:doctest:output} zu sehen.

Da doctest selbst keine Test Fälle unterstützt besitzt \nameref{python-tools:unittest} eine
Integration für doctest, diese ermöglicht es Tests aus Kommentaren sowie Textdateien in
unittest zu integrieren und zu gliedern.

Da sich mit Doctest leider Funktionen nicht präparieren lassen, ist dieses Test Modul für die
alleinige Anwendung in TDD nicht nutzbar. Das Tool bietet keine Möglichkeiten Objekte zu \gls{mock}en
wodurch Tests an nicht implementieren Methoden und Funktionen scheitern. Auch \Glspl{fixture} sind
nur bedingt realisierbar und benötigen viel Code der dupliziert werden muss, da dieser vor und nach
jedem Test geschrieben werden muss.