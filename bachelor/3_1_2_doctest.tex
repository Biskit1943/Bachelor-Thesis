% !TeX root = ../bachelor.tex
\subsubsection{doctest}\label{python-tools:doctest}

"`Das doctest Modul sucht nach Textstücken, die wie interaktive Python-Sitzungen
aussehen, und führt diese Sitzungen dann aus, um sicherzustellen, dass sie genau
wie gezeigt funktionieren."'\footnote{Übersetzt aus dem Englischen}
(\cite{docs.python:doctest}). Diese Textstücke müssen sich in Kommentaren
befinden, da sie sonst von Python als Code interpretiert werden.

\lstinline{doctest} bietet dem Nutzer eine Möglichkeit mit Hilfe von einem Test
den Code zu dokumentieren. Da \lstinline{doctest} lediglich Code, wie in einer
interaktiven Python-Sitzung ausführt, werden die Möglichkeiten der Ausführung
auf das, was im Code geschrieben ist, beschränkt. Testfälle oder gar
\Glspl{mock} sind hierbei nur bedingt realisierbar. \Glspl{fixture} hingegen
sind in Form von Code, der vor dem Test ausgeführt wird, teilweise realisierbar.

\lstinline{doctest} selbst nutzt keine \lstinline{assert} Funktionen oder
-Methoden, stattdessen wird der Output eines Befehls überprüft. So ergibt eine
Funktion \lstinline{return_3(s=None, i=None)} bei \lstinline{s=True} eine
\lstinline{'3'} und bei \lstinline{i=True} eine \lstinline{3}. Dieses Beispiel 
ist in Listing \ref{listing:doctest:example} zu sehen. Um einen Output zu 
erzeugen wurde noch ein fehlerhafter Test hinzu gefügt. Dieser Output ist in 
Listing \ref{listing:doctest:example_output} zu sehen.

Möchte der Entwickler einen etwas größeren Test schreiben, so kann er sich einer
Funktionalität von \lstinline{doctest} bedienen, die es ihm ermöglicht Tests in
eine externe Datei zu verlagern. Dabei wird die Datei als ein \Gls{docstring}
behandelt, wodurch sie keine \lstinline{"""} (drei Interpunktionszeichen) 
benötigt. Um jedoch Code in diese Datei ausführen zu können muss das zu 
testende Modul importiert werden.

Durch das Schreiben der Tests in den \Glspl{docstring} werden die Module, in
denen längere Tests geschrieben werden, schnell sehr unübersichtlich und lang.
Zwar wird durch externe Textdateien Abhilfe geschaffen, dennoch sind zu lange 
und komplexe doctests schwer zu lesen und nach zu vollziehen.
\newline

Das Listing \ref{listing:doctest:advanced} zeigt den Code aus Listing
\ref{listing:base:my_module} versehen mit \Glspl{docstring}. Dieser Test wurde
mit Hilfe der \lstinline{main} Funktion von Python ausgeführt. Der in Listing
\ref{listing:doctest:advanced_text} ausgelagerte Test wurde mit Hilfe des CLIs
von \lstinline{doctest} ausgeführt. Da beide Tests keine Fehler werfen, 
existiert auch kein Output für diese Tests.
Die einzige Möglichkeit hier einen Output zu bekommen wäre mit \lstinline{-v} im
CLI oder \lstinline{verbose=True} im Funktionsaufruf. Die dort dargestellten
Informationen zeigen lediglich, welche Kommandos ausgeführt wurden, was erwartet
wurde und dass, das Erwartete eingetroffen ist. Ein kleines Beispiel ist in
Listing \ref{listing:doctest:output} zu sehen.

Da \lstinline{doctest} selbst keine Testfälle unterstützt, besitzt
\nameref{python-tools:unittest} eine Integration für \lstinline{doctest}.
Diese ermöglicht es Tests aus Kommentaren, sowie Textdateien in
\lstinline{unittest} zu integrieren und zu gliedern. Wie
\nameref{python-tools:unittest}, ist auch doctest in der STDLIB, wodurch keine
externen Abhängigkeiten geladen werden müssen. Für die Analyse wird diese 
Integration allerdings nicht in Betracht gezogen, da \lstinline{doctest} als 
Modul selbst analysiert werden soll und nicht die Funktionalität die von 
\lstinline{unittest} geboten wird.
\newline

Im Bezug auf die Anwendbarkeit von \lstinline{doctest} lässt sich sagen, dass
nicht alles zur Verfügung steht um TDD zu betreiben. Tests können nicht vor der
eigentlichen Funktionalität geschrieben werden. \Glspl{fixture} und 
\Glspl{mock} werden nicht unterstützt.

Die Anforderungen für \lstinline{doctest} sind sehr gering. Das Testen geschieht
mit Hilfe der interaktiven Python-Sitzung, welche jedem Pythonentwickler
bekannt ist. Die Tests selbst sind Aufrufe der geschriebenen Funktionen und
Methoden, sowie ein Abgleich des Outputs mit dem erwarteten Wert. Sollten
allerdings \Glspl{fixture} gewünscht sein, so ist dies nur mit Vorarbeit
möglich, da diese jedes mal an der benötigten Stelle geschrieben werden müssen.
Der Output von \lstinline{doctest} gleicht dem von \lstinline{unittest} und
verfügt über die selben Schwächen. Bei vielen Fehlern und langem
\gls{stacktrace} wird der Output im Terminal schwer lesbar. Die Effizienz mit
der \lstinline{doctest} genutzt werden kann, ist demnach mittelmäßig. Zum einen
ist es einfach Tests zu schreiben und zum anderen ist viel Code nötig.
Funktionalität, wie das \gls{mock}en, fehlen.

\lstinline{doctest} selbst bietet wenig Komplexität von sich aus. Stattdessen
ist die Komplexität, die möglich ist vom Entwickler abhängig, da
dieser die Tests komplett selber schreiben muss. Je nach zu testender
Funktion/Methode kann dies abhängig des Test Aufwands schnell unübersichtlich
werden, wenn viel Code abseits des eigentlichen Tests benötigt wird.
\lstinline{doctest} stellt dem Entwickler lediglich ein paar Möglichkeiten
Optionen zu aktivieren, die den Test beeinflussen können. Diese sind in der
Dokumentation unter dem Punkt \lstinline{Option Flags} zu
finden\footnote{https://docs.python.org/3.7/library/doctest.html\#option-flags}.
\\
Als Erweiterung bestet lediglich die Integration in
\lstinline{unittest}\footcite{docs.python:doctest} sowie
\lstinline{pytest}\footcite{docs.pytest.org:4.4}.
\newline

Da sich mit \lstinline{doctest} Funktionen nicht präparieren lassen, ist dieses
Testmodul für die alleinige Anwendung in TDD nicht nutzbar. Das Tool bietet
keine Möglichkeiten Objekte zu \gls{mock}en, wodurch Tests an nicht
implementierbaren Methoden und Funktionen scheitern. Auch \Glspl{fixture} sind
nur bedingt realisierbar und benötigen viel Code, der dupliziert werden muss, da
dieser vor und nach jedem Test geschrieben werden muss.