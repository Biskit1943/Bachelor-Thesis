% !TeX root = ../bachelor.tex
\section{Fazit}\label{fazit}

Durch das testen der verschiedenen Tool der Sprache Python, hat sich diese
Arbeit mit dem Vergleich sowie der Wertung dieser im Bezug auf TDD auseinander
gesetzt. Das Ziel dabei war es ein oder mehrere Tool(s) zu finden, welche für
den allgemeinen Einsatz von Python und TDD am besten geeignet sind. Dieses
Ziel wurde mithilfe der gewählten Methoden erreicht und ergab, dass vier Tools
gemeinsam eine sehr stabile Basis für die Anwendung von TDD bieten. Diese vier
Tools sind: \lstinline{pytest}, \lstinline{mocktest}, \lstinline{hypothesis} und
\lstinline{doctest}.

Die dabei verwendete Testmethode ermöglicht zwar eine hohe Validität, jedoch
wäre es möglich die Reliabilität noch weiter zu steigern, indem die Testfälle
mehr Funktionen abdecken. Jedoch war die gewählte Methode für diese Arbeit die
Richtige und führt zu verwendbaren Ergebnissen.

Bei der Analyse der verschiedenen Tools ergab sich, dass das anwenden von TDD 
anstrengender ist als zunächst erwartet. Um die Testmethode für diese Arbeit zu 
erstellen musste eine "`Software"' erstellt werden, welche diverse 
Anforderungen erfüllen muss, welche wiederum die Funktionen eines Tools 
spiegeln. Dieser erste schritt gleicht dem ersten schritt vor der Anwendung von 
TDD, da die Architektur der Software entworfen werde musste. Dies erwies sich 
als schwieriger als zunächst erwartet, da die Methodik möglichst komplex sein 
sollte.

Diese Arbeit hat mir im Bezug auf TDD, Testen und Python viel neues bei
gebracht. So würde ich zwar TDD nicht unbedingt einsetzen wollen, dies beruht
aber nicht darauf, dass ich es für ineffektiv halte, sondern vielmehr aus dem
Grund, dass es ein sehr hoher Aufwand ist, der sich erst sehr spät bezahlbar
macht. Ich selbst würde viel lieber einen Ansatz nutzen bei dem Tests neben dem 
Code geschrieben werden, dies erspart viel Zeit und würde zu einer ähnlich 
hohen, wenn auch nicht so perfekten Testabdeckung führen.

Für mich wäre TDD nach dieser Arbeit zwar eine Methode die ich definitiv einmal 
in einem Professionellen Umfeld anwenden würde, jedoch aber eben auch nur dort. 
Meiner Meinung nach ist die Planung bei TDD das wichtigste und sorgt dafür, 
dass die Entwickler auch Lust und Spaß an der Arbeit haben.

Dennoch finde ich sind Tests eine wichtige Sache in der Software Entwicklung 
und ich werde dank dieser Arbeit über genug KnowHow verfügen um Tests mit 
Python schreiben zu können. Dabei werde ich auch immer das TDD im Hinterkopf 
behalten und bestimmt ein paar seiner Methoden anwenden.
