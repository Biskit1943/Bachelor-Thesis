% !TeX root = ../bachelor.tex
\subsection{Kombinierung von Tools}\label{python-tools:kombination}
Um eine verbesserte Funktionalität zu bekommen, ist es ratsam verschiedene
Tools miteinander zu kombinieren. Die in diesem Kapitel aufgezeigten
Möglichkeiten zur Kombinierung ergeben sich aus den analysierten Features der
jeweiligen Tools. Dabei wird zuerst geprüft ob die beiden, am besten geeigneten
Tools aus unit- und \gls{mock}-testing Tools miteinander verwendet werden
können, um so einen Mehrwert zu erhalten.

Bei den unit-testing Tools ist \lstinline{doctest} das am wenigsten geeignete
Tool für die alleinige Anwendung mit TDD. Dennoch bietet das Tool Features, die
für die Anwendung von TDD nützlich sind. So kann \lstinline{doctest} als
testende Dokumentation gesehen werden. Manche Funktionen oder Methoden sind in
ihrem Funktionsumfang so gering, dass ein ganzer Test als zu viel erscheint.
Bei diesen Funktionen oder Methoden kann beispielsweise ein
\lstinline{doctest} geschrieben werden, der diese kleine Funktionalität prüft.
Des-weiteren kann \lstinline{doctest} als ergänzendes Tool für die
Dokumentation des Codes gesehen werden, denn die Dokumentation ist ein
wichtiger Bestandteil eines Projekts (\cite{python.org:PEP8}).
\lstinline{doctests} können dafür sorgen, dass die Dokumentation besser
verstanden wird und eindeutiger ist. Aus diesen Gründen kann
\lstinline{doctest} immer als ergänzendes Tool in jede Entwicklung aufgenommen
werden.

Da das Tool \lstinline{hypothesis} alleiniger Vertreter seiner Kategorie, der 
\gls{fuzz} Tools ist, ist es das empfohlene Tool, wenn \gls{fuzz} benötigt wird.
Da \lstinline{hypothesis} mit jedem anderen Tool verwendet werden kann,
empfiehlt es sich, dieses so oft wie es möglich ist ein zu setzen um eine
bessere Testabdeckung zu erreichen.

Demnach ist es ratsam sowohl \lstinline{doctest} als auch 
\lstinline{hypothesis} immer ein zu setzen wo es möglich ist.
\newline

Grundsätzlich lässt sich ein jedes unit-testing Tool mit jedem  
\gls{mock}-testing Tool kombinieren. Der dadurch erhaltene Mehrwert ist meist 
beachtlich gegenüber der Verwendung der Tools alleine. Es ist jedem Entwickler 
überlassen, welches Tool er mit welchem kombiniert, dennoch bietet es sich an, 
\lstinline{pytest} zusammen mit \lstinline{mocktest} zu verwenden. Wenn dazu 
noch \lstinline{doctest} und \lstinline{hypothesis} benutzt wird, erhält der 
Entwickler ein Toolset, das ihm alles an Funktionalität bietet, was er 
benötigt. Ist es erwünscht \lstinline{unittest} statt \lstinline{pytest} 
zu verwenden, so kann dies auch getan werden. Jedoch zeigt die Analyse, dass 
der Einsatz von \lstinline{pytest} dem Einsatz von \lstinline{unittest} vor zu 
ziehen ist.