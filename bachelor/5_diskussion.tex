% !TeX root = ../bachelor.tex
\section{Diskussion}\label{diskussion}
Um für die Anwendung von TDD die passenden Tools zu finden, wurden verschiedene
Tools der Programmiersprache Python analysiert und auf ihre Tauglichkeit im
Bezug auf TDD geprüft. Daraus entstand eine Empfehlung für vier Tools, die
gemeinsam für TDD genutzt werden können, um so optimale Funktionalität zum
Schreiben von Tests, sowie eine optimalen Testabdeckung zu garantieren. Die 
dabei erhaltenen Ergebnisse zeigen auf, dass ein Tool alleine zwar ausreichend 
für die Anwendung von TDD sein kann, jedoch ist die Verwendung von mehreren 
Tools zusammen für den Entwickler und für das Produkt besser.

Im Rahmen dieser Arbeit wurden lediglich Tools analysiert, deren letzte
Aktivität nicht weiter als ein Jahr zurück liegt. Tools, die eine Erweiterung zu
einem anderen Tool darstellen, wurden nicht behandelt, da dies den Rahmen der
Arbeit gesprengt hätte. Aus diesem Grund kann es sein, dass ein beliebtes oder
bekanntes Tool in dieser Arbeit nicht behandelt wurde. Das bedeutet nicht, dass
dieses Tool zwangsläufig nicht geeignet ist für TDD. Die Ergebnisse dieser
Arbeit sind also nicht repräsentativ für alle Tools die es für die
Programmiersprache Python gibt und würden eventuell anders ausfallen, wenn
andere Tools behandelt worden würden.

Im Bezug auf die Forschungsfrage, welches aktuelle Python Test-Tool das Beste 
ist, lässt sich Folgendes sagen: Viele Tools sind in ihrem Funktionsumfang sehr 
ähnlich, allerdings stellten sich \lstinline{pytest} bei den unit-testing-, 
\lstinline{mocktest} bei den mock-testing- und \lstinline{hypothesis} bei den 
fuzz-testing Tools als die beste Wahl heraus. Dennoch ergab sich, dass 
\lstinline{doctest} immer als ergänzendes Tool verwendet werden kann, egal 
welche anderen Tools verwendet werden.

Diese Arbeit hat sich jedoch lediglich mit einer begrenzten Anzahl von Tools 
beschäftigt und hat diese anhand eines spezifischen Beispiels getestet. Jeder 
Entwickler oder Software-Architekt sollte sich darüber informieren, welches 
Tool für den geplanten Anwendungszweck sinnvoll ist, da manche Tools 
vielleicht eine bessere Wahl, je nach Art der Anwendung, darstellen.