% !TeX root = ../bachelor.tex
\subsection{Vergleich der Tools}\label{python-tools:vergleich}

Die in Kapitel \ref{python-tools:stdlib} und \ref{python-tools:extlib}
analysierten Tools, werden anhand der Ergebnisse aus der Anwendung des
jeweiligen Tools in diesem Kapitel verglichen. Der Vergleich wird in unit- und
\gls{mock} Tools gegliedert. Die \gls{fuzz}-testing Tools werden hier
nicht verglichen, da nur ein Tool analysiert wurde. Am Ende dieses Kapitels 
wird für die jeweilige Kategorie (unit- und mock-testing Tools) eine Empfehlung 
ausgesprochen, die anhand der Ergebnisse des Vergleichs festgelegt wird. Anhand 
dieser Empfehlungen werden in Kapitel \ref{python-tools:kombination} 
verschiedene Möglichkeiten aufgezeigt, die Tools untereinander zu kombinieren 
um mehr Funktionalität für eine mögliche Anwendung zu bekommen.

\subsubsection{Unit-testing Tools}\label{python-tools:vergleich:unit}
Die unit-testing Tools belaufen sich auf drei Tools, \lstinline{unittest},
\lstinline{doctest} und \lstinline{pytest}. Die ersten beiden Tools sind beide
in der STDLIB, während \lstinline{pytest} mit sechs weiteren Paketen betrieben
werden muss. Während \lstinline{unittest} und \lstinline{pytest} sich sehr
ähnlich sind, ist \lstinline{doctest} vollkommen anders im Bezug auf Anwendung
und gegebener Funktionalität. \lstinline{doctest} unterstützt keine Testfälle
oder Testumgebungen, stattdessen wird im \gls{docstring} der jeweiligen
Funktion oder Methode der Test geschrieben. Zwar ist es möglich diesen in eine
externe Datei aus zu lagern um so einen zentralen Punkt zu schaffen, an dem sich
alle Tests befinden (Testumgebung/ Testfälle), jedoch ist hier die Datei das
Konstrukt, dass die Gliederung ermöglicht. Sowohl \lstinline{unittest} als auch
\lstinline{pytest} bieten hier die Möglichkeiten, Tests in Testumgebungen
(Klassen) und Testfällen (Funktionen/ Methoden) zu gliedern. \lstinline{pytest}
bietet zusätzlich auch noch die Möglichkeit diese Tests zu markieren um eine
noch feinere Gliederung zu schaffen oder um zwischen Testumgebungen und 
Testfälle zu kombinieren. Im Fall der Gliederung und Verwaltung von Tests sind 
hier sowohl \lstinline{pytest} als auch \lstinline{unittest}, 
\lstinline{doctest} überlegen. Lässt man die \lstinline{unittest} Integration 
für \lstinline{doctest} außen vor, so ist \lstinline{unittest} in diesem Fall 
der Gewinner. Wird allerdings die Anwendung der \lstinline{unittest} 
Integration von \lstinline{doctest} in Betracht gezogen, so sind 
\lstinline{unittest} und \lstinline{doctest} auf einem Level, was Gliederung 
und Verwaltung von Tests betrifft.

Ein Feature, das eine Gliederung voraussetzt, sind die Fixtures. Da
\lstinline{doctest} keine Gliederung bietet, sind Fixtures für dieses Tool
nicht als Feature integriert. Es ist zwar möglich Fixtures manuell zu
schreiben, jedoch ist der Aufwand, der dafür betrieben werden muss, zu hoch, als
dass es sich rentieren würde. Die \lstinline{unittest} Integration ermöglicht es
zwar Fixtures zu nutzen, jedoch wird diese Integration außen vor
gelassen. \lstinline{unittest} und \lstinline{pytest} hingegen unterstützen
Testfixtures mit ihren \lstinline{setUp()} und \lstinline{tearDown()} Methoden.
Beide Tools bieten die Möglichkeit die Fixtures sowohl für jedem Testfall, jede
Testumgebung oder jedes Modul aus zu führen. \lstinline{pytest} bietet
zusätzlich noch weitere Funktionalität, sowie fertige Fixtures die dem
Entwickler bei der Nutzung zur Verfügung stehen. Diese wurden in Kapitel
\ref{python-tools:pytest} beschrieben und sind in der 
\href{https://docs.pytest.org/en/latest/fixture.html}{Dokumentation}\footnote{\url{https://docs.pytest.org/en/latest/fixture.html}}
zu finden. Im Bezug auf Testfixtures ist also \lstinline{pytest} das Tool
mit den meisten Features, gefolgt von \lstinline{unittest}. \lstinline{doctest}
scheidet bei dieser Kategorie vollkommen aus, da die Funktionalität gänzlich
fehlt.

Als nächstes integriertes Feature der unit-testing Tools sind \Glspl{mock} und
\Glspl{stub} zu betrachten. Um für TDD in Frage zu kommen sollte ein
unit-testing Tool \Glspl{stub} und \Glspl{mock} unterstützen. Eine ausführliche
Erklärung dazu ist in Kapitel \ref{python-tools} zu finden. Wie in Kapitel
\ref{python-tools:doctest} aus der Analyse von \lstinline{doctest} hervorgeht,
gibt es weder die Möglichkeit einen \Gls{mock} noch einen \Gls{stub} auf zu
setzen, weshalb \lstinline{doctest} in dieser Kategorie als unbrauchbar gesehen
werden kann. \lstinline{unittest} als Modul selbst, besitzt keine \gls{mock}ing
Funktionalität, jedoch ist es möglich, dass Submodul \lstinline{unittest.mock}
dafür zu verwenden. \lstinline{unittest.mock} ist zwar eine Erweiterung in Form 
eines Submoduls, weshalb es zu \lstinline{unittest} dazu gehört und in diesem 
Vergleich beachtet wird. Damit unterstützt \lstinline{unittest} sowohl das 
\Gls{mock}en, als auch das \Gls{stub}en. \lstinline{pytest} hingegen verfügt in 
seiner Basisausführung lediglich über die Möglichkeit des \Gls{stub}ens, nicht 
jedoch über die Möglichkeit ein Objekt zu \gls{mock}en. In diesem Falls ist 
also \lstinline{unittest} der klare Gewinner, gefolgt von \lstinline{pytest}.
\newline

Im Bezug auf die Effizienz der unit-testing Tools lässt sich folgendes
feststellen. \lstinline{doctest} ist zwar für jeden Entwickler einfach und
verständlich anwendbar, jedoch ist die Vorarbeit um Tests auf zu setzen im
Vergleich zu \lstinline{unittest} und \lstinline{pytest} höher. Sowohl
\lstinline{unittest} als auch \lstinline{pytest} benötigen das gleiche Maß an
Vorarbeit um Tests auf zu setzen, wobei der Aufwand das jeweilige Tool ein zu
setzen bei \lstinline{unittest} etwas höher ist, da es so viele verschiedene
\lstinline{assert} Methoden zu merken gibt. Die Auswertung der Tests erfolgt
bei allen drei unit-testing Tools auf dem Terminal. \lstinline{doctest} und
\lstinline{unittest} haben beide einen Output der voneinander fast nicht zu
trennen ist. Beide Tools haben den Nachteil gegenüber \lstinline{pytest}, dass
ihr Output nicht farbig, sondern monochrom ist.  Aus diesem Grund ist 
\lstinline{pytest}, \lstinline{unittest} und \lstinline{doctest} im Bezug auf
effiziente Anwendung vor zu ziehen, wobei \lstinline{unittest} noch etwas 
besser als \lstinline{doctest} ist.
\newline

Der Aspekt der Komplexität ist in zwei Unterkategorien gegliedert. So wird zum 
Einen die Fülle an Funktionen, die geboten werden und Komplexität liefern, und
andererseits die Menge an Code, die zu schreiben ist und die Anwendung und
Wartung Komplexer gestaltet, betrachtet. Im Falle der ersten Kategorie, erweist 
sich \lstinline{doctest} als nicht all zu komplex. Fast die Gesamte gebotene
Funktionalität bezieht sich auf die Ausführung, jedoch nicht das Schreiben der
Tests, welches alleine durch die Fähigkeit Python zu schreiben bestimmt wird.
\lstinline{unittest} und \lstinline{pytest} hingegen bieten dem Entwickler
einiges an Funktionalität um Tests zu schreiben und aus zu führen. Auch hier
kann \lstinline{unittest} nur mithalten, wenn das Submodul
\lstinline{unittest.mock} mit in die Wertung aufgenommen wird, da dieses
weitere Funktionalität bietet. \lstinline{unittest} und \lstinline{pytest}
haben viele gemeinsame Features, wovon lediglich das \gls{mock}ing in
\lstinline{pytest} nicht ohne Erweiterungen verfügbar ist. Ansonsten bietet
\lstinline{pytest} zusätzliche Funktionalität um Tests feiner ein zu
stellen oder um edge cases ab zu decken. In dieser Kategorie ist
\lstinline{pytest} also etwas besser als \lstinline{unittest}. Beide Tools
sind \lstinline{doctest} bei Weitem im Bezug auf die gebotenen Funktionalitäten 
überlegen.

In der zweiten Kategorie, der Komplexität im Bezug auf Codelesbarkeit und
-umfang, erweisen sich \lstinline{unittest} und \lstinline{pytest} als 
gleichauf. Beide Tools haben einen ähnlichen Aufbau, mit den der Code 
geschrieben wird (vgl. Listing \ref{listing:unittest:advanced} und
\ref{listing:pytest:advanced}). Die Lesbarkeit ist bei beiden Tools gleich gut
und die Menge an Code, die benötigt wird, gleicht sich auch. Hier ist es dem
Entwickler überlassen zu entscheiden, welches Tool den schöneren Code schreibt.
\lstinline{doctest} hingegen produziert viel Code, der mit zunehmender
Funktionalität an Lesbarkeit verliert.Im Bezug auf die Komplexität der Tools
ist \lstinline{pytest} also knapp vor \lstinline{unittest}, gefolgt von
\lstinline{doctest}.
\newline

Als letztes ist die Erweiterbarkeit der unit-testing Tools zu vergleichen.
Beginnend mit \lstinline{doctest}, lässt sich sagen, dass es zwei sehr gute
Erweiterungen gibt, die das Tool erheblich verbessern. Diese Erweiterungen sind
die Integration in \lstinline{unittest} und \lstinline{pytest}. Erstere wird
von der Python Sprache selbst geboten und ist in \lstinline{unittest} verfügbar,
zweite wird von \lstinline{pytest} geboten und ist auch im Basis Paket
enthalten. Abgesehen von diesen zwei gibt es noch weitere Integrationen in
andere Toolsets, jedoch wurden diese hier wie in Kapitel \ref{methodik} 
beschrieben, nicht behandelt. Das Modul \lstinline{unittest} verfügt über 
einige Erweiterungen welche neue Funktionalität hinzu fügen oder alte 
Funktionalität verbessern. Das Tool mit den meisten Erweiterungen ist 
\lstinline{pytest}, welches derzeit 683 (Stand: 19. Juli 
2019\footnote{\url{http://plugincompat.herokuapp.com/}}) Erweiterungen
bietet, wovon die meisten sehr nützliche Funktionalität bieten.
\newline

Im Vergleich der unit-testing Tools ergibt sich also ein klarer Sieger:
\lstinline{pytest}. Das Tool ist in fast allen Analysierten Aspekten die beste
Wahl für den Einsatz mit TDD. Jedoch bietet \lstinline{unittest} für
Entwickler, die keine externen Abhängigkeiten möchten eine Alternative, die
durchaus mit \lstinline{pytest} mithalten kann.

\subsubsection{Mock-testing Tools}\label{python-tools:vergleich:mock}
Die analysierten \gls{mock}-testing Tools, sind \lstinline{stubble},
\lstinline{mocktest}, \lstinline{flexmock} und \lstinline{doublex}. Alle vier
bieten verschiedene Möglichkeiten \Glspl{mock} und/oder \Glspl{stub} ein zu
setzen. Da es sich hier um Tools handelt, deren Funktionalität sich um das
\Gls{mock}en von Objekten dreht, wird der erste analysierte Aspekt, die
Anwendbarkeit, auf die gebotene Auswahl an Funktionalität im Bezug auf
\Gls{mock}en und \Gls{stub}en betrachtet.

Das einzige Tool, das bei der Anwendung auf den zu testenden Code keine
Probleme mit sich brachte, ist \lstinline{mocktest}. Das Tool bietet im Aspekt 
der Anwendbarkeit sämtliche Features um effektiv \Glspl{mock} und \Glspl{stub} 
ein zu setzen. \lstinline{mocktest} bietet seinem Anwender eine deskriptive 
Schreibweise, mit der es problemlos möglich ist Tests zu schreiben. Des 
Weiteren wird eine Testisolation geboten, mit deren Hilfe es möglich ist, Tests 
besser voneinander ab zu schotten. Zusätzlich benötigt das Tool keine weiteren
Abhängigkeiten zu sich selbst. Direkt danach wäre \lstinline{stubble} das  
nächstbeste Tool im Aspekt der Anwendbarkeit. So bietet das Tool zwar alles 
um  es für TDD zu verwenden, jedoch ist das \Gls{stub}en einer nicht 
existierenden Methode nur unter Einschränkungen möglich. Die Hauptfeatures von 
\lstinline{stubble} sind dabei die vorgefertigten \Glspl{stub}, welche 
nützliche Features für den Entwickler bieten. Um einen \Gls{stub} zu erstellen 
muss die zu \gls{stub}ende Klasse mit \lstinline{stubble} neu geschrieben 
werden. Zusätzlich benötigt \lstinline{stubble} eine weitere Abhängigkeit zu 
sich selbst. An dritter Stelle ist das Tool \lstinline{flexmock} welches,
abgesehen vom ersetzen einer nicht geschriebenen Methode um TDD an zu 
wenden, vieles bietet (vgl. Listing \ref{listing:flexmock:exception}). Wäre 
dieses Defizit nicht, würde das Tool im Aspekt der Anwendbarkeit sehr weit oben 
stehen. Flexmock vereint die \Glspl{mock} und \Glspl{stub} in einer Klasse, 
wodurch die gesamte Funktionalität mit einem Objekt geboten wird. Die Schreib 
weise der Tests ist hierbei, wie bei \lstinline{mocktest}, deskriptiv. Auch die 
benötigten Abhängigkeiten von \lstinline{flexmock} sprechen für das Tool, da 
keine weiteren zu sich selbst benötigt werden. Als letztes Tool im Aspekt der
Anwendbarkeit ist \lstinline{doublex} zu betrachten. Das Tool bietet zwar
einiges an Funktionalität um zu \gls{mock}en oder zu \gls{stub}en, jedoch ist
die Anwendung dieser Features für TDD nicht nutzbar. Es werden vier
verschiedene Interfaces geboten, deren Funktionalität sich überschneidet und
die jeweils ein Objekt hervorbringen, das ein Duplikat vom Zielobjekt ist.
Die daraus resultierenden Anwendungszwecke bieten wenig Nutzbarkeit für TDD.
Hinzu kommt, dass \lstinline{doublex} drei weitere Abhängigkeiten benötigt und
anscheinend nicht mehr aktiv weiter entwickelt wird.

Demnach ist im Aspekt der Anwendbarkeit \lstinline{mocktest} also
\lstinline{stubble} und \lstinline{flexmock} vor zu ziehen. \lstinline{doublex}
fällt aufgrund dessen, dass es nicht für TDD geeignet ist und derzeit nicht
aktiv weiter entwickelt wird, komplett raus. Aus diesem Grund wird dieses Tool
im weiteren Verlauf des Vergleichs nicht mehr berücksichtigt.
\newline

Als nächstes ist die Effizienz der jeweiligen Tools zu betrachten. Dabei spielt
vor allem der benötigte Aufwand das Tool ein zu setzen eine Rolle. Nebenbei
wird dabei auch noch auf die benötigte Vorarbeit geachtet, die geleistet werden
muss um das Tool an zu wenden. Sowohl bei \lstinline{mocktest} als auch bei
\lstinline{flexmock} ist der zu betreibende Aufwand sehr gering. Beide Tools
haben ein ähnliches Interface mit dem deskriptiv beschrieben wird, was ein
Objekt zu machen hat. Der dabei entstehende Aufwand ist sehr gering, da sich
der Code so schreibt, wie ein Entwickler sich die Funktionalität vorstellt.
Beide Tools benötigen die gleiche Menge an Vorarbeit und sind demnach im Bezug
auf die Effizienz mit der sie angewendet werden können gleich auf. Lediglich
\lstinline{stubble} ist in diesem Aspekt etwas schlechter als die beiden
anderen Tools. So ist die \gls{mock}ing Funktionalität ausschließlich mit einem
vorgefertigten \Gls{stub} möglich, wodurch dieser erst verstanden werden muss,
bevor er effektiv eingesetzt werden kann. Zusätzlich muss jedes zu
\gls{stub}ende Objekt überschrieben werden, wodurch die benötigte Vorarbeit 
sehr hoch ausfällt.

Demnach sollte im Aspekt der effizienten Anwendung der \gls{mock}-testing Tools 
\lstinline{stubble} nicht verwendet werden. Es ist dem Entwickler überlassen ob 
er in diesem Aspekt das Interface von \lstinline{mocktest} oder 
\lstinline{flexmock} lieber hat.
\newline

Zuletzt ist die Komplexität der \gls{mock}-testing Tools zu betrachten. Dabei
wird zum einen auf die gestellten Funktionalitäten abseits der Basis
Funktionalität geachtet und zum anderen auf die Lesbarkeit des entstehenden
Codes des Tools. Auch hier sind \lstinline{mocktest} und \lstinline{flexmock}
auf einem Level, was die zusätzliche Funktionalität betrifft. Jedoch hat
\lstinline{mocktest} etwas mehr zusätzliche Funktionalität gegenüber
\lstinline{flexmock} geboten (vgl. Kapitel \ref{python-tools:mocktest} und
\ref{python-tools:flexmock}). \lstinline{stubble} bietet auch in diesem Aspekt
weniger als die beiden Tools. Abseits der Basisfunktionalität wird von
\lstinline{stubble} nichts weiter geboten. Im Bezug der Lesbarkeit sieht es
ähnlich aus. \lstinline{mocktest} und \lstinline{flexmock} befinden sich auf
einem Level, während \lstinline{stubble} durch viel Code an Lesbarkeit
verliert. Im Fall \lstinline{mocktest} gegen \lstinline{flexmock} ist es wieder
dem Anwender überlassen, welches der beiden Tools lesbarer ist als das andere.

Im Bezug auf Komplexität geben sich \lstinline{mocktest} und
\lstinline{flexmock} fast nichts, jedoch hat \lstinline{mocktest} durch die
zusätzliche Funktionalität hier die Nase vorne. \lstinline{stubble} bleibt auch
hier auf dem letzten Platz.
\newline

Im gesamten Vergleich ist \lstinline{mocktest} durch seine Features und sein
Interface an erster Stelle, wenn es um den Einsatz von \gls{mock}-testing Tools
geht. Der zweite Platz ist schwer zu wählen, da \lstinline{flexmock} mit seiner
fehlenden Funktionalität nicht implementierte Methoden hinter
\lstinline{stubble} liegt, das jedoch auch nur bedingt diese Anforderung
erfüllt. In allen anderen Aspekten ist \lstinline{flexmock} \lstinline{stubble}
voraus. Aus diesem Grund erhält \lstinline{flexmock} den zweiten Platz der
\gls{mock}-testing Tools und \lstinline{stubble} den dritten. Auf dem vierten
wäre \lstinline{doublex}, jedoch ist dieses Tool für die Anwendung von TDD
ungeeignet, weshalb es keine Platzierung erhält.