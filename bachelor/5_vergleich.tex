% !TeX root = ../bachelor.tex
\subsection{Vergleich der Tools}\label{vergleich}

Die in Kapitel \ref{python-tools:stdlib} und \ref{python-tools:extlib}
analysierten Tools, werden anhand der Ergebnisse aus der Anwendung des
jeweiligen Tools in diesem Kapitel verglichen. Der Vergleich wird in unit- und
\gls{mock} Tools gegliedert. Die \gls{fuzz}-testing Tools werden hier
nicht verglichen, da nur ein Tool analysiert wurde.
Am Ende dieses Kapitels wird für die jeweilige Kategorie eine Empfehlung 
ausgesprochen, die anhand der Ergebnisse des Vergleichs festgelegt wird. Anhand 
dieser Empfehlungen werden in Kapitel \ref{kombinierung} verschiedene 
Möglichkeiten aufgezeigt, die Tools untereinander zu kombinieren um mehr 
Funktionalität für eine mögliche Anwendung zu bekommen.

\paragraph{Unit-testing Tools}\label{vergleich:unit}\mbox{}
\newline
Die unit-testing Tools belaufen sich auf drei Tools, \lstinline{unittest},
\lstinline{doctest} und \lstinline{pytest}. Die ersten beiden Tools sind beide
in der STDLIB während \lstinline{pytest} mit sechs weiteren Paketen betrieben
werden muss. Während \lstinline{unittest} und \lstinline{pytest} sich sehr
ähnlich sind, ist \lstinline{doctest} vorkommen anders im Bezug auf Anwendung
und gegebener Funktionalität. \lstinline{doctest} unterstützt keine Testfälle
oder Testumgebungen, stattdessen wird im \gls{docstring} der jeweiligen
Funktion oder Methode der Test geschrieben. Zwar ist es möglich diesen in eine
externe Datei aus zu lagern um so einen Zentralen Punkt zu schaffen an dem sich
alle Tests befinden (Testumgebung/Testfälle), jedoch ist hier die Datei das
Konstrukt, dass die Gliederung ermöglicht. Sowohl \lstinline{unittest} als auch
\lstinline{pytest} bieten hier die Möglichkeiten Tests in Testumgebungen
(Klassen) und Testfällen (Funktionen/Methoden) zu gliedern. \lstinline{pytest}
bietet zusätzlich auch noch die Möglichkeit diese Tests zu markieren um eine
noch feinere Gliederung zu schaffen oder um zwischen Testumgebungen, Testfälle
zu kombinieren. Im Fall der Gliederung und Verwaltung von Tests ist hier
\lstinline{pytest} sowohl \lstinline{unittest} als auch \lstinline{doctest}
voraus. Lässt man die \lstinline{unittest} Integration für \lstinline{doctest}
außen vor, so ist \lstinline{unittest} in diesem Fall der Gewinner. Wird
allerdings die Anwendung der \lstinline{unittest} Integration von
\lstinline{doctest} in Betracht gezogen, so sind \lstinline{unittest} und
\lstinline{doctest} auf einem Level, was die Gliederung und Verwaltung von
Tests betrifft.

Ein Feature, dass eine Gliederung voraussetzt sind die Fixtures. Da
\lstinline{doctest} keine Gliederung bietet sind Fixtures für dieses Tool
nicht als Feature integriert. Es ist zwar möglich Fixtures manuell zu
schreiben, jedoch ist der aufwand, der dafür betrieben werden muss zu hoch, als
dass es sich rentieren würde. Die \lstinline{unittest} Integration ermöglicht es
zwar Fixtures zu nutzen, jedoch wird diese Integration hier wieder außen vor
gelassen. \lstinline{unittest} und \lstinline{pytest} hingegen unterstützen
Testfixtures mit ihren \lstinline{setUp()} und \lstinline{tearDown()} Methoden.
Beide Tools bieten die Möglichkeit die Fixtures sowohl für jedem Testfall, jede
Testumgebung oder jedes Modul aus zu führen. \lstinline{pytest} bietet
zusätzlich noch weitere Funktionalität, sowie fertige Fixtures die dem
Entwickler zur Nutzung zur Verfügung stehen, welche in Kapitel
\ref{python-tools:pytest} beschrieben wurden und in der
\href{https://docs.pytest.org/en/latest/fixture.html}{Dokumentation}\footnote{\url{https://docs.pytest.org/en/latest/fixture.html}}
zu finden sind. Im Bezug auf Testfixtures ist also \lstinline{pytest} das Tool
mit den meisten Features, gefolgt von \lstinline{unittest}. \lstinline{doctest}
scheidet bei dieser Kategorie vollkommen aus, da die Funktionalität gänzlich
fehlt.

Als nächstes integriertes Feature der unit-testing Tools sind \Glspl{mock} und
\Glspl{stub} zu betrachten. Um für TDD in Frage zu kommen sollte ein
unit-testing Tool \Glspl{stub} und \Glspl{mock} unterstützen, eine ausführliche
Erklärung dazu ist in Kapitel \ref{python-tools} zu finden. Wie in Kapitel
\ref{python-tools:doctest} aus der Analyse von \lstinline{doctest} hervorgeht,
gibt es weder die Möglichkeit einen \Gls{mock} noch einen \Gls{stub} auf zu
setzen, weshalb \lstinline{doctest} in dieser Kategorie als unbrauchbar gesehen
werden kann. \lstinline{unittest} als Modul selbst, besitzt keine \gls{mock}ing
Funktionalität, jedoch ist es möglich, dass Submodul \lstinline{unittest.mock}
dafür zu verwenden. \lstinline{unittest.mock} ist zwar eine Erweiterung, jedoch
in Form eines Submoduls, weshalb es zu \lstinline{unittest} dazu gehört und in
diesem Vergleich beachtet wird. Damit unterstützt \lstinline{unittest} sowohl
das \gls{mock}en als auch das \gls{stub}en. \lstinline{pytest} hingegen verfügt
in seiner Basisausführung lediglich über die Möglichkeit des \gls{stub}ens,
nicht jedoch über die Möglichkeit ein Objekt zu \gls{mock}en. In diesem Falls
ist also \lstinline{unittest} der klare Gewinner, gefolgt von
\lstinline{pytest}.

Im Bezug auf die Effizienz der unit-testing Tools lässt sich folgendes
feststellen. \lstinline{doctest} ist zwar für jeden Entwickler einfach und
verständlich anwendbar, jedoch ist die Vorarbeit um Tests auf zu setzen im
Vergleich zu \lstinline{unittest} und \lstinline{pytest} höher. Sowohl
\lstinline{unittest} als auch \lstinline{pytest} benötigen das gleiche Maß an
Vorarbeit um Tests auf zu setzen, wobei der Aufwand das jeweilige Tool ein zu
setzen bei \lstinline{unittest} etwas höher ist, da es so viele verschiedene
\lstinline{assert} Methoden zu merken gibt. Die Auswertung der Tests erfolgt
bei allen drei unit-testing Tools auf dem Terminal. \lstinline{doctest} und
\lstinline{unittest} haben beide einen Output der voneinander fast nicht zu
trennen ist. Beide Tools haben den Nachteil gegenüber \lstinline{pytest}, dass
ihr Output nicht farbig sonder Monochrom ist.  Aus diesem Grund ist 
\lstinline{pytest}, \lstinline{unittest} und \lstinline{doctest} im Bezug auf
Effiziente Anwendung vor zu ziehen, wobei \lstinline{unittest} noch etwas 
besser als \lstinline{doctest} ist.
\newline
\\
TODO: Komplexität und Erweiterbarkeit.
