% !TeX root = ../bachelor.tex
\subsection{Test-driven development}\label{einleitung:tdd}
Das Test-driven development (TDD) wurde durch Kent Beck richtig Bekannt. Dieser
beschreibt in seinem Buch "`Test Driven Development: By Example"' wie bei der
Anwendung von TDD vorgegangen werden soll. Demnach sieht der Entwicklungsprozess
folgendermaßen aus (\cite{beck:tdd}):

Zunächst sollte geklärt sein, welche Funktionen erfüllt werden müssen. Anhand
dieser beginnt dann die Anwendung von TDD. Dazu wird der Folgende Zyklus
beschrieben der sich immer und immer wiederholt bis alle Features implementiert
und getestet sind.
Der Erste schritt dazu ist das schreiben eines Tests, der neue Funktionalität
oder Verbesserung zu einer bestehenden Unit hinzufügt. Dieser Test sollte
möglichst kurz und aussagekräftig sein. Dazu muss der Entwickler die
Spezifikationen und Anforderungen des Features genau verstehen um den Test
wirklich effektiv schreiben zu können. Ein Entwickler, der die Anforderung
nicht ganz versteht, ist nicht in der Laage einen Test zu schreiben, welcher
alle Bereiche abdeckt. Der Entwickler, der den Test geschrieben hat, muss
jedoch nicht auch der Entwickler sein, der die dazu passende Funktionalität
implementiert.
Als nächstes werden alle Tests die bereits bestehen, sowie der neue
durchgeführt um zu sehen, ob der neue Test fehlschlägt. Dies bestätigt unter
anderem auch, dass der neue Test die anderen Tests nicht beeinflusst und auch
ohne neuen Code nicht bestanden wurde. Dadurch kann der Entwickler
ausschließen, dass ein Test immer Korrekt ist, wodurch das Vertrauen des
Entwicklers in den Test gesteigert wird.
Der nächste schritt ist die Implementierung des Codes, der getestet werden soll,
dabei ist unwichtig wie der Code ist geschrieben wird, das bedeutet, dass es
erlaubt ist "`schmutzigen"' Code zu produzieren, welcher so nicht akzeptiert
werden würde. Die Hauptsache dabei ist, dass der Test bestanden wird. Die Code
Qualität spielt hier keine Rolle, da sie in Punkt Fünf überarbeitet wird.
Der Entwickler darf in diesem Prozess auch keinen Weiteren Code hinzufügen, der
nicht notwendig ist um den Test zu bestehen. Dies verhinder, das Code
geschrieben wird, der keine zugehörigen Tests hat.
Ist die gesamte Logik implementiert, werden alle Test durchgeführt und
überprüft, ob alle Tests erfolgreich waren. Sollte dies nicht der Fall sein
muss Punkt Drei wiederholt werden, welcher die Implementierung der
Funktionalität ist.
Sind alle Tests bestanden wird im Finalen schritt der neue Code aufgeräumt.
Dieser Punkt ist sehr wichtig, da in Punkt Drei der Code nur Funktionieren muss
kann es sein, dass hier Code entstanden ist, der nicht den Qualitativen
Anforderungen entspricht. Sämtliche Objekte sollten einen aussagekräftigen
Namen erhalten und der Code sollte, sofern dies nötig ist an einen Ort verlegt
werden, der seiner Logischen Aufgabe entspricht. Duplikationen müssen entfernt
werden und nach jeder Aufräumaktion sollten die Tests noch einmal ausgeführt
durchlaufen, um zu verifizieren, dass alles noch so Funktioniert wie es soll.

Dieser Zyklus wird nun von vorne wiederholt, bis jedes Feature implementiert
ist und die Software als Fertig gilt. Die schritte, die dabei gemacht werden
sollen, sollten möglichst klein sein, um eine höhere Test Abdeckung zu
erreichen.
