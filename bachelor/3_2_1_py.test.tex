% !TeX root = ../bachelor.tex
\subsubsection{pytest}\label{python-tools:pytest}\mbox{}
\newline
Das am 04. August 2009 in Version 1.0.0 veröffentlichte Tool pytest (auch py.test genannt)
ist ein sehr umfangreiches und weit entwickeltes Tool. Seit 2009 wird das Tool stets
weiter entwickelt und vorangetrieben, wodurch es eine Menge an Features gewonnen hat.
\noindent
Die Basis Features von pytest sind folgende:
\begin{itemize}
    \item Simple \lstinline{assert} statements.
    \subitem Kein \lstinline{self.assert}
    \subitem Error Überprüfung mit \Gls{context}
    \item Informativer Output in Farbe
    \subitem Der gesamte Output kann angepasst werden
    \item Feature reiche \Glspl{fixture}
    \subitem Vordefinierte \Glspl{fixture} von pytest
    \subitem Geteilte \Glspl{fixture} unter Tests
    \subitem Globale \Glspl{fixture} zwischen Modulen
    \subitem Parametrisierung von Test als \Glspl{fixture}
    \item Überprüfung von \lstinline{stdout} und \lstinline{stderr}
    \item Gliedern von Test in Fällen
    \subitem ... durch Markierung
    \subitem ... durch Nodes (Auswahl der Modul Abhängigkeit)
    \subitem ... durch String Abgleich der Funktions-Namen
\end{itemize}
\noindent
Da pytest nicht in der STDLIB ist, muss es mit einem \Gls{paket} installiert werden.
Dabei werden für \lstinline{pytest 4.4.0} die in Listing \ref{listing:pytest:requirements}
gezeigten Abhängigkeiten installiert. Demnach benötigt pytest sechs externe Abhängigkeiten
zusätzlich zu sich selbst.
\newline
\\
Wie anhand der Basis Features erkennbar ist, bietet pytest einiges um Tests zu schreiben und
aus zu führen. So ist mit den gebotenen \Glspl{fixture} bereits eine Voraussetzung mehr als
erfüllt, da pytest viele verschiedene Arten bietet \Glspl{fixture} zu benutzen.
\Glspl{fixture} werden in pytest allerdings nicht mit \lstinline{setUp} und \lstinline{tearDown}
geschrieben, sondern werden mithilfe eines \glspl{decorator} markiert und den Test Funktionen
als Parameter übergeben. Um die \lstinline{setUp} und \lstinline{tearDown} funktionalität zu
bekommen muss lediglich das keyword \lstinline{yield} verwendet werden, die \Gls{fixture} wird
dann den Code bis zum \lstinline{yield} ausführen und nach der Funktion den Rest nach
\lstinline{yield} ausführen.
\newline
\\
Diese lassen sich durch zusätzliche Parameter weiter anpassen. Diese
Funktionen sind in Kapitel fünf nach \cite{docs.pytest.org:4.4} beschrieben.

Des weiteren bietet pytest auch \gls{mock}ing, so lässt sich beispielsweise mit
\lstinline{monkeypatch.setattr()} der return Wert einer Funktion ersetzen. Dies wird
in pytest automatisch am ende der Funktion rückgängig gemacht, wodurch der Entwickler
sich mehr auf das eigentliche Testen konzentrieren kann.
\newline
\\
Auch hier bietet pytest weitere Möglichkeiten \Glspl{mock} zu verwenden, dazu hat
\cite{docs.pytest.org:4.4} in Kapitel sieben einige Worte geschrieben.

Ein weiteres sehr praktisches Feature von pytest ist die Gliederung in Fälle. Dies ist bei pytest
sehr fein einstellbar, so lässt sich wie in den Basis Features bereits beschrieben ein Test mit
einer oder mehreren Markierungen versehen wodurch eine Gliederung nach Markierung entsteht. Auch
durch das selektieren bestimmten Wörtern lassen sich Tests nach ihrem Namen gliedern, so entsteht
einerseits eine Gliederung zur Ausführung der Tests und andererseits eine Gliederung für die
Entwickler die sie selbst im Code sehen können. Als letzte alternative lassen sich Tests anhand
ihrer Module ausführen, dies geschieht durch die Angabe der Module. So würde
\lstinline{test_datei.py::TestKlasse::test_methode} den Test \lstinline{test_methode} der Klasse
\lstinline{TestKlasse} in der Datei \lstinline{test_datei.py} ausführen.
\newline
\\
Diese Features werden in Kapitel sechs von \cite{docs.pytest.org:4.4} beschrieben.

Selbstverständlich bietet pytest noch weitere Features jedoch sind diese nicht zwangsläufig notwendig
um TDD zu betreiben und sind mehr ein nice to have Feature als wirklich benötigt. Für eine
Vollständige Auflistung und Erklärung aller Features kann jederzeit unter
\url{https://docs.pytest.org/en/latest/} die aktuellste Version der Dokumentation abgefragt werden.

Demnach bietet pytest alles um TDD anwenden zu können und die sieben zusätzlichen Abhängigkeiten
die ein Entwickler bei der Nutzung von pytest eingeht sollten keinen Entwickler davon abhalten
pytest und seine Features zu genießen.
\newline
\\
Da pytest, wie bereits erwähnt keine neuen assert Methoden hinzufügt lässt sich sehr schnell und
einfach ein Test schreiben. Selbst die Nutzung von \Glspl{fixture} ist in pytest sehr einfach, da
lediglich eine Funktion geschrieben werden muss die mit \lstinline{@pytest.fixture} markiert wurde
und der Test Funktion oder Methode als Parameter übergeben wird. Den Rest erledigt pytest selbst
im Hintergrund. Genauso einfach gestaltet sich die Nutzung von \Glspl{mock}.

Die Effizienz, die bei der Nutzung von pytest entsteht ist demnach sehr hoch. Denn der Entwickler
muss nicht wirklich etwas neues dazu lernen um Tests zu schreiben oder zu präparieren. Genauso
leicht kann ein Entwickler den Output von pytest auswerten, da dieser erstens, in Farbe ist, was
die Lesbarkeit deutlich erhöht, zweitens sehr gut gegliedert ist und Wichtige Strukturen klar
darstellt und drittens nach den Wünschen der Entwickler sich gestalten und verbessern lässt.
\newline
\\
Durch die eben genannten Features im Bezug auf das schreiben von Tests mit assert, sowie
\Glspl{fixture} und \Glspl{mock} lässt sich sagen das pytest sehr viel Funktionalität bietet
wobei es trotzdem Struktur im Code bietet und komplexe Features einfach ermöglicht. Demnach
kann ein Entwickler mit wenig Code viel Testing Funktionalität erstellen.
\newline
\\
Sollten dem Entwickler die Features von pytest nicht ausreichen, so findet man unter
\url{http://plugincompat.herokuapp.com/} eine Liste von 618 (Stand: 2. April 2019) Erweiterungen
für \lstinline{pytest 4.3.0}. pytest selbst kann allerdings auch als Erweiterung zu unittest
genutzt werden, indem ein Entwickler pytest zum ausführen der unittests verwendet. Dadurch bietet
sich dem Entwickler ein verbesserter Output des Test Ergebnisses.
\newline
\\
Im Listing \ref{listing:pytest:advanced} und \ref{listing:pytest:advanced_output} befindet sich der
Code und der Output zum Test von dem in Listing \ref{listing:base:my_module} definierten Modul.