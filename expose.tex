\documentclass[a4paper, 12pt, oneside, headspline]{scrartcl}
% !TeX root = ../expose.tex
% Variablen welche innerhalb der gesamten Arbeit zur Verfügung stehen sollen
\newcommand{\titleDocument}{Exposé}
\newcommand{\subjectDocument}{Test-driven development mit Python}
\newcommand{\institution}{Hochschule für angewandte Wissenschaften Augsburg}
\newcommand{\insSection}{Fakultät für Informatik}
\newcommand{\name}{Maximilian Konter}
\newcommand{\email}{maximilian.konter@hs-augsburg.de}

% Farbige links
\usepackage{color}

% Umlaute unter UTF8 nutzen
\usepackage[utf8]{inputenc}

% deutsche Silbentrennung
\usepackage[ngerman]{babel}

% Grafiken aus PNG Dateien einbinden
\usepackage{graphicx}

% bricht lange URLs "schoend" um
\usepackage[hyphens,obeyspaces,spaces]{url}

% mehrseitige Tabellen ermöglichen
\usepackage{longtable}

% für Tabellen
\usepackage{array}

% Schaltet den zusätzlichen Zwischenraum ab, den LaTeX normalerweise nach einem Satzzeichen einfügt.
\frenchspacing

% Paket für Zeilenabstand
\usepackage{setspace}

% für Bildbezeichner
\usepackage{capt-of}

% für Stichwortverzeichnis
\usepackage{makeidx}

% Für Glossar
\usepackage{glossaries}

% Bessere Listen
\usepackage{enumitem}

% Festlegung Art der Zitierung - Havardmethode: Abkuerzung Autor + Jahr
\bibliographystyle{alphadin}

% für Listings
\usepackage{listings}
\lstset{numbers=left, numberstyle=\tiny, numbersep=5pt, keywordstyle=\color{black}\bfseries, stringstyle=\ttfamily,showstringspaces=false,basicstyle=\footnotesize,captionpos=b}
\lstset{language=python}

% Indexerstellung
\makeindex

% Abkürzungsverzeichnis
\usepackage[german]{nomencl}
\let\abbrev\nomenclature

% Abkürzungsverzeichnis LiveTex Version
\renewcommand{\nomname}{Abkürzungsverzeichnis}
\setlength{\nomlabelwidth}{.25\hsize}
\renewcommand{\nomlabel}[1]{#1 \dotfill}
\setlength{\nomitemsep}{-\parsep}
\makenomenclature
\makeglossaries

% Disable single lines at the start of a paragraph (Schusterjungen)
\clubpenalty = 10000
% Disable single lines at the end of a paragraph (Hurenkinder)
\widowpenalty = 10000
\displaywidowpenalty = 10000

% Hyperlinks im Text
\usepackage[bookmarksnumbered,pdftitle={\titleDocument},hyperfootnotes=false]{hyperref}
\hypersetup{colorlinks, citecolor=red, linkcolor=blue, urlcolor=black}

% neue Kopfzeilen mit fancypaket
\usepackage{fancyhdr} %Paket laden
\pagestyle{fancy} %eigener Seitenstil
\fancyhf{} %alle Kopf- und Fußzeilenfelder bereinigen
\fancyhead[L]{\nouppercase{\leftmark}} %Kopfzeile links
\fancyhead[C]{} %zentrierte Kopfzeile
\fancyhead[R]{\thepage} %Kopfzeile rechts
\renewcommand{\headrulewidth}{0.4pt} %obere Trennlinie
\fancyfoot[C]{\thepage} %Seitennummer
\renewcommand{\footrulewidth}{0.4pt} %untere Trennlinie

\begin{document}	
	% Titelseite
	% !TeX root = ../main.tex
\thispagestyle{empty}

\begin{figure}[t]
 \centering
 \includegraphics[width=0.3\textwidth]{abb/logoHSA}
\end{figure}


\begin{verbatim}


\end{verbatim}

\begin{center}
\Large{\institution}
\end{center}


\begin{center}
\Large{\insSection}
\end{center}
\begin{verbatim}




\end{verbatim}
\begin{center}
\doublespacing
\textbf{\LARGE{\titleDocument}}
\singlespacing
\begin{verbatim}

\end{verbatim}
\textbf{{~\subjectDocument~}}
\end{center}
\begin{verbatim}

\end{verbatim}
\begin{center}

\end{center}
\begin{verbatim}

\end{verbatim}
\begin{center}
\textbf{zur Erlangung des akademischen Grades \\ Bachelor of Science}
\end{center}
\begin{verbatim}






\end{verbatim}

\begin{footnotesize}
\begin{flushleft}
\begin{tabular}{llll}
\textbf{Thema:} & & \subjectDocument & \\
& & \\
\textbf{Autor:} & & \name & \\
& & \email & \\
& & MatNr. 951004 & \\
& & \\
\textbf{Version vom:} & & \today &\\
& & \\
\textbf{1. Betreuer:} & & Dipl.-Inf. (FH), Dipl.-De Erich Seifert, MA &\\
\textbf{2. BetreuerIn:} & & Prof. Dr. X &\\
\end{tabular}
\end{flushleft}
\end{footnotesize}
	
	% Inhaltsverzeichnis anzeigen
	\newpage
	\tableofcontents
	
	% das Abkürzungsverzeichnis
	\newpage
	% Abkürzungsverzeichnis soll im Inhaltsverzeichnis auftauchen
	\addcontentsline{toc}{section}{Abkürzungsverzeichnis}
	% das Abkürzungsverzeichnis entgültige Ausgeben
	\fancyhead[L]{Abkürzungsverzeichnis} %Kopfzeile links
	% !TeX root = ../../main.tex
\nomenclature{GPL}{GNU General Public License}
\nomenclature{GNU}{GNU is not Unix}
\nomenclature{LGPL}{GNU Lesser General Public License}
\nomenclature{TDD}{Test-driven development}
\nomenclature{GUI}{Graphical User Interface}
	\printnomenclature
    
    % das Glossar
    \newpage
    % Glossar soll im Inhaltsverzeichnis auftauchen
    \addcontentsline{toc}{section}{Glossar}
    % das Abkürzungsverzeichnis entgültige Ausgeben
    \fancyhead[L]{Glossar} %Kopfzeile links
    \newglossaryentry{Pythonista}{
    name={Pythonista},
    description={Jemand, der Python als seine favorisierte Sprache verwendet}
}
\newglossaryentry{Continous testing}{
    name={Continous testing},
    description={Bezeichnet den Prozess des automatischen Testens mit Hilfe
    eines Automatisierungstools. Hierbei werden die Tests durch ein Ereignis
    automatisch ausgeführt, wodurch der Entwickler sofort ein Feedback bekommt}
}
\newglossaryentry{mock}{
    name={mock},
    description={Etwas mocken bedeutet, ein Objekt durch ein falsches Objekt, den
    Mock zu ersetzen, das genau so aussieht, wie das erwartete Objekt, aber nicht das
    Gleiche ist. So ist es möglich eine Funktion zu mocken, wodurch nicht die Funktion,
    sondern der Mock aufgerufen wird und das eingestellte ergebnis liefert}
}
\newglossaryentry{commit}{
    name={commit},
    description={Ein commit ist die Sammlung von änderungen an Dateien, welche 
    mithilfe eines VCS verwaltet werden},
    plural=commits
}
\newglossaryentry{vcs}{
    name={Versions Control System},
    description={Ein VCS ist ein Dienst, der es seinen Nutzern ermöglicht änderungen
    an Dateien in einer Chronik zu speichern um später darauf zu zu greifen. Dies
    dient auch der verteilung von Daten über mehrere Systeme sowie sicherung der
    Daten vor verlust}
}
\newglossaryentry{fuzz}{
    name={fuzz-testing},
    description={Beschreibt eine Art von Test,
    bei dem das zu testende Modul oder Programm mit zufälligen Werten aufgerufen wird. Dabei
    soll jede mögliche anwendung des Moduls oder Programms dargestellt werden}
}
\newglossaryentry{bug}{
    name={bug},
    description={Ein Bug(zu deutsch Käfer) ist ein Fehler in einem Programm oder in einer Software}
}
\newglossaryentry{fixture}{
    name={fixture},
    description={Beschreibt die Präperation von Tests mit festen (fix) Werten, so wird zum Beispiel
    vor jedem Test eine Variable gesetzt. Dies dient der übersichtlichkeit, da dies dann nicht in
    jedem Test vorgenommen werden muss und um varianz zwischen Tests und Test läufen zu vermeiden}
}
\newglossaryentry{cli}{
    name={Command-line interface},
    description={Eine Benutzerschnitstelle für das Terminal}
}
\newglossaryentry{stacktrace}{
    name={StackTrace},
    description={Der StackTrace ist ein Protokoll der aufgerufenen Funktionen und Methoden die 
    ineinander verschachtelt sind, bis hin zu der tiefsten Funkion/Methode die einen Fehler
    geworfen hat}
}
\newglossaryentry{docstring}{
    name={docstring},
    description={Ein Docstring ist ein Block von Text der sich zwischen jeweils drei Anführungsstrichen befindet. Mit diesem Text wird ein Objekt in Python dokumentiert}
}
    \printglossaries
	
	%%%%% Beginn Dokument %%%%%
	\newpage
	\fancyhead[L]{\nouppercase{\leftmark}} %Kopfzeile links
	
	% 1,5 facher Zeilenabstand
	\onehalfspacing
	
	% einzelne Kapitel
	% !TeX root = expose.tex
\section{Ausgangslage}
TDD wird in der heutigen Softwareentwicklung immer verbreiteter und beliebter.
Die Ansprüche an Software sind in den letzten Jahren immer weiter gestiegen, dies
liegt vor allem an der Reichweite die heute Software hat. Circa 66\% aller Menschen
besitzen heute ein Smartphone \cite{FraukeSchobelt:Smartphone}, mit steigender
Nutzerzahlen steigen auch die Anforderungen die, die Nutzer an die Software stellen
und auch die Anzahl der gefundenen Bugs wird dementsprechend größer.



\section{Problembeschreibung}
Im Weltweiten Markt gibt es viele große unternehmen die gegenseitig um die
Nutzer kämpfen. Selbstverständlich gehen die Nutzer zu dem Anbieter der die bessere
Software bietet, doch das kann sich heute stetig ändern. Mit der steigenden Anzahl
an Bugs die gefunden werden, steigen auch die Nutzer die von diesen Bugs betroffen
sind. Diese Bugs sollen natürlich schnellstmöglich gefixt werden um so zu verhindern
das Nutzer die Software wechseln.

So schwer es ist seine Nutzer zu halten, umso schwerer ist es zum Start einer Software
Nutzer zu akquirieren. Gibt es bereits andere Software die ähnliche Services anbietet,
so ist es noch schwerer in den Markt rein zu kommen. Die Anforderungen sind durch die
anderen Firmen höher als die Anforderungen an ein neues Produkt und alte Fehler die
in anderer Software schon gelöst wurden sollten möglichst nicht auftauchen.

\section{Fragestellung}
Wie schaffen es Firmen Ihre Entwicklung so zu optimieren dass, ihre Software zu jeder
Zeit optimal Funktioniert und alte Fehler und Bugs nicht wieder auftauchen? Am besten
sollte dies von Anfang an im Projekt der Fall sein und sich bis zur Wartung der Software
durchziehen.
	
	% !TeX root = expose.tex
\section{Zielsetzung}
Das Ziel dieser Arbeit ist es, Test-Tools der Programmiersprache Python in den
Aspekten Anwendbarkeit, Effizienz, Komplexität und Erweiterbarkeit zu
vergleichen. Daraus resultierend soll eine Empfehlung entstehen wie man
möglichst einfach TDD mit Python betreiben kann, um so die Motivation Tests zu
schreiben zu steigern.

Um die effizient von TDD zu steigern wird in dieser Arbeit auf die Automation
der Tests eingegangen, dies wird mithilfe des Tools
Travis CI\footnote{\href{https://travis-ci.org/}{Travis CI Website}}
darzustellen.

Nach dem Vergleich soll der Leser ein Bild darüber haben welche Tools ihm zur
Verfügung stehen um Tests zu schreiben oder TDD zu betreiben. Zusätzlich soll
er ein wenig Verständnis darüber haben wie er mithilfe von
\gls{Continous testing} effektiver TDD betreiben kann.
\newline
Am Ende der Arbeit werden die Vor- und Nachteile von TDD abgewogen, für die
Nachteile soll eine Lösung angeboten werden (soweit dies möglich ist) um TDD
attraktiver zu gestalten.

	% !TeX root = ../expose.tex
\section{Theoretische Grundlage}
Hier aufgeführt findet sich einiges an Literatur dass relevant für diese Arbeit
sein könnte

\begin{itemize}
    \item 5 Jährige Studie von IBM aus dem Jahr 2007
    \url{http://citeseerx.ist.psu.edu/viewdoc/download?doi=10.1.1.104.6319&rep=rep1&type=pdf}
    \item Diskussion zu einem Buch über Empirische Softwareentwicklung (Paywall)
    \url{https://www.infoq.com/news/2009/03/TDD-Improves-Quality}
    \item Studie über den gebraucht von TDD und der daraus resultierenden design
    und Test verbesserungen
    \url{https://arxiv.org/pdf/1711.05082.pdf}
    \item TDD vs Test-Last
    \url{https://www.researchgate.net/publication/315743099_An_Experimental_Evaluation_of_Test_Driven_Development_vs_Test-Last_Development_with_Industry_Professionals}
    \item TDD vs nicht TDD auf Github
    \url{https://peerj.com/preprints/1920/}
    \item weitere
\end{itemize}

	% !TeX root = ../expose.tex
\section{Konzept}
Hypothese: TDD ist besser?
Methodik: Wie soll dies bewiesen werden? Anhand der Planung einer komplette umgebung
	%% !TeX root = ../expose.tex
\section{Vorläufige Gliederung}
\begin{enumerate}
	\item Abstract
	\item Einleitung
\end{enumerate}
	% !TeX root = expose.tex
\section{Motivation}
Da ich selbst ein \gls{Pythonista} bin, war es für mich naheliegend meine Bachelorarbeit
dieser Sprache zu widmen. Durch mein Praxissemester und dem daraus resultierenden
Werkstudentenjob, in dem ich seit eineinhalb Jahren an der Test- und Buildautomatisierung
arbeite, wurde mein Interesse in die automatisierte Ausführung von Tests geweckt. Auch mein
Bedürfnis, Tests für meine Programme zu schreiben, ist enorm gewachsen, da ich festgestellt
habe, wie gut das Gefühl ist, wenn man etwas im Code ändert und alle Tests danach noch
zu 100\% durchlaufen.

Seitdem ist Testen ein Teil meiner Entwicklung und auch meines Denkens. TDD selbst habe
ich jedoch noch nicht ausprobiert, aber der Reiz für mich ist hoch, da ich denke, dass
Testen von Anfang an sich nur positiv auf die Software auswirken kann.

Auch mein Interesse für die Automatisierung möchte ich in dieser Bachelorarbeit 
widerspiegeln, da dies den Entwicklungsprozess erheblich erleichtert und schneller
gestaltet.
	
	\onecolumn
	
	% einfacher Zeilenabstand
	\singlespacing
	
	% Literaturliste soll im Inhaltsverzeichnis auftauchen
	\newpage
	\addcontentsline{toc}{section}{Literaturverzeichnis}
	
	% Literaturverzeichnis anzeigen
	\renewcommand\refname{Literaturverzeichnis}
	\bibliography{expose}
\end{document}