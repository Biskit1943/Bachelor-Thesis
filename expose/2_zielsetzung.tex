% !TeX root = ../expose.tex
\section{Zielsetzung}
Das Ziel dieser Arbeit ist es, Test-Tools der Programmiersprache Python in den
Aspekten Anwendbarkeit, Effizienz, Komplexität und Erweiterbarkeit zu
vergleichen. Daraus resultierend soll eine Empfehlung entstehen, wie man
möglichst einfach TDD mit Python betreiben kann, um so die Motivation zu steigern
Tests zu schreiben.

Um die Effizienz von TDD zu steigern, wird in dieser Arbeit auf die Automatisierung
der Tests eingegangen. Dies wird mit Hilfe des Tools
Travis CI\footnote{\href{https://travis-ci.org/}{Travis CI Website}}
dargestellt.

Die Arbeit soll Anwendern einen Leitfaden yur verfügung stellen welche Tools
zur Zeit zum betreiben von TDD die besten sind.
Nach dem Vergleich soll der Leser ein Bild darüber haben, welche Tools ihm zur
Verfügung stehen um Tests zu schreiben oder TDD zu betreiben. Zusätzlich soll
er ein wenig Verständnis darüber haben, wie er mit Hilfe von
\gls{Continous testing} effektiver TDD betreiben kann.
\newline
Am Ende der Arbeit werden die Vor- und Nachteile von TDD abgewogen, für die
Nachteile soll, soweit dies möglich ist, eine Lösung angeboten werden um TDD
attraktiver zu gestalten.
