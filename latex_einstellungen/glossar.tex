\newglossaryentry{Pythonista}{
    name={Pythonista},
    description={Jemand, der Python als seine favorisierte Sprache verwendet}
}
\newglossaryentry{Continous testing}{
    name={Continous testing},
    description={Bezeichnet den Prozess des automatischen Testens mit Hilfe
    eines Automatisierungstools. Hierbei werden die Tests durch ein Ereignis
    automatisch ausgeführt, wodurch der Entwickler sofort ein Feedback bekommt}
}
\newglossaryentry{mock}{
    name={mock},
    description={Etwas mocken bedeutet, ein Objekt durch ein falsches Objekt, den
    Mock zu ersetzen, das genau so aussieht, wie das erwartete Objekt, aber nicht das
    Gleiche ist. So ist es möglich eine Funktion zu mocken, wodurch nicht die Funktion,
    sondern der Mock aufgerufen wird und das eingestellte ergebnis liefert.}
}
\newglossaryentry{commit}{
    name={commit},
    description={Ein commit ist die Sammlung von änderungen an Dateien, welche 
    mithilfe eines VCS verwaltet werden},
    plural=commits
}
\newglossaryentry{vcs}{
    name={Versions Control System},
    description={Ein VCS ist ein Dienst, der es seinen Nutzern ermöglicht änderungen
    an Dateien in einer Chronik zu speichern um später darauf zu zu greifen. Dies
    dient auch der verteilung von Daten über mehrere Systeme sowie sicherung der
    Daten vor verlust}
}
\newglossaryentry{fuzz}{
    name={fuzz-testing},
    description={Beschreibt eine Art von Test,
    bei dem das zu testende Modul oder Programm mit zufälligen Werten aufgerufen wird. Dabei
    soll jede mögliche anwendung des Moduls oder Programms dargestellt werden}
}
\newglossaryentry{bug}{
    name={bug},
    description={Ein Bug(zu deutsch Käfer) ist ein Fehler in einem Programm oder in einer Software}
}
\newglossaryentry{fixture}{
    name={fixture},
    description={Ein vordefiniertes Element welches einen festen Wert hat.
    Ein tests kann von einer Fixture abhängig sein, wodurch bei verschiedenen durchläufen davon ausgegangen
    werden kann das ein Fehler nicht an diesem Element liegt wenn es vorher bereits funktioniert hatte.
    Dadurch kann man fehler externer abhängigkeiten ausschließen.}
}
\newglossaryentry{cli}{
    name={Command-line interface},
    description={Eine Benutzerschnitstelle für das Terminal}
}