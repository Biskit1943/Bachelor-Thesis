% !TeX root = ../bachelor.tex
\newglossaryentry{Pythonista}{
    name={Pythonista},
    description={Jemand, der Python als seine favorisierte Sprache verwendet}
}
\newglossaryentry{Continous testing}{
    name={Continous testing},
    description={Bezeichnet den Prozess des automatischen Testens mit Hilfe
    eines Automatisierungstools. Hierbei werden die Tests durch ein Ereignis
    automatisch ausgeführt, wodurch der Entwickler sofort ein Feedback bekommt}
}
\newglossaryentry{mock}{
    name={mock},
    description={Ein Mock (zu Detusch: Attrappe) ist ein Objekt, dass von dem zu
    mockenden Objekt abstammt. Das heißt es besitzt alle eigenschaften des originals,
    jedoch fügt es noch weitere Methoden hizu. Diese ermöglichen das prüfen, welche
    weiteren Methoden und Funktionen mit welchen Parametern ausgeführt wurden}
}
\newglossaryentry{stub}{
    name={stub},
    description={Ein Stub (zu Deutsch: Stumpf) ist, ähnlich wie ein Mock, eine Attrappe.
    Jedoch werden hier nicht weitere Methoden hinzugefügt, sondern jediglich die vorhandenen
    Methoden werden ersetzt. Dabei wird allerdings keine Logik implementiert, sondern es
    wird ein Wert gesetzt den die Methode zurück gibt}
}
\newglossaryentry{commit}{
    name={commit},
    description={Ein commit ist die Sammlung von änderungen an Dateien, welche 
    mithilfe eines VCS verwaltet werden},
    plural=commits
}
\newglossaryentry{vcs}{
    name={Versions Control System},
    description={Ein VCS ist ein Dienst, der es seinen Nutzern ermöglicht änderungen
    an Dateien in einer Chronik zu speichern um später darauf zu zu greifen. Dies
    dient auch der verteilung von Daten über mehrere Systeme sowie sicherung der
    Daten vor verlust}
}
\newglossaryentry{fuzz}{
    name={fuzz-testing},
    description={Beschreibt eine Art von Test,
    bei dem das zu testende Modul oder Programm mit zufälligen Werten aufgerufen wird. Dabei
    soll jede mögliche anwendung des Moduls oder Programms dargestellt werden}
}
\newglossaryentry{bug}{
    name={bug},
    description={Ein Bug(zu deutsch Käfer) ist ein Fehler in einem Programm oder in einer Software}
}
\newglossaryentry{fixture}{
    name={fixture},
    description={Beschreibt die Präperation von Tests mit festen (fix) Werten, so wird zum Beispiel
    vor jedem Test eine Variable gesetzt. Dies dient der übersichtlichkeit, da dies dann nicht in
    jedem Test vorgenommen werden muss und um varianz zwischen Tests und Test läufen zu vermeiden}
}
\newglossaryentry{cli}{
    name={Command-line interface},
    description={Eine Benutzerschnitstelle für das Terminal}
}
\newglossaryentry{stacktrace}{
    name={StackTrace},
    description={Der StackTrace ist ein Protokoll der aufgerufenen Funktionen und Methoden die 
    ineinander verschachtelt sind, bis hin zu der tiefsten Funkion/Methode die einen Fehler
    geworfen hat}
}
\newglossaryentry{docstring}{
    name={docstring},
    description={Ein Docstring ist ein Block von Text der sich zwischen jeweils drei 
    Anführungsstrichen befindet. Mit diesem Text wird ein Objekt in Python dokumentiert}
}
\newglossaryentry{context}{
    name={contextmanager},
    description={Ein Contextmanager in Python ist ein Objekt, dass einen Kontext bietet,
    in dem gearbeitet werden kann. Dieser Kontext wird mit \lstinline{with {contextmanager} as c}
    geöffnet und schließt sich selbst beim verlassen des Kontextes. Dadurch stehen dem entwickler
    der nutzbare Kontext als Variable \lstinline{c} zur Verfügung}
}
\newglossaryentry{paket}{
    name={Paket-Manager},
    description={Ein Paket-Manager ist ein Tool zum verwalten von externen Python bibliotheken,
    mitdessen Hilfe Pakete (bibliotheken) installiert und versioniert werden können}
}
\newglossaryentry{decorator}{
    name={decorator},
    description={Ein Decorator ist eine Funktion die eine andere Funktion oder Methode umschließt,
    wodurch der Code im Decorator vor und/oder nach der Funktion oder Methode ausgeführt werden kann.
    Ein Decorator wird einer anderen Funktion mit \lstinline{@\{decorator-funktion\}} über der
    Funktions- oder Methoden-Definition übergeben. Das adjektiv für einen Decorator ist allerdings
    nicht dekorieren, sondern annotieren}
}
\newglossaryentry{annotation}{
    name={annotation},
    description={Eine Annotation ist das gleiche wie ein decorator, mit ausnahme, dass eine Annotation
    über jedem Wert, jeder Funktion und jeder Klasse stehen kann. Eine Annotation wird mit hilfe von
    \lstinline{@\{name\}} über das zu annotierende Objekt geschrieben. Diese dienen dazu gewisse aktionen aus
    zu führen oder dem Programm etwas mit zu teilen, wie zum Beispiel, dass eine Methode veraltet ist}
}
\newpage
% Glossar soll im Inhaltsverzeichnis auftauchen
\phantomsection
\addcontentsline{toc}{subsection}{Glossar}
\fancyhead[L]{Glossar} % Kopfzeile links
% das Abkürzungsverzeichnis entgültige Ausgeben
\printglossaries