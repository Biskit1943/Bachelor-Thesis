% !TeX root = ../expose.tex
\section{Konzept}
Die Test-Tools aus der Standard Bibliothek von Python sowie die größten
Test-Tools die es sonst noch gibt werden anhand folgender Aspekte verglichen:

\begin{itemize}
    \item Anwendbarkeit
        \subitem Bietet das Tool alles um TDD betreiben zu können? (Bei TDD
        kann man Tests \gls{mocken}, wodurch sie nicht schief gehen)
        Ist das Tool abhängig von vielen Paketen, die man extra installieren
        muss, da sie nicht in der Sandard Bibliothek sind?
    \item Effizienz
        \subitem Wie viel lässt sich mit diesem Tool möglichst einfach und
        schnell erreichen? Ist besonders viel Vorarbeit notwendig um die Tests
        auf zu setzen oder kann sofort mit dem schreiben der Tests begonnen
        werden.
        \newline
        \\
        Genau so stellt sich die Frage wie schnell kann der Entwickler sehen
        und verstehen wieso Tests schiefgegangen sind?
    \item Komplexität
        \subitem Wie komplex ist das Tool? Das heißt, wie viel Funktionalität
        bietet das Tool dem Entwickler von Haus aus, aber auch wie schwer es
        ist Code zu schreiben oder wie schnell Code unübersichtlich wird, da
        das Tool viel Code abseits der Tests benötigt.
    \item Erweiterbarkeit
        \subitem Wie leicht lässt sich das Tool mit anderen Tools erweitern?
        Gibt es vielleicht Erweiterungen der Community für dieses Tool die sehr
        hilfreich sind?
    \item Eventuell werden beim schreiben der Arbeit weitere Aspekte entstehen
    die hier aufgeführt werden können.
\end{itemize}

Dieser Vergleich entsteht zum einen durch das analysieren der Tools auf der
jeweiligen Website sowie durch ein komplexes Code-Beispiel das mit dem
jeweiligen Tool geschrieben wird. Passend zum Code-Beispiel wird auch analysiert
wie einfach die Automatisierung mit Travis CI verläuft indem dies auch mit
echtem Code getestet wird.

Sind die Grundlagen für den Vergleich geschaffen, werden die Tools
untereinander verglichen und ihre schwächen und stärken noch einmal zusammen
gefasst.
\newline
\\
Sofern dies möglich ist wird versucht die Tools zu kombinieren um etwaige
Schwächen einen Tools zu umgehen.

Nachdem das/die beste(n) Tool(s) gefunden wurde, wird auf die schwächen von TDD
im allgemeinen eingegangen. Dies behandelt den erforderlichen Aufwand und die
daraus resultierende Zeit die "verloren" geht (Zeit die statt für Tests
schreiben auch für Code schreiben genutzt werden könnte) sowie die Disziplin
die den Entwicklern abverlangt wird, zuerst Tests zu schreiben bevor die
tatsächliche Funktionalität geschrieben wird. Dabei wird auch noch auf den
Wirtschaftlichen Aspekt von TDD eingegangen den ein Arbeitgeber in zu beachten
hat bei der Entscheidung TDD zu verwenden.

Am Ende der Arbeit werde ich meine Persönliche Meinung über TDD und ob ich es
als Entwickler oder Arbeitgeber nutzen würde Preis geben.
