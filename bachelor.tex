\documentclass[a4paper, 12pt, oneside, headspline]{scrartcl}
% !TeX root = ../main.tex
% Variablen welche innerhalb der gesamten Arbeit zur Verfügung stehen sollen
\newcommand{\titleDocument}{Bachelorarbeit}
\newcommand{\subjectDocument}{Test-driven development mit Python}
\newcommand{\institution}{Hochschule für angewandte Wissenschaften Augsburg}
\newcommand{\insSection}{Fakultät für Informatik}
\newcommand{\name}{Maximilian Konter}
\newcommand{\email}{maximilian.konter@hs-augsburg.de}
\usepackage[openbib]{currvita}

% wichtig für Trennung von Wörtern mit Umlauten
\usepackage[T1]{fontenc}

% verbesserter Randausgleich
\usepackage{microtype}

% Farbige links
\usepackage{color}

% Umlaute unter UTF8 nutzen
\usepackage[utf8]{inputenc}

% deutsche Silbentrennung
\usepackage[ngerman]{babel}

% Grafiken aus PNG Dateien einbinden
\usepackage{graphicx}

% bricht lange URLs "schoend" um
\usepackage[hyphens,obeyspaces,spaces]{url}

% mehrseitige Tabellen ermöglichen
\usepackage{longtable}

% für Tabellen
\usepackage{array}

% Schaltet den zusätzlichen Zwischenraum ab, den LaTeX normalerweise nach einem Satzzeichen einfügt.
\frenchspacing

% Paket für Zeilenabstand
\usepackage{setspace}

% für Bildbezeichner
\usepackage{capt-of}

% für Stichwortverzeichnis
\usepackage{makeidx}

% Für Glossar
\usepackage{glossaries}

% Ermöglicht das Fußnotenverzeichnis
\usepackage[titles]{tocloft}

%% Fußnotenverzeichnis von https://stackoverflow.com/a/6914775/8339179
\newcommand{\listfootnotesname}{Fußnotenverzeichnis}%     % 'List of Footnotes' title 
\newlistof[chapter]{footnotes}{fnt}{\listfootnotesname}%  % New 'List of...' for footnotes 
\let\oldfootnote\footnote%                                % Save the old \footnote{...} command 
\renewcommand\footnote[1]{%                               % Redefine the new footnote to also add 'List of Footnote' entries. 
    \refstepcounter{footnotes}%                           % Add and step a reference to the footnote/counter. 
    \oldfootnote{#1}%                                     % Make a regular footnote. 
    \addcontentsline{fnt}{footnotes}{\protect% 
        \numberline{\thefootnotes}#1}%                    % Add the 'List of...' entry. 
}

% Ermöglicht mehr als 3 punkte tiefe -> 1.2.3.4
\setcounter{secnumdepth}{4}
\setcounter{tocdepth}{4}

\newcommand{\myparagraph}[1]{\paragraph{#1}\mbox{}\\\noindent}

% Von https://tex.stackexchange.com/a/10110 mit additions aus https://stackoverflow.com/a/6914775/8339179
\newcommand{\footlabel}[2]{%
    \refstepcounter{footnotes}%
    \addcontentsline{fnt}{footnotes}{\protect 
        \numberline{\thefootnotes}#2}%
    \addtocounter{footnote}{1}%
    \footnotetext[\thefootnote]{%
        \addtocounter{footnote}{-1}%
        \refstepcounter{footnote}\label{#1}
        #2%
    }%
    $^{\ref{#1}}$%
}

\renewcommand{\footref}[1]{%
    $^{\ref{#1}}$%
}


% Bessere Listen
\usepackage{enumitem}

\usepackage[
    maxnames=6,
    minnames=1,
    style=apa,
    backend=biber,
    bibencoding=utf8,
    sortlocale=de_DE,
]{biblatex}
\DeclareLanguageMapping{ngerman}{ngerman-apa}
\DefineBibliographyStrings{ngerman}{
    andothers = {{et\,al\adddot}},
}

% für Listings
\usepackage{listings}
\lstset{
    numbers=left,
    numberstyle=\tiny,
    numbersep=5pt,
    keywordstyle=\color{black}\bfseries,
    stringstyle=\ttfamily,
    showstringspaces=false,
    basicstyle=\footnotesize,
    captionpos=b
}


% Abkürzungsverzeichnis
\usepackage[german]{nomencl}
\let\abbrev\nomenclature

% Abkürzungsverzeichnis LiveTex Version
\renewcommand{\nomname}{Abkürzungsverzeichnis}
\setlength{\nomlabelwidth}{.25\hsize}
\renewcommand{\nomlabel}[1]{#1 \dotfill}
\setlength{\nomitemsep}{-\parsep}
\makenomenclature
\makeglossaries

% Disable single lines at the start of a paragraph (Schusterjungen)
\clubpenalty = 10000
% Disable single lines at the end of a paragraph (Hurenkinder)
\widowpenalty = 10000
\displaywidowpenalty = 10000

% Hyperlinks im Text
\usepackage[bookmarksnumbered,pdftitle={\titleDocument},hyperfootnotes=false]{hyperref}
\hypersetup{colorlinks, citecolor=red, linkcolor=blue, urlcolor=blue}

% neue Kopfzeilen mit fancypaket
\usepackage{fancyhdr}                  % Paket laden
\pagestyle{fancy}                      % eigener Seitenstil
\fancyhf{}                             % alle Kopf- und Fußzeilenfelder bereinigen
\fancyhead[L]{\nouppercase{\leftmark}} % Kopfzeile links
\fancyhead[C]{}                        % zentrierte Kopfzeile
\fancyhead[R]{\thepage}                % Kopfzeile rechts
\renewcommand{\headrulewidth}{0.4pt}   % obere Trennlinie
\fancyfoot[C]{\thepage}                % Seitennummer
\renewcommand{\footrulewidth}{0.4pt}   % untere Trennlinie

% Bibliothek hinzufuegen
\addbibresource{bachelor.bib}

\begin{document}
    % Leere Seite am Anfang
    \thispagestyle{empty} % erzeugt Seite ohne Kopf- / Fusszeile
    \section*{ }
    
    \newpage

    % Titelseite
    % !TeX root = ../main.tex
\thispagestyle{empty}

\begin{figure}[t]
 \centering
 \includegraphics[width=0.3\textwidth]{abb/logoHSA}
\end{figure}


\begin{verbatim}


\end{verbatim}

\begin{center}
\Large{\institution}
\end{center}


\begin{center}
\Large{\insSection}
\end{center}
\begin{verbatim}




\end{verbatim}
\begin{center}
\doublespacing
\textbf{\LARGE{\titleDocument}}
\singlespacing
\begin{verbatim}

\end{verbatim}
\textbf{{~\subjectDocument~}}
\end{center}
\begin{verbatim}

\end{verbatim}
\begin{center}

\end{center}
\begin{verbatim}

\end{verbatim}
\begin{center}
\textbf{zur Erlangung des akademischen Grades \\ Bachelor of Science}
\end{center}
\begin{verbatim}






\end{verbatim}

\begin{footnotesize}
\begin{flushleft}
\begin{tabular}{llll}
\textbf{Thema:} & & \subjectDocument & \\
& & \\
\textbf{Autor:} & & \name & \\
& & \email & \\
& & MatNr. 951004 & \\
& & \\
\textbf{Version vom:} & & \today &\\
& & \\
\textbf{1. Betreuer:} & & Dipl.-Inf. (FH), Dipl.-De Erich Seifert, MA &\\
\textbf{2. BetreuerIn:} & & Prof. Dr. X &\\
\end{tabular}
\end{flushleft}
\end{footnotesize}

    % 1.5 facher Zeilenabstand
    \onehalfspacing

    % Einleitung / Abstract
    % !TeX root = ../bachelor.tex
\begin{abstract}
    Abstract kommt hier hin
\end{abstract}


    % einfacher Zeilenabstand
    \singlespacing

    % Inhaltsverzeichnis anzeigen
    \newpage
    \tableofcontents

    % das Abbildungsverzeichnis
    %\newpage
    % Abbildungsverzeichnis soll im Inhaltsverzeichnis auftauchen
    %\phantomsection
    %\addcontentsline{toc}{section}{Abbildungsverzeichnis}
    % \fancyhead[L]{Abbildungsverzeichnis} % Kopfzeile links
    % Abbildungsverzeichnis endgültig anzeigen
    %\listoffigures

    % das Tabellenverzeichnis
    %\newpage
    % Abbildungsverzeichnis soll im Inhaltsverzeichnis auftauchen
    %\phantomsection
    %\addcontentsline{toc}{section}{Tabellenverzeichnis}
    % \fancyhead[L]{Tabellenverzeichnis} % Kopfzeile links
    % Abbildungsverzeichnis endgueltig anzeigen
    %\listoftables

    % das Listingverzeichnis
    % !TeX root = ../bachelor.tex
\newpage
% Listingverzeichnis soll im Inhaltsverzeichnis auftauchen
\phantomsection
\addcontentsline{toc}{section}{Listingverzeichnis}
\fancyhead[L]{Listingverzeichnis} % Kopfzeile links
\renewcommand{\lstlistlistingname}{Listingverzeichnis}
\lstlistoflistings
      
    % das Fußnotenverzeichnis
    % !TeX root = ../bachelor.tex
\newpage
% Fußnotenverzeichnis soll im Inhanltsverzeichnis auftauchen
\phantomsection
\addcontentsline{toc}{subsection}{Fußnotenverzeichnis}
\fancyhead[L]{Fußnotenverzeichnis} % Kopfzeile links
\fancyhead[L]{\nouppercase{\leftmark}} % Kopfzeile links
% Fußnotenverzeichnis endgültig anzeigen
\listoffootnotes
    
    % das Glossar
    \newglossaryentry{Pythonista}{
    name={Pythonista},
    description={Jemand, der Python als seine favorisierte Sprache verwendet}
}
\newglossaryentry{Continous testing}{
    name={Continous testing},
    description={Bezeichnet den Prozess des automatischen Testens mit Hilfe
    eines Automatisierungstools. Hierbei werden die Tests durch ein Ereignis
    automatisch ausgeführt, wodurch der Entwickler sofort ein Feedback bekommt}
}
\newglossaryentry{mock}{
    name={mock},
    description={Etwas mocken bedeutet, ein Objekt durch ein falsches Objekt, den
    Mock zu ersetzen, das genau so aussieht, wie das erwartete Objekt, aber nicht das
    Gleiche ist. So ist es möglich eine Funktion zu mocken, wodurch nicht die Funktion,
    sondern der Mock aufgerufen wird und das eingestellte ergebnis liefert}
}
\newglossaryentry{commit}{
    name={commit},
    description={Ein commit ist die Sammlung von änderungen an Dateien, welche 
    mithilfe eines VCS verwaltet werden},
    plural=commits
}
\newglossaryentry{vcs}{
    name={Versions Control System},
    description={Ein VCS ist ein Dienst, der es seinen Nutzern ermöglicht änderungen
    an Dateien in einer Chronik zu speichern um später darauf zu zu greifen. Dies
    dient auch der verteilung von Daten über mehrere Systeme sowie sicherung der
    Daten vor verlust}
}
\newglossaryentry{fuzz}{
    name={fuzz-testing},
    description={Beschreibt eine Art von Test,
    bei dem das zu testende Modul oder Programm mit zufälligen Werten aufgerufen wird. Dabei
    soll jede mögliche anwendung des Moduls oder Programms dargestellt werden}
}
\newglossaryentry{bug}{
    name={bug},
    description={Ein Bug(zu deutsch Käfer) ist ein Fehler in einem Programm oder in einer Software}
}
\newglossaryentry{fixture}{
    name={fixture},
    description={Beschreibt die Präperation von Tests mit festen (fix) Werten, so wird zum Beispiel
    vor jedem Test eine Variable gesetzt. Dies dient der übersichtlichkeit, da dies dann nicht in
    jedem Test vorgenommen werden muss und um varianz zwischen Tests und Test läufen zu vermeiden}
}
\newglossaryentry{cli}{
    name={Command-line interface},
    description={Eine Benutzerschnitstelle für das Terminal}
}
\newglossaryentry{stacktrace}{
    name={StackTrace},
    description={Der StackTrace ist ein Protokoll der aufgerufenen Funktionen und Methoden die 
    ineinander verschachtelt sind, bis hin zu der tiefsten Funkion/Methode die einen Fehler
    geworfen hat}
}
\newglossaryentry{docstring}{
    name={docstring},
    description={Ein Docstring ist ein Block von Text der sich zwischen jeweils drei Anführungsstrichen befindet. Mit diesem Text wird ein Objekt in Python dokumentiert}
}

    % das Abkürzungsverzeichnis
    % !TeX root = ../../main.tex
\nomenclature{GPL}{GNU General Public License}
\nomenclature{GNU}{GNU is not Unix}
\nomenclature{LGPL}{GNU Lesser General Public License}
\nomenclature{TDD}{Test-driven development}
\nomenclature{GUI}{Graphical User Interface}

    %%%%% Beginn Dokument %%%%%
    \newpage
    \fancyhead[L]{\nouppercase{\leftmark}} % Kopfzeile links

    % 1,5 facher Zeilenabstand
    \onehalfspacing

    % einzelne Kapitel
    % !TeX root = ../bachelor.tex
\section*{Einleitung}\label{einleitung}
\addcontentsline{toc}{section}{Einleitung}

Die Ansprüche an Software sind in den letzten Jahren immer weiter gestiegen.
Dies liegt vor allem an der Reichweite, die Software heute hat, so besitzen
im Jahr 2018 bereits circa 66\% aller Menschen ein Smartphone
(\cite{FraukeSchobelt:Smartphone}) und im Arbeitsleben ist ein PC meist nicht
mehr weg zu denken. Die dadurch benötigte Software muss bestimmte
Anforderungen, welche die Nutzer an diese stellen gewährleisten und
sicherstellen. Mit steigenden Nutzer Zahlen, steigen oft auch die Anforderungen
an eine Software sowie die gefundenen \Glspl{bug} in dieser.

Im weltweiten Markt gibt es viele große Unternehmen, die gegenseitig um ihre
Nutzer kämpfen. Die Anwender präferieren meist denjenigen Anbieter, welcher die
bessere Software bietet. An dieser Stelle ist der Software-Architekt einer
jeden Anwendung gefragt, welche Technologie eingesetzt wird um die Qualitativen
Anforderungen an diese zu gewährleisten. Eine dieser Technologien ist das
Test-driven development, welches sich durch eine hohe resultierende
Test-Abdeckung ihren Namen gemacht hat. Bei dieser Methode werden zuerst die
Tests und dann der Code geschrieben, um so später sicher stellen zu können,
dass die Software alle Anforderungen erfüllt und keine \Glspl{bug} enthält.

Gerade zum eintritt einer Neuen Software ist es für Unternehmen schwer Nutzer
zu akquirieren. Andere Anwendungen, die bereits auf dem Markt sind, haben
schon viele ihrer \Glspl{bug} gefixed und verfügen über eine durch den
Anwender ausgiebig getestete Anwendung. Um eine neue Applikation in diesen
Markt zu bringen, ist es erforderlich, dass diese zum Start ohne Fehler läuft.
Dies kann beispielsweise mit Test-driven development sicher gestellt werden.

Eine der Programmiersprachen mit der Test-driven development möglich ist, ist
Python. Python wird heutzutage von vielen Entwicklern immer häufiger eingesetzt
und ist Mittlerweile die am zweit meisten geliebte und am vierthäufigsten
genutzte Sprache Weltweit (\cite{stackoverflow:2019}).

Aufgrund der Zunehmenden Beliebtheit, sowie meiner Eigenen Vorliebe für die
Programmiersprache Python (Die Bezeichnung dafür ist \Gls{Pythonista}) habe ich
mich dazu entschieden meine Bachelorarbeit dieser zu widmen. Da ich beginnend
mit dem Praxissemester und danach als Werkstudent in einer Firma an der
Entwicklung der Hausinternen Test-Umgebung mitgewirkt habe, wurde ich immer
interessierter an dem Thema Testen. Durch die Beschäftigung mit Tests in meiner
Freizeit bin ich irgendwann auf das Test-driven development gestoßen und war
sofort überzeugt, das dies eine Technologie ist, die ich anwenden möchte. Doch
leider war die Auswahl der Tools, welche für Test-driven development in Frage
kamen so hoch, das ich Angst hatte etwas verpassen zu können, da ich ein Tool
einem anderen vorzog. Beim Analysieren der Tools viel mir auf, dass viele der
Tools veraltet oder ungeeignet waren, was einen Einsatz für Test-driven
development ausschloss. Die übrig gebliebenen Tools waren in Ihrem Umfang so
mächtig, dass ich darüber lustig machte, dass es möglich sei eine ganze Arbeit
darüber zu schreiben, weshalb ich das Projekt erst einmal zur Seite legte.

Kurze Zeit später musste ich mich für meine Bachelorarbeit bereit machen und
kam wieder auf das Thema zurück. Da ich bereits ein fundiertes Wissen bezüglich
Test-driven development besaß, entschied ich mich dazu dieses Wissen aufs
Papier zu bringen um anderen Entwicklern die Arbeit zu ersparen die einzelnen
Tools Analysieren zu müssen.

Das Ziel dieser Arbeit ist es, verschiedene Aktuelle Test-Tools der
Programmiersprache Python auf ihre Tauglichkeit für den Einsatz von Test-driven
development zu untersuchen und zu vergleichen. Dabei wird zuerst jedes Tool und
seine Funktionalitäten beschrieben. Im Anschluss wird ein jedes auf die
verschiedene Anforderungen untersucht, welche es erfüllen muss um für
Test-driven development in Frage zu kommen. Dies geschieht anhand eines
erstellten Codes Beispiels, welche gewisse Funktionalität liefert (oder nicht).
Dieses Beispiel wird mithilfe der Verschiedenen Tools auf die gleiche Art
getestet. Die dabei entstehenden Komplikationen sowie der benötigte Aufwand
ergeben die Ergebnisse, welche für die Analyse benötigt werden.

Das Ergebnis der Arbeit soll aufzeigen Welche Tools zusammen die optimale
Verbindung bieten um Test-driven development durch zu führen. Dabei erwarte ich
mir auch tiefere Einblicke in das Test-driven development zu erhalten um so
heraus zu finden ob diese Technologie für mich relevant ist.

Dazu wird in Kapitel \ref{definition} zunächst geklärt, welche Eigenschaften
die Programmiersprache Python hat, sowie die Definition von Test-driven
development. Die genutzte Methodik wird in Kapitel \ref{methodik} beschrieben
und anschließend in Kapitel \ref{python-tools} angewendet. Die Analyse ist in
vier Unterpunkte gegliedert, in Punkt \ref{python-tools:stdlib} werden zunächst
die Tools der Standardbibliothek analysiert. Anschließend werden unter Punkt
\ref{python-tools:extlib} die Tools, welche nicht in der Standardbibliothek
sind analysiert. Darauffolgend werden unter Punkt \ref{python-tools:vergleich}
die analysierten Tool verglichen um dann unter Punkt
\ref{python-tools:kombination} mögliche Kombinationen dieser auf zu Zeigen.
Bevor in Kapitel \ref{fazit} das Fazit gezogen wird, werden die Ergebnisse der
Arbeit in Kapitel \ref{diskussion} diskutiert.
    % !TeX root = ../bachelor.tex
\section{Python Test-Tools}\label{python-tools}

Dieses Kapitel befasst sich mit den von der STDLIB bereitgestellten Test-Tools
sowie denen aus externen Paketen.
Diese werden unter \ref{python-tools:stdlib} und \ref{python-tools:extlib}
zusammengefasst, wobei diese unterteilt sind in unit-testing -,
\gls{mock}-testing - und \gls{fuzz} Tools.

Die unit-testing Tools sind Tools die Funktionalität zum testen bereitstellen.
Jedoch wird für TDD weit mehr als nur Tests benötigt, aus diesem Grund werden
\gls{mock}-testing - und \gls{fuzz} Tools zusätzlich behandelt. Dabei soll aber
zwischen einer reinen Erweiterung eines Tools und der Erweiterung von allen Tools
unterschieden werden.
Ist zum Beispiel ein Tool nur zusammen mit einem anderen Tool, das die Funktionalität
bereit stellt Tests aus zu führen, so ist dieses Tool als Erweiterung zu sehen. Bietet
ein Tool allerdings Code zum Erweitern von verschiedenen Test Tools zur Verfügung, so
wird es hier zur Analyse verwendet.
Diese Unterscheidung wird verwendet um Duplikationen in den einzelnen vergleichen zu
vermeiden, des weiteren gibt es Tools mit x Erweiterungen, diese alle zu vergleichen würde
den Rahmen dieser Arbeit bei weitem sprengen.
Sollten sich für ein Tool besonders interessante Erweiterungen finden, so werden diese
in der Analyse des jeweiligen Tools erwähnt und verlinkt.
\newline
\\
Jedes Tool wird anhand folgender Aspekte untersucht:
\begin{itemize}
    \item \underline{Anwendbarkeit}:\newline
    Bietet das Tool alles, um TDD betreiben zu können? (\Glspl{fixture} und \Glspl{mock})
    Mit wie vielen Paketen muss das Tool betrieben werden? (Mehr Abhängigkeiten führen zu
    mehr externe Entwicklern auf die man sich verlassen muss.)
    
    \item \underline{Effizienz}:\newline
    Wie viel lässt sich mit diesem Tool möglichst einfach und
    schnell erreichen? Ist besonders viel Vorarbeit notwendig um die Tests
    auf zu setzen oder kann sofort mit dem Schreiben der Tests begonnen
    werden?
    \newline
    Genauso stellt sich die Frage, wie Effizient der Entwickler die Tests
    auswerten kann.
    
    \item \underline{Komplexität}:\newline
    Wie komplex ist das Tool? Das heißt, wie viel Funktionalität
    bietet das Tool dem Entwickler von Haus aus, aber auch wie schwer
    ist es einen Code zu schreiben oder wie schnell wird ein Code unübersichtlich, da
    das Tool viel Code abseits der Tests benötigt.
    
    \item \underline{Erweiterbarkeit}:\newline
    Wie leicht lässt sich das Tool mit anderen Tools erweitern?
    Gibt es vielleicht Erweiterungen der Community für dieses Tool, die sehr
    hilfreich sind?
\end{itemize}

Diese Aspekte eines jeden Tools, werden mithilfe von dem in Listing \ref{listing:base:my_module}
abgebildeten Code auf gezeigt. Dieser Code dient als das zu testende Modul.

Das Modul enthält eine selbst geschriebene Implementierung der Funktion \lstinline|pow()|,
welche eine Zahl \lstinline|a| mit einer Zahl \lstinline|b| quadriert. Um die Komplexität
zu erhöhen wurde eine in-memory Datenbank implementiert, welche Items enthält. Jedes Item 
hat eine ID (\lstinline|id|), einen Namen (\lstinline|name|), einen Lager Platz
(\lstinline|storage_location|) und eine Anzahl der vorrätigen Items \lstinline|amount|.
Jedes Item besitzt zudem eine Methode \lstinline|do_something()| welche eine externe
Funktion/Methode aufruft, die jedoch noch nicht existiert
\lstinline|do_something_which_does_not_exist()|.

% !TeX root = ../bachelor.tex
\subsection{Tools der Standard Bibliothek}\label{python-tools:stdlib}

Die Standard Bibliothek von Python bietet zwei verschiedene Test-Tools (\cite{wiki.python:PythonTestingToolsTaxonomy}).
Zum einen ist dies
\href{http://pyunit.sourceforge.net/pyunit.html}{unittest}\footnote{http://pyunit.sourceforge.net/pyunit.html}
und zum anderen
\href{https://docs.python.org/3/library/doctest.html}{doctest}\footnote{https://docs.python.org/3/library/doctest.html}.
Diese beiden Tools reichen sind im ihrem Umfang bereits so vielseitig dass, es einfach ist eine hohe Test-Abdeckung eines Programms oder eine Bibliothek zu erreichen.
\newline
\\
Beide Tools zählen zu den "`Unit Testing Tools"' (\cite{wiki.python:PythonTestingToolsTaxonomy})
auf Deutsch Modul Test-Tools, mit deren Hilfe die einzelnen Module eines Programms
getestet werden können. In einem Programm oder einer Bibliothek wären dies die einzelnen
Funktionen und Methoden.

% !TeX root = ../bachelor.tex
\subsubsection{unittest}\label{python-tools:unittest}

Das von JUnit inspirierte (\cite{docs.python:unittest}) Tool unittest, ist bestand
der Python Standardbibliothek und bietet seit jeher seinen Nutzern ein umfangreiches
Repertoire an Funktionen zum testen von Python Code.

Die Funktionen von unittest lassen sich unter folgenden Punkten beschreiben:
\begin{itemize}
    \item \Gls{fixture}, zum präparieren der Tests.
    \item Test Fälle, zum gliedern einzelner Tests.
    \item Testumgebungen, zum gliedern von zusammengehörigen Tests.
    \item Test runner, zum ausführen von Testumgebungen oder Test Fällen.
\end{itemize}

Mithilfe der genannten Punkte ist es dem Entwickler möglich eine Stabile Test Umgebung
auf zu bauen. Jedoch bietet unittest alleine nicht alles um TDD betreiben zu können.

Bei TDD werden zuerst die Tests und dann die Funktionalitäten geschrieben, daher
muss es möglich sein andere Module(units) zu \gls{mock}en auf denen ein Test basiert.
Durch die \Glspl{fixture} ist es bereits möglich den Test oder die Tests so vor zu
bereiten, dass diese funktionieren, jedoch bieten \Glspl{mock} einfachere und
schnellere Möglichkeiten Funktionen, Methoden, Klassen usw. zu imitieren.

Jedoch gibt es in der STDLIB eine Erweiterung zu unittest mit dem Namen
\lstinline|unittest.mock| welche unter eben diesem importiert werden kann um die \gls{mock}ing Funktionalität
zu bekommen.

Das Tool bietet des weiteren einen CLI, mit welchem es dem Benutzer möglich ist
seine Tests gebündelt aus zu führen und aus zu werten. Mit dem CLI ist es auch
auch möglich automatisch Tests in einem Ordner zu "`entdecken"'(discover) und
aus zu führen. Dadurch ist es sehr leicht neue Tests in ein bestehendes Test
System ein zu führen und diese ohne Veränderungen am bestehenden System aus zu
führen.
\newline
\\
Mit Hilfe von unittest lässt sich sehr einfach und schnell ein Test schreiben,
so würde das der Code aus Listing \ref{listing:unittest:example} bereits unsere
quadrat-Funktion testen, einmal mit unserem selbst berechneten Wert und
einmal gegen den wert der quadrat-Funktion aus der STDLIB.

\lstinputlisting[
    language=Python,
    label=listing:unittest:example,
    caption=unittest einfaches Beispiel
]{analyse/unittest_/example.py}

Die basis-Funktionalität von unittest ist schnell verstanden und setzt sich
eigentlich nur aus \lstinline|self.assert{irgendwas}(...)| zusammen. Hat man die
richtige assert Funktion gefunden lässt sich eigentlich jede unabhängige
Funktion testen.

Möchte man allerdings fortgeschrittenere Tests schreiben so muss man sich der
Dokumentation bedienen, welche unter \url{https://docs.python.org/3/library/unittest.html}
zu finden ist. Würde man die Seite als PDF herunterladen so wären dies 58
Seiten Fließtext. Möchte man nun zum Beispiel vor den Tests etwas vorbereiten
oder präparieren so lässt sich mit \lstinline|setUp()| und \lstinline|tearDown()| dies realisieren,
diese zwei Methoden überschreiben die Methoden aus \lstinline|unittest.TestCase| und
werden vor jeder Funktion ausgeführt. Das Gleiche gibt es auch für den Test Fall, bei dem
\lstinline|setUp()| und \lstinline|tearDown()| allerdings nur beim eintritt der Klasse und beim
austritt ausgeführt werden. Diese Methoden sind die sogenannten \Glspl{fixture}.

Das folgende Listing zeigt wie
% !TeX root = ../bachelor.tex
\subsubsection{doctest}\label{python-tools:doctest}

\begin{itemize}
    \item keine Abhängigkeiten, da STDLIB
    \item gleicht der Python shell
    \item integrierbar mit unittest
    \item Tests als Dokumentation
    \item Kein assert, Output wird überprüft
\end{itemize}

% !TeX root = ../bachelor.tex
\subsection{Tools abseits der Standard Bibliothek}\label{python-tools:extlib}
Abseits der Standard Bibliothek gibt es einige Tools deren Nutzung von
Vorteil gegenüber der STDLIB ist. Diese werden unter diesem Punkt aufgeführt.
\newline
\\
Auf \url{https://wiki.python.org/moin/PythonTestingToolsTaxonomy} werden viele
externe Tools gelistet, jedoch scheinen viele inaktiv zu sein da ihre
\glspl{commit} teilweise mehr als ein Jahr zurück liegen. Dies ist weder für
eine Bibliothek noch für ein Tool ein gutes Zeichen, da sich die Anforderungen
stetig ändern und niemals alle Bugs gefixt sind.
\newline
\\
Manche der dort aufgelisteten Tools sind bereits oder werden gerade in andere Tools
integriert. Dies kann zum einem sein, da ein Tool eine Erweiterung für ein anderes
war und die Entwickler die Änderungen angenommen haben und zum anderen um die Tools
zu verbessern und mehr Entwickler zur Verfügung zu haben. Eventuell sind auch
andere Gründe dafür verantwortlich, jedoch war dies der Grund bei zum Beispiel
\href{https://github.com/Yelp/Testify/}{Testify}
von Yelp \footnote{https://github.com/Yelp/Testify/}.
\newline
\\
Da sowohl Software als auch Programmiersprachen sich stetig weiter entwickeln
werden in dieser Arbeit nur jene Tools behandelt die sich diesen Entwicklungen
Anpassen, sei dies durch das unterstützen der aktuellsten Version von Python als
auch durch neue Innovationen, sowie \Gls{bug}-fixes. Aus diesem Grund werden
Tools deren letzter \Gls{commit} weiter als ein Jahr zurück liegt hier nicht
behandelt.
\newline
\\
Lässt man zusätzlich die Erweiterungen von Tool zunächst außen vor, so bleiben
folgende Modul Test-Tools zur Verfügung
(\cite{wiki.python:PythonTestingToolsTaxonomy}):
\begin{itemize}
    \item \href{https://github.com/pytest-dev/pytest/}{pytest}\footnote{https://github.com/pytest-dev/pytest/}
\end{itemize}
Tools wie \href{https://www.reahl.org/docs/4.0/devtools/tofu.d.html}{reahl.tofu}\footlabel{reahl.tofu}{https://www.reahl.org/docs/4.0/devtools/tofu.d.html} oder
\href{https://pypi.org/project/zope.testing/}{zope.testing}\footlabel{zope.testing}{https://pypi.org/project/zope.testing/} sind zwar mehr oder weniger aktiv, da sie
beide auf die neusten Python Versionen patchen, jedoch bieten sie sonst keinen Mehrwert in ihren Updates.
\href{https://www.reahl.org/docs/4.0/devtools/tofu.d.html}{reahl.tofu}\footref{reahl.tofu} selbst ist
auch nur eine Erweiterung für bestehende Test Tools, wie zum Beispiel pytest, weshalb es hier nicht behandelt wird. So wird
\href{https://pypi.org/project/zope.testing/}{zope.testing}\footref{zope.testing} auch nicht behandelt, da dieses Tool
wie bereits beschrieben nicht aktiv weiter entwickelt wird und es eher dafür gemacht wurde \href{http://www.zope.org/en/latest/}{zope}\footnote{http://www.zope.org/en/latest/}
Applikationen zu testen und nicht Python Applikationen.

Der Test-Runner \href{https://pypi.org/project/nose/1.3.7/}{nose}\footnote{https://pypi.org/project/nose/1.3.7/}, der eine Erweiterung
zu \nameref{python-tools:unittest} bietet ist nicht mehr in Entwicklung, dennoch haben sich ein paar Liebhaber des Tools zusammengeschlossen
und \href{https://pypi.org/project/nose2/}{nose2}\footlabel{nose2}{https://pypi.org/project/nose2/} geschrieben.
Da \href{https://pypi.org/project/nose2/}{nose2}\footref{nose2} wie sein Vorfahre eine Erweiterung zu \nameref{python-tools:unittest}
darstellt wird es hier nicht analysiert.

Auch sehr bekannt ist \href{https://pypi.org/project/testtools/}{testtools}\footnote{https://pypi.org/project/testtools/}, welches jedoch auch als
Erweiterung zu \nameref{python-tools:unittest} gesehen werden muss. Jedoch findet dort auch keine wirkliche weiter Entwicklung statt.

Die Quelle aus dem Offiziellen Python Wiki beschreibt sonst keine weiteren
Tools. Auch die suche auf einschlägigen Suchmaschinen liefert sonst keine
weiteren relevanten Tools.
\newline
\\
Als weitere Kategorie werden \Gls{mock}-Tools geführt. Auch wenn fast alle
unittest-Tools integriertes \gls{mock}ing haben, so lässt sich mit diesen
Tools meist mehr erreichen. Es wurden die gleichen Filter-Kriterien verwendet
wie bei den unit-testing Tools(\cite{wiki.python:PythonTestingToolsTaxonomy}).
\begin{itemize}
    \item \href{https://www.reahl.org/docs/4.0/devtools/stubble.d.html}{stubble}\footnote{https://www.reahl.org/docs/4.0/devtools/stubble.d.html}
    \item \href{https://github.com/timbertson/mocktest/tree/master}{mocktest}\footnote{https://github.com/timbertson/mocktest/tree/master}
    \item \href{https://github.com/bkabrda/flexmock}{flexmock}\footnote{https://github.com/bkabrda/flexmock}
    \item \href{https://bitbucket.org/DavidVilla/python-doublex}{python-doublex}\footnote{https://bitbucket.org/DavidVilla/python-doublex}
    \item \href{https://github.com/ionelmc/python-aspectlib}{python-aspectlib}\footnote{https://github.com/ionelmc/python-aspectlib}
\end{itemize}
Auch hier lassen sich sonst keine weiteren relevanten Tools finden.
\newline
\\
Als letzte Kategorie werden hier die \Gls{fuzz} Tools behandelt, da diese eine
gute Möglichkeit bieten Code ausgiebig zu testen. Das wohl umfangreichste und nach den obigen Kriterien
einzige Tool ist \href{https://github.com/HypothesisWorks/hypothesis}{hypothesis}\footnote{https://github.com/HypothesisWorks/hypothesis} (\cite{wiki.python:PythonTestingToolsTaxonomy}).

% !TeX root = ../bachelor.tex
\subsubsection{pytest}\label{python-tools:pytest}\mbox{}
\newline
Das am 04. August 2009 in Version
1.0.0\footnote{https://github.com/pytest-dev/pytest/releases/tag/1.0.0}
veröffentlichte Tool \lstinline{pytest} (auch py.test genannt)
ist ein sehr umfangreiches und weit entwickeltes Tool. Seit 2009 wird das Tool
stets weiter entwickelt und vorangetrieben, wodurch es eine Menge an Features
gewonnen hat.
\noindent
Die Basis Features von \lstinline{pytest} sind folgende(\cite{docs.pytest.org:4.4}):
\begin{itemize}
    \item Simple \lstinline{assert} statements.
    \subitem Kein \lstinline{self.assert}
    \subitem Error Überprüfung mit \Gls{context}
    \item Informativer Output in Farbe
    \subitem Der gesamte Output kann angepasst werden
    \item Feature reiche \Glspl{fixture}
    \subitem Vordefinierte \Glspl{fixture} von \lstinline{pytest}
    \subitem Geteilte \Glspl{fixture} unter Tests
    \subitem Globale \Glspl{fixture} zwischen Modulen
    \subitem Parametrisierung von Test als \Glspl{fixture}
    \item Überprüfung von \lstinline{stdout} und \lstinline{stderr}
    \item Gliedern von Test in Fällen
    \subitem ... durch Markierung
    \subitem ... durch Nodes (Auswahl der Modul Abhängigkeit)
    \subitem ... durch String Abgleich der Funktions-Namen
\end{itemize}
\noindent
Wie anhand der Basis Features erkennbar ist, bietet \lstinline{pytest} einiges
um Tests zu schreiben und aus zu führen. So ist mit den gebotenen
\Glspl{fixture} bereits eine Voraussetzung für TDD erfüllt, da
\lstinline{pytest} viele verschiedene Arten bietet diese zu benutzen.
\Glspl{fixture} werden in \lstinline{pytest} allerdings nicht mit
\lstinline{setUp} und \lstinline{tearDown} geschrieben, sondern werden mithilfe
eines \glspl{decorator} markiert und den Test Funktionen als Parameter
übergeben. Um die \lstinline{setUp} und \lstinline{tearDown} Funktionalität zu
bekommen muss lediglich das keyword \lstinline{yield} verwendet werden, die
\Gls{fixture} wird dann den Code bis zum \lstinline{yield} ausführen und nach
der Funktion den Rest nach \lstinline{yield} ausführen.
\newline
\\
Diese lassen sich durch zusätzliche Parameter weiter anpassen. Diese
Funktionen sind in Kapitel fünf nach \cite{docs.pytest.org:4.4} beschrieben.

Des weiteren bietet \lstinline{pytest} auch \gls{mock}ing, so lässt sich beispielsweise mit
\lstinline{monkeypatch.setattr()} der return Wert einer Funktion ersetzen. Dies wird
in \lstinline{pytest} automatisch am ende der Funktion rückgängig gemacht, wodurch der Entwickler
sich mehr auf das eigentliche Testen konzentrieren kann.
\newline
\\
Auch hier bietet \lstinline{pytest} weitere Möglichkeiten \Glspl{mock} zu verwenden, dazu hat
\cite{docs.pytest.org:4.4} in Kapitel sieben einige Worte geschrieben.

Ein weiteres sehr praktisches Feature von \lstinline{pytest} ist die Gliederung in Fälle. Dies ist bei \lstinline{pytest}
sehr fein einstellbar, so lässt sich wie in den Basis Features bereits beschrieben ein Test mit
einer oder mehreren Markierungen versehen wodurch eine Gliederung nach Markierung entsteht. Auch
durch das selektieren bestimmten Wörtern lassen sich Tests nach ihrem Namen gliedern, so entsteht
einerseits eine Gliederung zur Ausführung der Tests und andererseits eine Gliederung für die
Entwickler die sie selbst im Code sehen können. Als letzte alternative lassen sich Tests anhand
ihrer Module ausführen, dies geschieht durch die Angabe der Module. So würde
\lstinline{test_datei.py::TestKlasse::test_methode} den Test \lstinline{test_methode} der Klasse
\lstinline{TestKlasse} in der Datei \lstinline{test_datei.py} ausführen.
\newline
\\
Diese Features werden in Kapitel sechs von \cite{docs.pytest.org:4.4} beschrieben.

Selbstverständlich bietet \lstinline{pytest} noch weitere Features jedoch sind diese nicht zwangsläufig notwendig
um TDD zu betreiben und sind mehr ein nice to have Feature als wirklich benötigt. Für eine
Vollständige Auflistung und Erklärung aller Features kann jederzeit unter
\url{https://docs.pytest.org/en/latest/} die aktuellste Version der Dokumentation abgefragt werden.

Da \lstinline{pytest} nicht in der STDLIB ist, muss es mit einem \Gls{paket} installiert werden.
Dabei werden für \lstinline{pytest 4.4.0} die in Listing \ref{listing:pytest:requirements}
gezeigten Abhängigkeiten installiert. Demnach benötigt \lstinline{pytest} sechs externe Abhängigkeiten
zusätzlich zu sich selbst.
Demnach bietet \lstinline{pytest} alles um TDD anwenden zu können und die sieben zusätzlichen Abhängigkeiten
die ein Entwickler bei der Nutzung von \lstinline{pytest} eingeht sollten keinen Entwickler davon abhalten
\lstinline{pytest} und seine Features zu genießen.
\newline
\\
Da \lstinline{pytest}, wie bereits erwähnt keine neuen assert Methoden hinzufügt lässt sich sehr schnell und
einfach ein Test schreiben. Selbst die Nutzung von \Glspl{fixture} ist in \lstinline{pytest} sehr einfach, da
lediglich eine Funktion geschrieben werden muss die mit \lstinline{@pytest.fixture} markiert wurde
und der Test Funktion oder Methode als Parameter übergeben wird. Den Rest erledigt \lstinline{pytest} selbst
im Hintergrund. Genauso einfach gestaltet sich die Nutzung von \Glspl{mock}.

Die Effizienz, die bei der Nutzung von \lstinline{pytest} entsteht ist demnach sehr hoch. Denn der Entwickler
muss nicht wirklich etwas neues dazu lernen um Tests zu schreiben oder zu präparieren. Genauso
leicht kann ein Entwickler den Output von \lstinline{pytest} auswerten, da dieser erstens, in Farbe ist, was
die Lesbarkeit deutlich erhöht, zweitens sehr gut gegliedert ist und Wichtige Strukturen klar
darstellt und drittens nach den Wünschen der Entwickler sich gestalten und verbessern lässt.
\newline
\\
Durch die eben genannten Features im Bezug auf das schreiben von Tests mit assert, sowie
\Glspl{fixture} und \Glspl{mock} lässt sich sagen das \lstinline{pytest} sehr viel Funktionalität bietet
wobei es trotzdem Struktur im Code bietet und komplexe Features einfach ermöglicht. Demnach
kann ein Entwickler mit wenig Code viel Testing Funktionalität erstellen.
\newline
\\
Sollten dem Entwickler die Features von \lstinline{pytest} nicht ausreichen, so findet man unter
\url{http://plugincompat.herokuapp.com/} eine Liste von 618 (Stand: 2. April 2019) Erweiterungen
für \lstinline{pytest 4.3.0}. \lstinline{pytest} selbst kann allerdings auch als Erweiterung zu unittest
genutzt werden, indem ein Entwickler \lstinline{pytest} zum ausführen der unittests verwendet. Dadurch bietet
sich dem Entwickler ein verbesserter Output des Test Ergebnisses.
\newline
\\
Im Listing \ref{listing:pytest:advanced} und \ref{listing:pytest:advanced_output} befindet sich der
Code und der Output zum Test von dem in Listing \ref{listing:base:my_module} definierten Modul.
% !TeX root = ../bachelor.tex
\subsubsection{Mocking Tools}\label{python-tools:extlib:mock}

Blubb

\input{bachelor/3_2_2_1_stubble}
\input{bachelor/3_2_2_2_mocktest}
\input{bachelor/3_2_2_3_flexmock}
\input{bachelor/3_2_2_4_python_doublex}
\input{bachelor/3_2_2_5_python_aspectlib}
% !TeX root = ../bachelor.tex
\subsubsection{Fuzz-testing Tools}\label{python-tools:extlib:fuzz}

Zusätzlich zu den hier behandelten Test runnern und den \gls{mock}ing Tools 
werden Fuzz-testing Tools behandelt. Unter Fuzz-testing 
versteht man das Testen eines Programms oder einem Modul mit zufälligen Werten. 
Diese Werte können komplett zufällig sein, aber auch einem gewissen Format 
oder einem Typ entsprechen.

Fuzz-testing funktioniert dabei anders als die gewohnten Methoden des Testens. 
Normalerweise werden im Test Daten präpariert, die im Test ausgeführt und 
danach validiert werden, sodass das Ergebnis richtig ist. Mit Fuzz-testing 
sieht das ganze etwas anders aus: zuerst wird festgelegt, welchem Schema die 
Daten entsprechen müssen (Beispiel: Es wird eine IP erwartet, 
\lstinline|[1-9]{1,3}.[1-9]{1,3}.[1-9]{1,3}.[1-9]{1,3}|), danach wird mit den 
Daten der Test ausgeführt und schlussendlich validiert 
(\cite{hypothesis:doc:4.18.0}).

Der Unterschied ist, dass der Entwickler sich selbst keine Daten ausdenken 
muss. Im genannten Beispiel müsste der Entwickler sich verschiedene IP Adressen 
ausdenken, die entweder richtig oder falsch sind und danach diese überprüfen. 
Fuzz-testing übernimmt das Ausdenken der Daten und sorgt dafür, dass diese der 
Spezifikation entsprechen. So kann eine viel größere Anzahl an Tests mit 
erheblich weniger Aufwand auf einem Modul oder Programm ausgeführt werden.

Schlägt ein Test mit Daten, die per  Spezifikation richtig sind, fehl, so wird 
der Wert der Daten für später gespeichert und, sofern möglich, versucht mit 
ähnlichen Daten dieser Fehler zu erreichen. Dadurch hat ein Entwickler am Ende 
der Tests Daten, die fehlgeschlagen sind. Mit diesen kann er die weiteren Tests 
befüllen um so  zu verhindern, dass diese Daten zu Fehlern führen. Im 
Normalfall übernimmt dies allerdings das Tool, wodurch der Entwickler sich nur 
darum kümmern muss, dass die Tests erfolgreich verlaufen.

Von Zeit zu Zeit entsteht dadurch ein Katalog an Daten für verschiedene Tests, 
der vorgibt bei welchen Daten ein Test fehl geschlagen ist. Dadurch wird 
verhindert, dass bei der Änderung des Codes dieser Fehler wieder auftritt. 
Führt man  dies über den gesamten Entwicklungsprozess hinweg durch, so erhält 
man am Ende ein sehr ausgiebig getestetes Programm.

Diese Methode des Testens wird auch "`property based testing"', also 
Eigenschaftsbasierte Tests genannt. In dieser Arbeit, wird allerdings weiterhin 
der englische Begriff Fuzztesting genutzt.

\input{bachelor/3_2_3_hypothesis}



    % !TeX root = ../bachelor.tex
\section{Diskussion}\label{diskussion}
Um für die Anwendung von TDD die passenden Tools zu finden, wurden verschiedene
Tool der Programmiersprache Python analysiert und auf ihre Tauglichkeit im
Bezug auf TDD geprüft. Daraus entstand eine Empfehlung für vier Tools die
gemeinsam für TDD genutzt werden können, um so die optimale Funktionalität zum
schreiben von Tests, sowie zu einer optimalen Testabdeckung zu gewähren. Die 
dabei erhaltenen Ergebnisse zeigen auf, dass ein Tool alleine zwar ausreichend 
sein kann für die Anwendung von TDD, jedoch ist die Verwendung von mehreren 
Tools zusammen für den Entwickler und für das Produkt besser.

Im Rahmen dieser Arbeit wurden lediglich Tools analysiert, deren letzte
Aktivität nicht weiter als ein Jahr zurück liegt. Tools die eine Erweiterung zu
einem anderen Tool darstellen wurden nicht behandelt, da dies den Rahmen der
Arbeit gesprengt hätte. Aus diesem Grund kann es sein, dass ein Beliebtes oder
Bekanntes Tool in dieser Arbeit nicht behandelt wurde. Das bedeutet nicht, dass
diese Tool zwangsläufig nicht geeignet ist für TDD. Die Ergebnisse dieser
Arbeit sind also nicht repräsentativ für alle Tools die es für die
Programmiersprache Python gibt und würden eventuell anders ausfallen, wenn
andere Tools behandelt worden würden.

Im Bezug auf die Forschungsfrage, welches aktuelle Python Test-Tool das beste 
ist, lässt sich folgendes sagen: Viele Tools sind in ihrem Funktionsumfang sehr 
ähnlich, allerdings stellten sich \lstinline{pytest} bei den unit-testing-, 
\lstinline{mocktest} bei den mock-testing- und \lstinline{hypothesis} bei den 
fuzz-testing Tools als die beste Wahl heraus. Dennoch ergab sich, dass 
\lstinline{doctest} immer als ergänzendes Tool verwendet werden kann, egal 
welche anderen Tools verwendet werden.

Diese Arbeit hat sich jedoch lediglich mit einer begrenzten Anzahl von Tools 
beschäftigt und hat diese Anhand eines Spezifischen Beispiels getestet. Jeder 
Entwickler oder Software-Architekt sollte sich vorher noch einmal genaustens 
darüber informieren welches Tool für den geplanten Anwendungszweck das beste 
ist, da manche Tools abhängig ihrer Anwendung vielleicht eine bessere Wahl 
darstellen.
    % !TeX root = ../bachelor.tex
\section{Fazit}\label{fazit}

FAZIT

    % !TeX root = ../bachelor.tex
\section{Nachwort}\label{nachwort}

Nachwort


    \onecolumn

    % einfacher Zeilenabstand
    \singlespacing

    % Literaturverzeichnis
    % !TeX root = ../bachelor.tex
\newpage
% Literaturverzeichnis soll im Inhaltsverzeichnis auftauchen
\phantomsection
\addcontentsline{toc}{subsection}{Literaturverzeichnis}
\fancyhead[L]{Literaturverzeichnis} % Kopfzeile links
% Literaturverzeichnis endgültig anzeigen
\renewcommand\refname{Literaturverzeichnis}
\printbibliography
    
    % die Listings
    % !TeX root = ../bachelor.tex
\lstinputlisting[
    language=Python,
    label=listing:base:my_module,
    caption=Basis Modul zum testen,
    lastline=42
]{analyse/my_package/my_module.py}
\newpage

\lstinputlisting[
    language=Python,
    label=listing:unittest:example,
    caption=unittest einfaches Beispiel
]{analyse/unittest_/example.py}

\lstinputlisting[
    label=listing:unittest:example_output_success,
    caption=unittest einfaches Beispiel: Output erfolgreich
]{analyse/unittest_/example_output_success.txt}

\lstinputlisting[
    label=listing:unittest:example_output_failure,
    caption=unittest einfaches Beispiel: Output misslungen
]{analyse/unittest_/example_output_failure.txt}
\newpage

\lstinputlisting[
    language=Python,
    label=listing:unittest:advanced,
    caption=unittest Basis Features
]{analyse/unittest_/advanced_example.py}

    % 1,5 facher Zeilenabstand
    \onehalfspacing

    % Eidesstattliche Erklärung
    % !TeX root = bachelor.tex
\newpage

\section*{Eidesstattliche Erklärung}
\thispagestyle{empty}
\addcontentsline{toc}{section}{Eidesstattliche Erklärung}
\fancyhead[L]{Eidesstattliche Erklärung} % Kopfzeile links

\begin{verbatim}

\end{verbatim}

\begin{LARGE}
\begin{center}
	Eidesstattliche Erklärung zur Abschlussarbeit
\end{center}
\end{LARGE}
\begin{verbatim}


\end{verbatim}
Hiermit versichere ich, die eingereichte Abschlussarbeit selbständig verfasst und
keine andere als die von mir angegebenen Quellen und Hilfsmittel benutzt zu haben.
Wörtlich oder inhaltlich verwendete Quellen wurden entsprechend den anerkannten
Regeln wissenschaftlichen Arbeitens zitiert. Ich erkläre weiterhin, dass die
vorliegende Arbeit noch nicht anderweitig als Abschlussarbeit eingereicht wurde.
\\
\\
Das Merkblatt zum Täuschungsverbot im Prüfungsverfahren der Hochschule
Augsburg habe ich gelesen und zur Kenntnis genommen. Ich versichere, dass die
von mir abgegebene Arbeit keinerlei Plagiate, Texte oder Bilder umfasst, die durch
von mir beauftragte Dritte erstellt wurden.
\begin{verbatim}



\end{verbatim}


\begin{displaymath}
% use packages: array
\begin{array}{ll}
Unterschrift:~~~~~~~~~~~~~~~~~~~~~~~~~~~~~~~~~~~~~~~~~~
& Ort, Datum:~~~~~~~~~~~~~~~~~~~~~~~~~~~~~~~~~~~~~~~~~~
\end{array}
\end{displaymath}
\end{document}

% Indexerstellung
\makeindex
