% !TeX root = expose.tex
\section{Ausgangslage}
TDD wird in der heutigen Softwareentwicklung immer verbreiteter und beliebter.
Die Ansprüche an Software sind in den letzten Jahren immer weiter gestiegen, dies
liegt vor allem an der Reichweite die heute Software hat. Circa 66\% aller Menschen
besitzen heute ein Smartphone \cite{FraukeSchobelt:Smartphone}, mit steigender
Nutzerzahlen steigen auch die Anforderungen die, die Nutzer an die Software stellen
und auch die Anzahl der gefundenen Bugs wird dementsprechend größer.



\section{Problembeschreibung}
Im Weltweiten Markt gibt es viele große unternehmen die gegenseitig um die
Nutzer kämpfen. Selbstverständlich gehen die Nutzer zu dem Anbieter der die bessere
Software bietet, doch das kann sich heute stetig ändern. Mit der steigenden Anzahl
an Bugs die gefunden werden, steigen auch die Nutzer die von diesen Bugs betroffen
sind. Diese Bugs sollen natürlich schnellstmöglich gefixt werden um so zu verhindern
das Nutzer die Software wechseln.

So schwer es ist seine Nutzer zu halten, umso schwerer ist es zum Start einer Software
Nutzer zu akquirieren. Gibt es bereits andere Software die ähnliche Services anbietet,
so ist es noch schwerer in den Markt rein zu kommen. Die Anforderungen sind durch die
anderen Firmen höher als die Anforderungen an ein neues Produkt und alte Fehler die
in anderer Software schon gelöst wurden sollten möglichst nicht auftauchen.

\section{Fragestellung}
Wie schaffen es Firmen Ihre Entwicklung so zu optimieren dass, ihre Software zu jeder
Zeit optimal Funktioniert und alte Fehler und Bugs nicht wieder auftauchen? Am besten
sollte dies von Anfang an im Projekt der Fall sein und sich bis zur Wartung der Software
durchziehen.
