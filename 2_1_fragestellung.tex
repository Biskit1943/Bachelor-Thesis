% !TeX root = expose.tex
\section{Ausgangslage}
TDD wird in der heutigen Softwareentwicklung immer verbreiteter und beliebter.
Die Ansprüche an Software sind in den letzten Jahren immer weiter gestiegen. Dies
liegt vor allem an der Reichweite, die Software heute hat. So besitzen im Jahr
2018 bereits circa 66\% aller Menschen ein Smartphone \cite{FraukeSchobelt:Smartphone},
im Arbeitsleben ist ein PC meist gar nicht mehr weg zu denken. Doch mit den steigenden
Nutzerzahlen steigen auch die Anforderungen, die die Nutzer an die Software stellen.
Somit wird die Anzahl der gefundenen Bugs dementsprechend größer.

\section{Problembeschreibung}
Im weltweiten Markt gibt es viele große Unternehmen, die gegenseitig um die
Nutzer kämpfen. Selbstverständlich präferieren die Nutzer denjenigen Anbieter, welcher die bessere
Software bietet. Dies kann sich heute jedoch stetig ändern. Mit der steigenden Anzahl
an Bugs, die gefunden werden, steigt auch die Anzahl der Nutzer, die von diesen Bugs betroffen
sind. Diese Bugs sollen natürlich schnellstmöglich gefixt werden um so zu verhindern,
dass die Nutzer die Software wechseln.

So schwer es ist seine Nutzer zu halten, umso schwerer ist es zum Start einer Software
Nutzer zu akquirieren. Es gibt bereits andere Software, die ähnliche Services anbieten.
So ist es noch schwerer dem Markt beizutreten. Die Anforderungen sind durch die
anderen Firmen höher als die Anforderungen an ein neues Produkt und alte Fehler, die
in anderer Software schon gelöst wurden, sollten möglichst nicht auftauchen.

Für Unternehmen sind diese Anforderungen meist schwer zu meistern weshalb Software
meist mit Fehlern released wird, um diese dann von den Nutzern aufdecken zu lassen und
zu fixen.

\section{Fragestellung}
Welche Möglichkeiten stehen einem Entwickler/Arbeitgeber zur Verfügung Python
als Sprache für TDD einzusetzen? Daraus resultierend stellt sich die Frage
welches der Test-Tools das beste dafür ist, oder welche sich kombinieren lassen
um das beste aus ihnen heraus zu holen?

Welchen Vorteil zieht ein Entwickle/Arbeitgeber daraus TDD zu verwenden? Ist
der betriebene aufwand und die daraus resultierende Test-Abdeckung es wert TDD
anderen Techniken vorzuziehen?
