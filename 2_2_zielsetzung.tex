% !TeX root = expose.tex
\section{Zielsetzung}
Das Ziel dieser Arbeit ist es, Test-Tools der Programmiersprache Python in den
Aspekten Anwendbarkeit, Effizienz, Komplexität und Erweiterbarkeit zu
vergleichen. Daraus resultierend soll eine Empfehlung entstehen wie man
möglichst einfach TDD mit Python betreiben kann, um so die Motivation Tests zu
schreiben zu steigern.

Um die effizient von TDD zu steigern wird in dieser Arbeit auf die Automation
der Tests eingegangen, dies wird mithilfe des Tools
Travis CI\footnote{\href{https://travis-ci.org/}{Travis CI Website}}
darzustellen.

Nach dem Vergleich soll der Leser ein Bild darüber haben welche Tools ihm zur
Verfügung stehen um Tests zu schreiben oder TDD zu betreiben. Zusätzlich soll
er ein wenig Verständnis darüber haben wie er mithilfe von
\gls{Continous testing} effektiver TDD betreiben kann.
\newline
Am Ende der Arbeit werden die Vor- und Nachteile von TDD abgewogen, für die
Nachteile soll eine Lösung angeboten werden (soweit dies möglich ist) um TDD
attraktiver zu gestalten.
